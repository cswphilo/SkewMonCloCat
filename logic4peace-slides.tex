\documentclass[10pt,t]{beamer}

\usepackage{etex}

\mode<presentation>
{
%  \usetheme{Goettingen}
  \useoutertheme{default}
  \useinnertheme{rounded}
  \setbeamercovered{transparent}
  \usecolortheme{rose}
}

\usepackage{amssymb,amsmath}
\usepackage{proof}
\usepackage{stmaryrd}
\usepackage[all]{xy}
\renewcommand{\arraystretch}{1}
%\usepackage{txfonts}

%\usepackage{ulem}

\usepackage{mathtools}

\newcommand{\mathsout}[1]% will draw line through middle of #1
{\bgroup\mathchoice
  {\sbox0{$\displaystyle{#1}$}%
    \usebox0\hspace{-\wd0}%
    \rule[0.5\ht0-0.5\dp0-.5pt]{\wd0}{1pt}}%
  {\sbox0{$\textstyle{#1}$}%
    \usebox0\hspace{-\wd0}%
    \rule[0.5\ht0-0.5\dp0-.5pt]{\wd0}{1pt}}%
  {\sbox0{$\scriptstyle{#1}$}%
    \usebox0\hspace{-\wd0}%
    \rule[0.5\ht0-0.5\dp0-.5pt]{\wd0}{1pt}}%
  {\sbox0{$\scriptscriptstyle{#1}$}%
    \usebox0\hspace{-\wd0}%
    \rule[0.5\ht0-0.5\dp0-.5pt]{\wd0}{1pt}}%
\egroup}

% ==========================================================================

\newcommand{\gray}[1]{\textcolor{gray}{#1}}
\newcommand{\red}[1]{\textcolor{red}{#1}}
\colorlet{mygreen}{green!70!black}
\newcommand{\green}[1]{\textcolor{mygreen}{#1}}

\newcommand{\Set}{\mathbf{Set}}

\newcommand{\eqdf}{\mathbin{=_{\mathrm{df}}}}

\newcommand{\I}{\mathsf{I}}
\newcommand{\ot}{\otimes}
\newcommand{\C}{\mathbb{C}}
\newcommand{\Cop}{\mathbb{C}^\mathrm{op}}
\newcommand{\J}{\mathbb{J}}
\newcommand{\al}{\alpha}
\newcommand{\lam}{\lambda}
\newcommand{\n}{{-}}
\newcommand{\Var}{\mathsf{At}}
\renewcommand{\v}{`}
\newcommand{\tto}{\Longrightarrow}
\newcommand{\Tm}{\mathsf{Fma}}
\newcommand{\id}{\mathsf{id}}
\newcommand{\comp}{\circ}
\newcommand{\dcomp}{\mathsf{comp}}

\newcommand{\plug}{\circ_0}

\newcommand{\ax}{\mathsf{ax}}
\newcommand{\uf}{\mathsf{pass}}
\newcommand{\IL}{\I\mathsf{L}}
\newcommand{\otL}{\ot\mathsf{L}}
\newcommand{\IR}{\I\mathsf{R}}
\newcommand{\otR}{\ot\mathsf{R}}

\newcommand{\loL}{{\lo}\mathsf{L}}
\newcommand{\loC}{{\lo}\mathsf{C}}
\newcommand{\loR}{{\lo}\mathsf{R}}

%\newcommand{\act}{\mathsf{act}}
%\newcommand{\IC}{\I\mathsf{Lc}}
%\newcommand{\otC}{\ot\mathsf{Lc}}

\newcommand{\ILinv}{\IL^{-1}}
\newcommand{\otLinv}{\otL^{-1}}

%\newcommand{\otLstar}{\otL^\star}
\newcommand{\otLinvstar}{\mathsf{L}^{-1}}

\newcommand{\cut}{\mathsf{cut}}
\newcommand{\scut}{\mathsf{scut}}
\newcommand{\ccut}{\mathsf{ccut}}

\newcommand{\wk}{\mathsf{wk}}
\newcommand{\ctr}{\mathsf{ctr}}
\newcommand{\exch}{\mathsf{exch}}


\newcommand{\sound}{\mathsf{sound}}
\newcommand{\cmplt}{\mathsf{cmplt}}
\newcommand{\strcmplt}{\mathsf{strcmplt}}

\newcommand{\scomp}{\mathsf{ssubst}}
\newcommand{\ccomp}{\mathsf{csubst}}

\renewcommand{\vdash}{\longrightarrow}
\newcommand{\vdashL}{\vdash_\mathsf{L}}
\newcommand{\vdashR}{\vdash_\mathsf{R}}
\newcommand{\msw}{\mathsf{switch}}
\newcommand{\mswLC}{\msw_{\mathsf{LC}}}
\newcommand{\mswRL}{\msw_{\mathsf{RL}}}
\newcommand{\mswat}{\msw\mathsf{at}}
\newcommand{\mswnot}{\msw\mathsf{not}}
\newcommand{\mswatRL}{\msw\mathsf{at}_{\mathsf{RL}}}
\newcommand{\mswnotRL}{\msw\mathsf{not}_{\mathsf{RL}}}
\newcommand{\switch}{\mathsf{switch}}
\newcommand{\focus}{\mathsf{focus}}
\newcommand{\emb}{\mathsf{emb}}

\newcommand{\focderivs}{\mathsf{focderivs}}

\newcommand{\mcutj}{\mathsf{mcut}^\mathsf{j}}
\newcommand{\mcutn}{\mathsf{mcut}^\mathsf{n}}

\newcommand{\ufq}{\uf^?}


%\newcommand{\proofbox}[1]{\begin{tabular}{c} #1 \end{tabular}}

\newcommand{\List}{\mathsf{List}}

\newcommand{\lb}{\langle\hspace*{-0.9mm}\langle}

\newcommand{\asem}[1]{\llbracket #1 \rrbracket}
\newcommand{\csem}[2]{#1\, \lb #2 \rrbracket}
\newcommand{\ssem}[1]{\llbracket #1 \lb\, }

\newcommand{\dotminus}{\stackrel{.}{-}}
\newcommand{\Ptd}{\mathbf{Ptd}}

\newcommand{\inl}{\mathsf{inl}}
\newcommand{\inr}{\mathsf{inr}}

\newcommand{\e}{\mathsf{e}}
\newcommand{\m}{\mathsf{m}}
\renewcommand{\c}{\mathsf{c}}
\newcommand{\cst}{\mathsf{cst}}
\newcommand{\cstR}{\mathsf{cstR}}
\newcommand{\JJ}{\mathcal{J}}

\newcommand{\Lan}{\mathrm{Lan}}

\newcommand{\SC}{\mathbf{SC}}
\newcommand{\SCn}{\mathbf{SC}^\mathsf{n}}

\newcommand{\Fsk}{\mathbf{Fsk}}
\newcommand{\con}[1]{[ #1 ]}

\newcommand\stseq[3]{#1 \mid #2 \vdash #3}
\newcommand\irr{{\mathrm{irr}}}

%\renewcommand{\arraystretch}{2}

\newcommand{\lo}{\multimap}
\newcommand{\ol}{\multimapinv}


\newcommand\stseqL[4][]{#2 \mid #3 \overset{#1}{\longrightarrow_{\mathsf{L}}} #4}
\newcommand\stseqR[4][]{#2 \mid #3 \overset{#1}{\longrightarrow_{\mathsf{R}}} #4}
\newcommand\stseqLR[4]{#1 \mid #2 \longrightarrow_{#4} #3}
\newcommand{\dottedmid}{\raisebox{-1.5pt}[0.5\height]{\hspace*{1pt}\vdots\hspace*{1pt}}}
\newcommand\stseqC[5][]{#2 \mid #3 \mathbin{\dottedmid} #4 \overset{#1}{\longrightarrow_{\mathsf{C}}} #5}

\newcommand\stseqLN[2]{#1 \longrightarrow_{\mathsf{L}} #2}
\newcommand\stseqRN[2]{#1 \longrightarrow_{\mathsf{R}} #2}


\newcommand{\laminv}{\lam^{-1}}
\newcommand{\alinv}{\al^{-1}}
\newcommand{\rhoinv}{\rho^{-1}}
\newcommand{\alinvc}{\al^{-1}_{\mathrm{c}}}

\newcommand\otLcxt[1]{\ot\mathsf{C}_{#1}}
\newcommand\IC[1]{\mathsf{IC}_{#1}}
\newcommand\JJC[1]{\ot\mathsf{C}^\mathrm{c}_{#1}}
\newcommand\otC[1]{\ot\mathsf{C}_{#1}}
\newcommand\otRem{\otR_2}
\newcommand\ILgeninv{\ILgen^{-1}}
\newcommand\otLgeninv{\otLgen^{-1}}
\newcommand\act{\mathsf{move}}

\newcommand\ICn[1]{\mathsf{IC}^{\mathrm{c}}_{#1}}

\newcommand{\iR}{{i^{\mathsf{R}}}}
\newcommand{\jR}{{j^{\mathsf{R}}}}
\newcommand{\LR}{{L^{\mathsf{R}}}}

\newcommand{\jY}{\widehat{j}}
%\newcommand{\jY}{\widehat{\jmath}}
\newcommand{\iY}{\widehat{\imath}}
\newcommand{\LY}{\widehat{L}}
%\newcommand{\jY}{j\mathsf{Y}}
%\newcommand{\LY}{L\mathsf{Y}}

\newcommand{\lolli}{\lo}

\newcommand{\proofbox}[1]{\begin{tabular}{l} #1 \end{tabular}}

\newcommand{\LI}{\mathsf{LI}}
\newcommand{\F}{\mathsf{F}}
\newcommand{\Pass}{\mathsf{P}}
\newcommand{\RI}{\mathsf{RI}}

% ==========================================================================


\begin{document}

\title{Logics of skew categorical structures}

\author{\underline{Tarmo Uustalu}, Reykjavik U. \\ Niccol\`o Veltri, Tallinn U. of Technology \\ Cheng-Syuan Wan, Tallinn U. of Technology}

%\date{MFPS 2018, Halifax, NS, 6--9 June 2018}
%\date{Kalle Kaarli 70, Tartu, 18--20 July 2018}
%\date{Logic and Philos. of Math. Seminar, Univ. of Helsinki, 7 Sept. 2018}
%\date{ICE-TCS seminar, Reykjavik U., 24 Feb.\ 2020}

%\date{CS Colloquium, Augusta Univ., 26 Feb.\ 2021}
%\date{CMAT Colloquium, Univ. do Minho, 25 March 2021}

\date{Logic4Peace, 23 Apr.\ 2022}

\begin{frame}

\maketitle

\end{frame}


\begin{frame}

\frametitle{Logics vs categorical structures}

\begin{itemize}

\item There is a correspondence between \emph{logics} and
  \emph{categorical structures}, first noticed by Lambek, then further
  developed by Lawvere, Mann, Szabo, Mints, Soloviev, Dosen and
  Petri\'c et al.

\bigskip

\begin{center}
\scriptsize
\begin{tabular}{l|l}
(conj-impl) intuit logic & Cartesian closed categories \\ \hline
intuit logic & Cartesian closed categories \\ 
  & \quad with finite coproducts \\ \hline
intuit S4 & Cartesian closed categories \\ 
& \quad with a lax monoidal comonad \\ \hline
mult intuit linear logic & symm monoidal closed categories \\ \hline
noncomm mult intuit linear logic & monoidal closed categories \\ \hline
Lambek calculus & monoidal biclosed categories 
\end{tabular}
\end{center}

\bigskip

\item This is similar to the algebraic logic correspondence of logics
  and \emph{algebraic structures} as in algebraic logic, but
  proof-relevant.

\bigskip

\item Categorical logic equips a logic with notions of
  \emph{derivation} (as a opposed to just \emph{consequence}) and
  \emph{identity of derivations}.

\end{itemize}

\end{frame}


\begin{frame}

\frametitle{Skew structured categories}

\begin{itemize}

\item Intuitionistic linear logic (for monoidal categories) drops the
  structural rules of weakening and contraction of intuitionistic
  logic:
\[
\scriptsize
\begin{array}{c}
\mathsout{w : A \tto \I}
\\
\mathsout{c : A \tto A \otimes A}
\end{array}
\]
It is therefore called \emph{substructural} and can be thought of as a
\emph{resource} logic rather than a truth logic.

\bigskip

\item Recent years have seen the discovery and study of \emph{skew}
  monoidal, \emph{skew} closed and related kinds of structured
  categories by Szlach\'anyi, Street, Bourke, Lack, others.

\item These drop one half of unitality and associativity of
  conjunction:
\[
\scriptsize
\begin{array}{cc}
\lam: \I \ot A \tto A 
& 
\mathsout{\lam^{-1} : A \tto \I \otimes A} \\
\rho: A \tto A \ot \I 
& 
\mathsout{\rho^{-1} : A \ot \I \tto A} \\
\qquad
\al : (A \ot B) \ot C \tto A \ot (B \ot C) 
& 
\mathsout{\al^{-1} : A \ot (B \ot C) \tto (A \ot B) \ot C}
\end{array}
\]

\item Skew structured categories define logics yet more substructural
  than intuitionistic linear logic.

\end{itemize}

\end{frame}

\begin{frame}

\frametitle{This talk: Skew categorical logic}

\begin{itemize}

\item We have been developing the proof theory of skew structured
  categories.

\bigskip

\item This talk: 

\begin{itemize} 
\item skew monoidal categories (U., V., Zeilberger, MFPS 2018) 
\item skew monoidal closed categories (U., V., W., NCL 2022)
\end{itemize}

\bigskip

\item Other work:

\begin{itemize}
\item partially normal skew monoidal categories \\ (U., V., Zeilberger, ACT 2020)
\item skew closed and skew prounital closed categories \\
  (including natural deduction) \\ (U., V., Zeilberger,
  LFMTP 2020)
\item symmetric skew monoidal categories (V., WoLLIC 2021)
\end{itemize}

\item In progress or stuck:

\begin{itemize}
\item Cartesian skew monoidal categories
\item skew biclosed categories
\end{itemize}

\end{itemize}

\end{frame}


\begin{frame}

\frametitle{Monoidal categories}

\begin{itemize}

\item A \emph{monoidal category} (B\'enabou, Mac Lane) is a category $\C$
  together with an object $\I$, a functor $\ot : \C \times \C \to \C$
  and \red{nat.\ isomorphisms} $\lam$, $\rho$, $\al$ with components
\[
\scriptsize
\begin{array}{c}
\lam_{A} : \I \ot A \to A \\
\rho_{A} : A \to A \ot \I\\
\al_{A,B,C} : (A \ot B) \ot C \to A \ot (B \ot C)
\end{array}
\]
such that
\[
\scriptsize
\begin{array}{c}
\mathrm{(m1)}
\xymatrix@R=1.2pc@C=0.2pc{
    & \I \ot \I \ar[dr]^-{\lambda_\I} & \\
    \I \ar[ur]^-{\rho_\I} \ar@{=}[rr] & & \I
    }
\quad
\mathrm{(m2)}
\xymatrix@R=1.1pc{
      (A\ot \I) \ot B \ar[r]^{\alpha_{A,\I,B}}
      & A \ot (\I\ot B) \ar[d]^{A\ot \lambda_{B}}\\
      A \ot B \ar@{=}[r] \ar[u]^{\rho_{A}\ot B}&  A \ot B 
    }
\\
\mathrm{(m3)}
\xymatrix@C=0.2pc@R=1.2pc{
  (\I \ot A) \ot B \ar[dr]_{\lambda_A \ot B} \ar[rr]^{\alpha_{\I,A,B}} 
           & &  \I \ot (A \ot B) \ar[dl]^{\lambda_{A \ot B}}\\
  & A \ot B & 
    }
\quad
\mathrm{(m4)}
\xymatrix@C=0.2pc@R=1.3pc{
  (A \ot B) \ot \I \ar[rr]^{\alpha_{A,B,\I}} 
           & &  A \ot (B \ot \I) \\
  & A \ot B \ar[ul]^{\rho_{A \ot B}} \ar[ur]_{A \ot \rho_B} & 
    }
\\
\mathrm{(m5)}
\xymatrix@R=1.3pc@C=2.5pc{
(A\ot (B \ot C)) \ot D \ar[rr]^{\alpha_{A,B \ot C,D}}
  & & A\ot ((B \ot C)\ot D) \ar[d]^{A\ot \alpha_{B,C,D}}
  \\
((A\ot B) \ot C) \ot D \ar[u]^{\alpha_{A,B,C} \ot D}
      \ar[r]^{\alpha_{A\ot B,C,D}}
  & (A\ot B) \ot (C \ot D) \ar[r]^{\alpha_{A,B,C\ot D}}
    & A\ot (B \ot (C \ot D))
}
\end{array}
\]

\item Kelly found that (m1), (m3), (m4) \red{follow} from (m2), (m5).

\end{itemize}

\end{frame}

\begin{frame}

\frametitle{Examples}


\begin{itemize}

\item $(\Set, 1, \times)$ is a monoidal category. 

\item $(\Set, 0, +)$ is also a monoidal category.

\bigskip

\item A preorder is the same as a thin category  (at
  most one map between any two objects).

\item A monoid is the same as a discrete monoidal category. 

\item A preordered monoid is the same as a thin monoidal category.


\bigskip

\item A category is a ``proof-relevant''  generalization of a preordered set.

\item A monoidal category is a ``proof-relevant'' generalization of a
  preordered monoid.

\end{itemize}


\end{frame}

\begin{frame}

\frametitle{Coherence}

\begin{itemize}

\item (Mac Lane) The free monoidal category on a set of objects enjoys
  a very simple form of (effective) coherence.

\begin{itemize}
\item It is (very easily) decidable if there is a map between two
  objects $A$, $B$, and to exhibit one in this case.

\item Moreover, if there is a map, it is unique.
\end{itemize}

\end{itemize}

\end{frame}



\begin{frame}

\frametitle{Skew monoidal categories}

\begin{itemize}

\item A \emph{skew monoidal category} (Szlach{\'a}nyi) is a category $\C$
  together with an object $\I$, a functor $\ot : \C \times \C \to \C$
  and \red{nat.\ transfs.} $\lam$, $\rho$, $\al$ with components
\[
\scriptsize
\begin{array}{c}
\lam_{A} : \I \ot A \to A \\
\rho_{A} : A \to A \ot \I\\
\al_{A,B,C} : (A \ot B) \ot C \to A \ot (B \ot C)
\end{array}
\]
such that
\[
\scriptsize
\begin{array}{c}
\mathrm{(m1)}
\xymatrix@R=1.2pc@C=0.2pc{
    & \I \ot \I \ar[dr]^-{\lambda_\I} & \\
    \I \ar[ur]^-{\rho_\I} \ar@{=}[rr] & & \I
    }
\quad
\mathrm{(m2)}
\xymatrix@R=1.1pc{
      (A\ot \I) \ot B \ar[r]^{\alpha_{A,\I,B}}
      & A \ot (\I\ot B) \ar[d]^{A\ot \lambda_{B}}\\
      A \ot B \ar@{=}[r] \ar[u]^{\rho_{A}\ot B}&  A \ot B 
    }
\\
\mathrm{(m3)}
\xymatrix@C=0.2pc@R=1.2pc{
  (\I \ot A) \ot B \ar[dr]_{\lambda_A \ot B} \ar[rr]^{\alpha_{\I,A,B}} 
           & &  \I \ot (A \ot B) \ar[dl]^{\lambda_{A \ot B}}\\
  & A \ot B & 
    }
\quad
\mathrm{(m4)}
\xymatrix@C=0.2pc@R=1.3pc{
  (A \ot B) \ot \I \ar[rr]^{\alpha_{A,B,\I}} 
           & &  A \ot (B \ot \I) \\
  & A \ot B \ar[ul]^{\rho_{A \ot B}} \ar[ur]_{A \ot \rho_B} & 
    }
\\
\mathrm{(m5)}
\xymatrix@R=1.3pc@C=2.5pc{
(A\ot (B \ot C)) \ot D \ar[rr]^{\alpha_{A,B \ot C,D}}
  & & A\ot ((B \ot C)\ot D) \ar[d]^{A\ot \alpha_{B,C,D}}
  \\
((A\ot B) \ot C) \ot D \ar[u]^{\alpha_{A,B,C} \ot D}
      \ar[r]^{\alpha_{A\ot B,C,D}}
  & (A\ot B) \ot (C \ot D) \ar[r]^{\alpha_{A,B,C\ot D}}
    & A\ot (B \ot (C \ot D))
}
\end{array}
\]

\item (m1), (m3), (m4) \red{do not follow} from (m2), (m5) in this
  situation.

\end{itemize}

\end{frame}

\begin{frame}

\frametitle{Examples}

\begin{itemize}

\item $(\mathbf{Ptd}, 0', +')$ where \\
$\mathbf{Ptd}$ is the class of pointed sets \\
$0' = (1, \ast)$ \\
$(X, p) +' (Y, q) = (X + Y, \mathsf{inl}\, p)$ \\
is a skew monoidal category.

\bigskip

\item Given a category $\C$ and a functor $J : \J \to \C$
  such that $\Lan_J\, F : \C \to \C$ exists for any $F : \J \to \C$.

  Let $F \cdot^J G = \Lan_J\, F \cdot G$. 

  Then $([\J,\C], J, \cdot^J)$ is a skew monoidal category. \\[6pt]

  Relative monads on $J$ are the same as monoids in this skew monoidal
  category.


\end{itemize}

\end{frame}

% \begin{frame}

% \frametitle{Examples ctd.}

% \begin{itemize}

% \item Given a (skew) monoidal category $(\C, \I, \ot)$. 

%   Given also a lax monoidal comonad $(D, \e, \m)$ on $\C$.

%   Let $A \ot^D B = A \ot D B$.  

%   Then $(\C, \I, \ot^D)$ is a skew monoidal category.

% \medskip

% \item A similar skew monoidal structure on $\C$ is obtained with an
%   oplax monoidal monad $(T, \e, \m)$ taking
%   $A \mathbin{{}^T\ot} B = T A \ot B$.

% \bigskip

% \item Given a (skew) monoidal category $(\C, \I, \ot)$.

%   Given a lax monoidal monad $(T, \m)$ ($\e = \eta_\I)$ on $\C$.

%   Let
%   $A \ot^T B = A \ot B$ and \\ $f \ot^T g = \m \comp f \ot g$ 
%   for
%   $f : A \to T A'$, $g : B \to T B'$. 
 
%   Then $(\mathbf{Kl}(T), \I, \ot^T)$ is a skew monoidal category.



% \end{itemize}

% \end{frame}

\begin{frame}

  \frametitle{Categorical calculus}

\begin{itemize}

\item Essentially by definition, the free skew monoidal category on a
  set $\Var$ can be described as a deductive system, a ``categorical''
  or Hilbert-style calculus.

\medskip

\item Objects are formulae. 

\item \emph{Formulae} are atoms $X \in \Var$, $\I$ and $A \ot B$ where
  $A$, $B$ are formulae.

\medskip

\item Maps are equivalence classes of derivations of sequents
  $A \tto C$ where $A$, $C$ are (single) formulae.

\item \emph{Derivations} are constructed with these inference rules:
\[
\scriptsize
\begin{array}{c}
\infer[\id]{A \tto A}{}
\qquad
\infer[\dcomp]{A \tto C}{A \tto B & B \tto C}
\\[12pt]
\infer[\ot]{A \ot B \tto C \ot D}{A \tto C & B \tto D}
\\[12pt]
\infer[\lam]{\I \ot A \tto A}{}
\qquad
\infer[\rho]{A \tto A \ot \I}{}
\qquad
\infer[\al]{(A \ot B) \ot C \tto A \ot (B \ot C)}{}
\end{array}
\]

\end{itemize}


\end{frame}


\begin{frame}

\frametitle{Categorical calculus ctd.}

\begin{itemize}

\item \emph{Equivalence of derivations} is the congruence $\doteq$
  induced by the equations
\[
\begin{array}{c}
\id \comp f \doteq f
\qquad
f \doteq f \comp \id
\qquad
(f \comp g) \comp h \doteq f \comp (g \comp h)
\\[6pt]
\id \ot \id \doteq \id
\qquad 
(h \comp f) \ot (k \comp g) \doteq h \ot k \comp f \ot g
\\[6pt]
\lam \comp \id \ot f \doteq f \comp \lam
\\
\rho \comp f \doteq f \ot \id \comp \rho
\\
\al \comp (f \ot g) \ot h \doteq f \ot (g \ot h) \comp \al
\\[6pt]
\lam \comp \rho \doteq \id
\qquad
\id \doteq \id \ot \lam \comp \al \comp \rho \ot \id
\\
\lam \comp \al \doteq \lam \ot \id
\qquad
\al \comp \rho \doteq \id \ot \rho
\\
\al \comp \al \doteq \id \ot \al \comp \al \comp \al \ot \id 
\end{array}
\]

\end{itemize}

\end{frame}



\begin{frame}

\frametitle{Coherence?}

\begin{itemize}

\item It is not obvious at all when we have zero, one or more maps
  between two given objects in the free skew monoidal category or when
  two given maps between two given objects are the same.

\bigskip

\item There are no derivations (no maps) \\[6pt]

 $X \tto \I \ot X$, \\ 

 $X \ot \I \tto X$, \\

 $X \ot (Y \ot Z) \tto (X \ot Y) \ot Z$.

\medskip

\item We have inequivalent derivations (distinct maps) \\[6pt]

 $\rho \comp \lam \not\doteq \id : \I \ot \I \tto \I \ot \I$, \\

 $\id \not\doteq \al \comp  \rho \ot \lam : 
 X \ot (\I \ot Y) \tto X \ot (\I \ot Y) $, \\

 $\id \not\doteq \rho \ot \lam \comp \al :
 (X \ot \I) \ot Y \tto (X \ot \I) \ot Y$.

\bigskip


\item This means that the logic of skew monoidal categories is more
  interesting in comparison linear logic with mult conj only---the
  same consequence can have multiple distinct derivations.

\end{itemize}

\end{frame}



\begin{frame}

\frametitle{Sequent calculus}

\begin{itemize}

\item Here is a different deductive description of the free skew
  monoidal category, a (cut-free) ``sequent calculus''.

\item \emph{Sequents} now take the form $S \mid \Gamma \vdash C$
  where
\begin{itemize}
\item $S$ (stoup) is an optional formula,
\item $\Gamma$ (context) is a list of formulae,
\item $C$ is a single formula.
\end{itemize}

\item \emph{Derivations} are constructed with these inference rules:
\[
\scriptsize
\begin{array}{c@{\quad \quad}c}
\infer[\uf]{\n \mid A, \Gamma \vdash C}{
  A \mid \Gamma \vdash C
}
&
\infer[\ax]{A \mid ~\vdash A}{
}
\\[6pt]
\infer[\IL]{\I \mid \Gamma \vdash C}{
  \n \mid \Gamma \vdash C
}
&
\infer[\IR]{\n \mid ~ \vdash \I}{
}
\\[6pt]
\infer[\otL]{A \ot B \mid \Gamma \vdash C}{
  A \mid B, \Gamma \vdash C
}
&
\infer[\otR]{S \mid \Gamma, \Delta \vdash A \otimes B}{
  S \mid \Gamma \vdash A
  &
  \n \mid \Delta \vdash B
}
\end{array}
\]

\item $\IL$, $\otL$ only apply in the stoup. \\ 
 $\otR$ sends the stoup formula, if present, to the
  1st premise.

% \item Two forms of cut are now admissible:
% \[
% \scriptsize
% \infer[\scut]{S \mid \Gamma, \Delta \vdash C}{
%   S \mid \Gamma \vdash A
%   &
%   A \mid \Delta \vdash C
% }
% \qquad
% \infer[\ccut]{S \mid \Delta_0, \Gamma, \Delta_1 \vdash C}{
%   \n \mid \Gamma \vdash A
%   &
%   S \mid \Delta_0, A, \Delta_1 \vdash C
% }
% \]

\end{itemize}


\end{frame}

%\end{document}

\begin{frame}

\frametitle{Sequent calculus ctd.}

\begin{itemize}

\item \emph{Equivalence of derivations} is the congruence $\circeq$
  induced by
\[
\scriptsize
\begin{array}{c@{\quad \circeq\quad}c}
\proofbox{
\infer[\ax]{\I \mid ~ \vdash \I}{
}
}
&
\proofbox{
\infer[\IL]{\I \mid ~ \vdash \I}{
  \infer[\IR]{\n \mid ~ \I}{
  }
}
}
\\[12pt]
\proofbox{
\infer[\ax]{A \ot B \mid ~ \vdash A \ot B}{
}
}
&
\proofbox{
\infer[\otL]{A \ot B \mid ~ \vdash A \ot B}{
  \infer[\otR]{A \mid B \vdash A \ot B}{
   \infer[\ax]{A \mid ~ \vdash A}{
   }
   &
   \infer[\uf]{\n \mid B \vdash B}{
     \infer[\ax]{B \mid ~ \vdash B}{
     }
   }
 }
} 
}
\\[36pt]
\proofbox{
\infer[\otR]{\n \mid A', \Gamma, \Delta \vdash A \otimes B}{
  \infer[\uf]{\n \mid A', \Gamma \vdash A}{
    A' \mid \Gamma \vdash A
  }
  &
  \n \mid \Delta \vdash B
}
}
&
\proofbox{
\infer[\uf]{\n \mid A', \Gamma, \Delta \vdash A \otimes B}{
  \infer[\otR]{A' \mid \Gamma, \Delta \vdash A \otimes B}{
    A' \mid \Gamma \vdash A
    &
    \n \mid \Delta \vdash B
  }
}
}
\\[18pt]
\proofbox{
\infer[\otR]{\I \mid \Gamma, \Delta \vdash A \otimes B}{
  \infer[\IL]{\I \mid \Gamma \vdash A}{
    \n \mid \Gamma \vdash A
  }
  &
  \n \mid \Delta \vdash B
}
}
&
\proofbox{
\infer[\IL]{\I \mid \Gamma, \Delta \vdash A \otimes B}{
  \infer[\otR]{\n \mid \Gamma, \Delta \vdash A \otimes B}{
    \n \mid \Gamma \vdash A
    &
    \n \mid \Delta \vdash B
  }
}
}
\\[18pt]
\proofbox{
\infer[\otR]{A' \ot B' \mid \Gamma, \Delta \vdash A \otimes B}{
  \infer[\otL]{A' \ot B' \mid \Gamma \vdash A}{
    A' \mid B', \Gamma \vdash A
  }
  &
  \n \mid \Delta \vdash B
}
}
&
\proofbox{
\infer[\otL]{A' \ot B' \mid \Gamma, \Delta \vdash A \otimes B}{
  \infer[\otR]{A' \mid B', \Gamma, \Delta \vdash A \otimes B}{
    A' \mid B', \Gamma \vdash A
    &
    \n \mid \Delta \vdash B
  }
}
}
\end{array}
\]

\end{itemize}

\end{frame}

%\end{document}

\begin{frame}

\frametitle{Sequent calculus ctd.}

\begin{itemize}

\item Define
\[
\begin{array}{c}
\ssem{\n} = \I \\ 
\ssem{A} = A
\end{array}
\]
and
\[
\begin{array}{c}
\csem{A}{~} = A \\
\csem{A}{B, \Gamma} = \csem{(A \otimes B)}{\Gamma}  
\end{array}
\]
so $\csem{A}{A_1, A_2 \ldots, A_n} = (\ldots (A \otimes A_1)
  \otimes A_2) \ldots) \otimes A_n$.

\bigskip

\item There is a bijection between
\begin{itemize}
\item derivations of
  $\csem{\ssem{S}}{\Gamma} \tto C$ in the categorical calculus (up to
  $\doteq$) and
\item derivations of  $S \mid \Gamma \vdash C$
  in the sequent calculus (up to $\circeq$) 
\end{itemize}

\end{itemize}

\end{frame}


\begin{frame}

\frametitle{What makes this work?}

\begin{itemize}

\item We can easily construct derivations to correspond to $\lam_A$,
  $\rho_A$, $\al_{A,B,C}$:

\[
\scriptsize
\infer[\otL]{\I \ot A \mid ~ \vdash A}{
  \infer[\IL]{\I \mid A \vdash A}{
    \infer[\uf]{\n \mid A \vdash A}{
      \infer[\ax]{A \mid ~ \vdash A}{
      }
    }
  }
}
\]
\[
\scriptsize
\infer[\otR]{A \mid ~ \vdash A \ot \I}{
  \infer[\ax]{A \mid ~ \vdash A}{
  }
  &
  \infer[\IR]{\n \mid ~ \vdash \I}{
  }
}
\qquad
\infer[\otL]{(A \ot B) \ot C \mid ~ \vdash A \ot (B \ot C)}{
  \infer[\otL]{A \ot B \mid C \vdash A \ot (B \ot C)}{
    \infer[\otR]{A \mid B, C \vdash A \ot (B \ot C)}{
      \infer[\ax]{A \mid ~ \vdash A}{
      }
      &
      \infer[\uf]{\n \mid B, C \vdash B \ot C}{
        \infer[\otR]{B \mid C \vdash B \ot C}{
          \infer[\ax]{B \mid ~ \vdash B}{
          }
          &
          \infer[\uf]{\n \mid C \vdash C}{
            \infer[\ax]{C \mid ~ \vdash C}{
            }
          }
        }
      }
    }
  }
}
\]


\end{itemize}

\end{frame}


\begin{frame}

\frametitle{What makes this work? ctd.}

\begin{itemize}

\item But we cannot construct derivations for converse sequents for
  $A = X$, $B = B$, $C = Z$:

\[
\scriptsize
\infer[\otL]{X \ot \I \mid \vdash X}{
  \deduce{X \mid \red{\I} \vdash X}{
    \red{??}
  }
}
\]
(we cannot apply $\IL$ in the context),

\medskip

\[
\scriptsize
\infer[\red{\otR}]{\red{X} \mid ~\vdash \I \ot X}{
  \deduce[\red{??}]{\red{X} \mid ~ \vdash \I}{
  }
  &
  \deduce[\red{??}]{\n \mid ~ \vdash X}{
  }
}
\]
(we cannot split the antecedent suitably at $\otR$),

\medskip

\[
\scriptsize
\infer[\otL]{X \ot (Y \ot Z) \mid ~\vdash (X \ot Y) \ot Z}{
  \infer[\red{\otR}]{X \mid \red{Y \ot Z} \vdash (X \ot Y) \ot Z}{
   \deduce[\red{??}]{X \mid \red{Y \ot Z} \vdash X \ot Y}{
   }
   &
   \deduce[\red{??}]{\n \mid ~ \vdash  Z}{
   }
 }
}
\qquad
\infer[\otL]{X \ot (Y \ot Z) \mid ~ \vdash (X \ot Y) \ot Z}{
  \infer[\red{\otR}]{X \mid \red{Y \ot Z} \vdash (X \ot Y) \ot Z}{
   \deduce[\red{??}]{X \mid ~ \vdash X \ot Y}{
   }
   &
   \deduce[\red{??}]{\n \mid \red{Y \ot Z} \vdash Z}{
   }
 }
}
\]
(we cannot apply $\otL$ in the context, must therefore apply $\otR$
first but cannot split the antecedent suitably).

\end{itemize}

\end{frame}


\begin{frame}

\frametitle{Focused fragment}

\begin{itemize}

\item The equational theory on sequent calculus derivations is locally
  confluent and strongly normalizing.

\item Normal-form derivations can be described as derivations in a
  focused fragment.

\item The focused calculus has two sequent forms. 

  $\mathsf{L}$-\emph{sequents} are $S \mid \Gamma \vdashL C$ where $S$
  is a general stoup.

  $\mathsf{R}$-\emph{sequents} are $T \mid \Gamma \vdashR C$ where $T$
  is an optional atom.

\item \emph{Derivations} are constructed with these inference rules:
\[
\scriptsize
\begin{array}{c@{\quad \quad}c@{\quad \quad}c}
\infer[\uf]{\n \mid A, \Gamma \vdashL C}{
  A \mid \Gamma \vdashL C
}
&
\infer[\msw]{T \mid \Gamma \vdashL C}{
  T \mid \Gamma \vdashR C
}
\hspace*{-12mm}
&
\infer[\ax]{X \mid ~\vdashR X}{
}
\\[6pt]
\infer[\IL]{\I \mid \Gamma \vdashL C}{
  \n \mid \Gamma \vdashL C
}
& &
\infer[\IR]{\n \mid ~ \vdashR \I}{
}
\\[6pt]
\infer[\otL]{A \ot B \mid \Gamma \vdashL C}{
  A \mid B, \Gamma \vdashL C
}
& &
\hspace*{5mm}
\infer[\otR]{T \mid \Gamma, \Delta \vdashR A \otimes B}{
  T \mid \Gamma \vdashR A
  &
  \n \mid \Delta \vdashL B
} 
\end{array}
\]

\item The focused rules define a sound and complete root-first proof
  search strategy.

\item Multiple derivations of an $\mathsf{L}$-sequent result from \\
  (i) choices between $\uf$ and $\msw$ and \\ (ii) choices between
  different splits of the context in $\otR$.

\end{itemize}

\end{frame}


%\end{document}

\begin{frame}

\frametitle{Focused fragment}

\begin{itemize}

\item There is a bijection between
\begin{itemize}
\item derivations of $S \mid \Gamma \vdash C$
  in the sequent calculus (up to $\circeq$) and 
\item derivations of $S \mid \Gamma \vdashL C$ in the focused calculus.
\end{itemize}

\bigskip

\item This gives an (effective) coherence result:

\begin{itemize}
\item To enumerate, without duplicates, all maps $A \tto C$ (presented as categorical calculus derivations): \\[9pt]

find all focused derivations of
  $A \mid ~ \vdashL C$ and translate those to the categorical calculus.

%  (There can be multiple different focused derivations. They
%  translate to different maps.)
\medskip

\item To compare two maps $A \tto C$ (presented as categorical calculus
  derivations) for equality: \\[9pt] translate them to focused derivations of
  $A \mid ~ \vdashL C$ and compare the results.

%  (Not all such maps are equal.)

\end{itemize}

\end{itemize}

\end{frame}

\begin{frame}

\frametitle{Skew monoidal closed categories}

\begin{itemize}
\item A \emph{skew monoidal closed category} is a skew monoidal closed
  category $(\C, \I, \otimes, \lam, \rho, \al)$ together with a
  functor $\lolli : \C^{\mathrm{op}} \times \C \to \C$ such that
\[
(-) \to B \vdash B \lolli (-)
\]
for any object $B$.
\end{itemize}

\end{frame}


\begin{frame}

\frametitle{Categorical calculus}

\begin{itemize}

\item Add formulas $A \lolli B$.

\bigskip

\item Add inference rules 

\[
\scriptsize
\infer[\pi]{A \tto B \lolli C}{
  A \ot B \tto C
}
\qquad
\infer[\pi^{-1}]{A \ot B \tto C}{
  A \tto B \lolli C
}
\]

and some equations for $\doteq$.

\end{itemize}

\end{frame}


\begin{frame}

\frametitle{Sequent calculus}

\begin{itemize}

\item Add formulae $A \lolli B$.

\bigskip

\item Add inference rules 

\[
\scriptsize
    \infer[\loL]{A \lolli B \mid \Gamma , \Delta \vdash C}{
      {-} \mid \Gamma \vdash A
      &
      B \mid \Delta \vdash C
    }
    \qquad
    \infer[\loR]{S \mid \Gamma \vdash A \lolli B}{S \mid \Gamma , A \vdash B}
\]

and some equations for $\circeq$.

\end{itemize}

\end{frame}



\begin{frame}


\frametitle{Focused fragment (a first attempt)}

\begin{itemize}

\item We need four sequent forms for four phases of proof search:
\[
\scriptsize
\begin{array}{c}
S \mid \Gamma \vdash_\RI C
\qquad S \mid \Gamma \vdash_\LI P
\qquad T \mid \Gamma \vdash_\Pass P
\qquad T \mid \Gamma \vdash_\F P
\end{array}
\]
where $S$ is an unrestricted stoup and $C$ and unrestricted formula,
but 
\begin{itemize}
\item $T$ is a negative stoup (neither $\I$ nor $A \ot B$) and
\item $P$ is a positive formula (not $A \lo B$).
\end{itemize}

\item The inference rules are:
\[
\scriptsize
  \begin{array}{l@{}c}
    \text{(right invertible)} & %\\ &
    \proofbox{
    \infer[\loR]{S \mid \Gamma \vdash_{\RI} A \lolli B}{S \mid \Gamma , A \vdash_{\RI} B}
    \qquad
    \infer[\LI 2 \RI]{S \mid \Gamma \vdash_{\RI} P}{S \mid \Gamma \vdash_{\LI} P}
    }
    \\[9pt]
    \text{(left invertible)} & %\\ &
    \proofbox{
    \infer[\IL]{\I \mid \Gamma \vdash_{\LI} P}{{-} \mid \Gamma \vdash_{\LI} P}
    \qquad
    \infer[\otL]{A \ot B \mid \Gamma \vdash_{\LI} P}{A \mid B , \Gamma \vdash_{\LI} P}
    \qquad
    \infer[\Pass 2 \LI]{T \mid \Gamma \vdash_{\LI} P}{T \mid \Gamma \vdash_{\Pass} P}
    }
    \\[9pt]
    \text{(passivation)} & %\\ &
    \proofbox{
      \infer[\uf]{{-} \mid A , \Gamma \vdash_{\Pass} P }{
      A\mid \Gamma \vdash_{\LI} P
    }
    \qquad
    \infer[\F 2 \Pass]{T \mid \Gamma \vdash_{\Pass} P}{
      T \mid \Gamma \vdash_{\F} P
    }
    }
    \\[9pt]
    \text{(focusing)} &    %\\ %&
    \hspace*{3mm}
    \infer[\ax]{X \mid \quad \vdash_{\F} X}{}
    \quad
    \infer[\IR]{{-} \mid \quad \vdash_{\F} \I}{} 
    \quad 
    \infer[\otR]{T \mid \Gamma , \Delta \vdash_{\F} A \ot B}{
      T \mid \Gamma \vdash_{\RI} A
      &
      {-} \mid \Delta \vdash_{\RI} B
    }
\\[9pt]
&
    \infer[\loL]{A \lolli B \mid \Gamma , \Delta \vdash_{\F} P}{
      {-} \mid \Gamma \vdash_{\RI} A
      &
      B \mid \Delta \vdash_{\LI} P
    }
   \end{array}
\]

\end{itemize}

\end{frame}


\begin{frame}{Focused fragment (good version)}

\begin{itemize}

\item There is too much nondeterminism between $\otR$ and $\loL$ as
  compared to what $\circeq$ allows.

\bigskip

\item Could try to order $\otR$ and $\loL$ in separate phases,
  but this does not work. Sometimes $\otR$ needs to be used first,
  sometimes $\loL$.

\bigskip

\item We need to keep them in the same phase.

\item But we can allow $\loL$ to be applied after $\otR$ only if the
  same application cannot be simulated with applying $\loL$ first.

\item Ie, apply $\loL$ before $\otR$ except when it is justified to do
  it after.

\item This requires some \emph{bookkeeping} added to the inference
  rules.

\bigskip 

\item There is also too much nondeterminism between $\otR$ and $\uf$.

\item This can be eliminated by similar prioritization of $\uf$ over
  $\otR$ with the same bookkeeping mechanism.

\end{itemize}


\end{frame}


\begin{frame}

\frametitle{Takeaway}

\begin{itemize}

\item Logic and category theory are mutually enriching, especially
  at their intersection, in categorical proof theory.

\begin{itemize}
\item category theory supplies well-motivated notions of derivation
  and identity of derivations
\item proof theory helps in stating and proving coherence theorems
\end{itemize}

\bigskip

\item Skew logics are very interesting both logically
  and category-theoretically. 

\item In particular, they cast light on the ``anatomy'' of stronger
  logics.

\end{itemize}

\end{frame}



\end{document}