\documentclass[submission,copyright,creativecommons]{eptcs}
\providecommand{\event}{NCL'22} % Name of the event you are submitting to
%\usepackage{breakurl}             % Not needed if you use pdflatex only.
\usepackage{underscore}           % Only needed if you use pdflatex.
\usepackage{amsmath}
\usepackage{amsthm}
\usepackage{amsfonts}
\usepackage{amssymb}
\usepackage{enumerate}
\usepackage{prftree}
%\usepackage{ntheorem}
%\usepackage{tikz-cd}
\usepackage{hyperref}
%\usepackage{float}
%\usepackage{graphicx}
\usepackage{quiver}
\usepackage[all,cmtip]{xy}
\usepackage{bussproofs}
\usepackage{proof}
%\usepackage[a4paper, total={6in, 8in}]{geometry}
%\usepackage{lscape}
%% \theorembodyfont{}
\newtheorem{theorem}{Theorem}
\newtheorem{corollary}[theorem]{Corollary}
\newtheorem{lemma}[theorem]{Lemma}
\newtheorem{defn}[theorem]{Definition}
\newtheorem{remark}[theorem]{Remark}
%\newtheorem*{proof}{Proof : }
\newtheorem{fact}[theorem]{Fact}
\newcommand{\ldbc}{[\![}
\newcommand{\rdbc}{]\!]}
\newcommand{\tbar}{[\vec{x}/\vec{t}]}
\newcommand{\ltbar}{[\vec{x}, x/\vec{t}, x]}
\newcommand{\tl}{\otimes \mathsf{L}}
\newcommand{\tr}{\otimes \mathsf{R}}
\newcommand{\lright}{\multimap \mathsf{R}}
\newcommand{\lleft}{\multimap \mathsf{L}}
\newcommand{\pass}{\mathsf{pass}}
\newcommand{\unitl}{\mathsf{IL}}
\newcommand{\unitr}{\mathsf{IR}}
\newcommand{\ax}{\mathsf{ax}}
\newcommand{\id}{\mathsf{id}}
\newcommand{\ot}{\otimes}
\newcommand{\lolli}{\multimap}
\newcommand{\illol}{\rotatebox[origin=c]{180}{$\multimap$}}
\newcommand{\I}{\mathsf{I}}
\newcommand{\msfL}{\mathsf{L}}
\newcommand{\defeq}{=_{\mathsf{df}}}
\newcommand{\comp}{\mathsf{comp}}
\newcommand{\RI}{\mathsf{RI}}
\newcommand{\LI}{\mathsf{LI}}
\newcommand{\Pass}{\mathsf{P}}
\newcommand{\F}{\mathsf{F}}
\newcommand{\xvdash}{\vdash^{x}}

\title{Focusing Skew Monoidal Closed Categories}
\author{Tarmo Uustalu
\institute{Reykjavik University, Iceland}
\institute{Tallinn University of Technology, Estonia}
\email{tarmo@ru.is}
\and
Niccol{\`o} Veltri \qquad\qquad Cheng-Syuan Wan
\institute{Tallinn University of Technology, Estonia}
\email{\quad niccolo@cs.ioc.ee \quad\qquad cswan@cs.ioc.ee}
}
\def\titlerunning{Focusing Skew Monoidal Closed Categories}
\def\authorrunning{T. Uustualu, N. Veltri \& C-S. Wan}
\begin{document}
\maketitle
\begin{abstract}
  Monoidal closed categories are used as standard categorical models of intuitionistic noncommutative linear logic: the monoidal unit and tensor model the multiplicative unit and conjunction; the internal hom models linear implication. In recent years, the weaker notion of skew monoidal closed category has been proposed by Ross Street, where the three structural laws of left and right unitality and associativity are not required to be invertible, they are merely natural transformations with a specific orientation. A question arises naturally: Is it possible to find logical proof systems which are naturally modelled by skew monoidal closed categories? We answer positively by introducing three equivalent deductive systems that serve the purpose: a Hilbert-style calculus; a cut-free sequent calculus; a calculus of normal forms obtained from an adaptation of Andreoli's focusing technique to our skew setting. The resulting focused sequent calculus peculiarly employs a system of tags for keeping track of new formulae appearing in the antecedent and appropriately reducing nondeterministic choices in proof search. Focusing solves the coherence problem for skew monoidal closed categories, which implies the existence of effective procedures for deciding equality of canonical morphisms in all models.

%%   In previous works from Uustalu et al, proof theory of some variants of skew monoidal categories are developed.
%%   We also know that given a skew monoidal category, there is a corresponding skew closed structure on the same category.
%%   It motivates us to develop a proof theory of such skew monoidal closed structures and see its meta properties.
%%
%%   In this paper, a Hilbert-style categorical calculus, a cut-free sequent calculus, and a tagged focused sequent calculus are presented.
%%   We prove the soundness and completeness between categorical calculus and sequent calculus.
%%   Moreover any two equivalent proofs from cut-free sequent calculus are mapped into same derivation in tagged focused calculus.
%%   Therefore we have an effective procedure to determine morphism existence and equivalence in skew monoidal closed categories.
\end{abstract}
\section{Introduction}
It is a well-known fact from the late 80s/early 90s that symmetric monoidal closed categories are standard categorical models of MILL, multiplicative intuitionistic linear logic, whose logical connectives comprise of multiplicative unit $\I$ and conjunction $\ot$ and linear implication $\lolli$ \cite{mellies:categorical:09}. Probably lesser-known, though a quite straightforward generalization of the previous sentence which actually predated the inception of linear logic altogether, is the fact that (non-necessarily symmetric) monoidal biclosed categories are categorical models of the noncommutative variant of MILL (NMILL), where the structural rule of exchange is absent and there are two ordered implications $\lolli$ and $\illol$ \cite{lambek:deductive:68}. A proof system for NMILL is typically called \emph{Lambek calculus}. Lambek refers to the implications $\lolli$ and $\illol$ as residuals, and models of Lambek calculus are thereupon called residual categories. The Lambek calculus is often practically employed in formal investigations of natural languages \cite{lambek:mathematics:58}.
By dropping one of the ordered implication of NMILL one obtains a fragment of the Lambek calculus enjoying categorical semantics in every monoidal closed category.

In recent years, Ross Street introduced the new notion of skew monoidal closed categories \cite{street:skew-closed:2013}. These are a weakening of monoidal closed categories: their structure includes a unit $I$, a tensor $\ot$, an internal hom $\lolli$ and an adjunction relating the latter two operations, as in usual non-skew monoidal categories. The difference lays in the three structural laws of left and right unitality, $\lambda_A : \I \ot A \to A$ and $\rho_A : A \to A \ot \I$, and associativity, $\alpha_{A,B,C} : (A \ot B) \ot C \to A \ot (B \ot C)$, which are usually required to be natural isomorphisms, but in the skew variant are merely natural transformations with the specified orientation. Street originally proposed this weaker notion for better understanding and fixing a famous imbalance, first noticed by Eilenberg and Kelly \cite{eilenberg:closed:1966}, found in the usual adjunction relating monoidal and closed structures \cite{street:skew-closed:2013,uustalu:eilenberg-kelly:2020}. In the last decade, skew monoidal closed categories, together with their non-closed/non-monoidal variants, have been thoroughly studied, with applications ranging from algebra and homotopy theory to programming language semantics \cite{szlachanyi:skew-monoidal:2012,lack:skew:2012,lack:triangulations:2014,altenkirch:monads:2014,buckley:catalan:2015,bourke:skew:2017,bourke:skew:2018,tomita:realizability:21}.

A question arises naturally: is it possible to characterize skew monoidal closed categories as standard categorical models of a certain logical proof system, e.g. a skew variant of NMILL?
This paper provides a positive answer to this question. We introduce a cut-free sequent calculus for a skew variant of NMILL, which we name (rather unoriginally) SkNMILL. Sequents are peculiarly defined as triples $S \mid \Gamma \vdash A$, where the succedent is a single formula $A$ (as in intuitionistic linear logic), but the antecedent is split in two parts: an optional formula $S$, called \emph{stoup} \cite{girard:constructive:91}, and an ordered list of formulae $\Gamma$. Inference rules are similar to the ones of NMILL but with specific structural restrictions for accurately capturing the structural laws of skew monoidal closed categories, and nothing more. In particular, and in analogy with NMILL, the structural rules of contraction, weakening and exchange are all absent. Sets of derivations are quotiented by a congruence relation $\circeq$, carefully chosen to serve as the proof-theoretic counterpart of the equational theory of skew monoidal closed categories. The design of the sequent calculus draws inspiration from, and further advance, the line of work of Uustalu, Veltri and Zeilberger on proof systems for various categories with skew structure: skew semigroup (a.k.a. Tamari order) \cite{zeilberger:semiassociative:19}, skew monoidal (non-closed) \cite{uustalu:sequent:2018,uustalu:proof:nodate} and its symmetric variant \cite{veltri:coherence:2021}, skew prounital (non-monoidal) closed \cite{uustalu:deductive:nodate}.

The metatheory of SkNMILL is developed in two different but related directions:
\begin{enumerate}[($i$)]
  \item We study the categorical semantics of SkNMILL, by showing that the cut-free sequent calculus admits an interpretation of formulae, derivations and equational theory $\circeq$ in every skew monoidal closed category. Moreover, the sequent calculus is \emph{canonical} among skew monoidal closed categories, in the sense that it gives a particular presentation of the \emph{initial} (also called \emph{term} or \emph{free}) model. This can be made precise by introducing a Hilbert-style calculus which directly represents the initial model and subsequently proving that derivations in the two calculi are in a bijective correspondence.
This shows that the categorical semantics of SkNMILL is sound and, most importantly, complete. The latter fact implies that deciding the commutativity of a diagram made of canonical morphisms (i.e. structural laws) in an arbitrary skew monoidal category, a problem typically referred to as \emph{coherence} in the categorical literature, is reduced to checking whether the corresponding derivations in the sequnt calculus are $\circeq$-related.

\item We investigate the proof-thereoretic semantics of SkNMILL, by defining a normalization strategy for sequent calculus derivations wrt. the congruence $\circeq$, when the latter is considered as weakly confluent and strongly normalizing reduction relation. The shape of normal forms is made explicit in a new \emph{focused} sequent calculus, whose derivations act as targets of the normalization procedure. The sequent calculus is ``focused'' in the sense of Andreoli \cite{andreoli:logic:1992}, as it implements a sound and complete goal-directed proof search strategy attempting to build a derivation in the (original, unfocused) sequent calculus. The focused system in this paper extends the previously
developed normal forms for skew monoidal (non-closed) categories \cite{uustalu:sequent:2018,uustalu:proof:nodate}. The presence of both positive ($\I$,$\ot$) and negative ($\lolli$) connectives requires some extra care in the implementation of the proof search algorithm, which is reflected in the design of the focused sequent calculus, in particular when aiming at removing all possible nondeterministic choices (wrt. the conversion $\circeq$) that can arise during proof search. This is technically realized in the focused sequent calculus by the peculiar employment of a system of \emph{tags} for keeping track of new formulae appearing in the antecedent. The focused sequent calculus can also be seen as a concrete presentation of the initial model for SkNMILL, and as such can be used for solving the coherence problem in an effective fashion: deciding commutativity of canonical diagrams in arbitrary skew monoidal categories is equivalent to deciding the \emph{syntactic} equality of the corresponding focused derivations.
\end{enumerate}

%% Srtucture of the paper?
%%
%% Recent discoveries on skew monoidal categories \cite{szlachanyi:skew-monoidal:2012} \cite{lack:skew:2012} \cite{lack:triangulations:2014} (check references in previous papers, at least include papers cited by previous three papers) provide us a good reasons to study their corresponding proof systems.
%% In previous work \cite{uustalu:sequent:2018} \cite{uustalu:deductive:nodate} \cite{uustalu:proof:nodate} \cite{veltri:coherence:2021}, sequent calculus proof system for skew monoidal categories, prounital closed categories, partial normal monoidal categories, and skew symmetric monoidal categories are presented respectively.
%% However, there is still lack of a proof analysis on skew monoidal closed categories.
%% Therefore, in this paper we give sound and complete sequent calculus system with skew monoidal closed categories.
%% Moreover, we use focusing strategy from \cite{andreoli:logic:1992} to solve the coherence problem of skew monoidal closed categories.
%%
%% Interestingly, because of having two connectives in our sequent calculus system, the focusing strategy becomes subtle and involved.
%% It cannot just divide derivations into invertible part and non-invertible part then fix their order.
%% We discovered that it is no harm to arrange invertible rules in a fixed order, but in non-invertible rules, bad thing happens.
%% The naive focused sequent calculus system cannot prove some provable sequents in original sequent calculus.
%% We will see this involved focused sequent calculus system in (?) section.
%%
%% This paper will go through in this order, in section two, we see the Hilbert style calculus of skew monoidal closed categories and its proof equivalences.
%%
%% In third section, a cut-free sequent calculus of skew monoidal closed categories is presented.
%% The key feature is that we have a special formula called stoup at the leftmost position in antecedent of a given sequent.
%% In the same time, for any left rules, it can only be applied to the stoup formula.
%% We will also see preliminarily relationship with Hilbert style calculus.
%%
%% Next section, we will see a focused sequent calculus system and its connection with original sequent calculus system.
%% We will see that all derivations in a same equivalence class under proof conversion relation, they will correspond to a unique derivation in focused system.

\section{Skew variant of NMILL}
In this section, we introduce a skew variant of NMILL, called SkNMILL
Following the settings from \cite{uustalu:sequent:2018}, \cite{uustalu:deductive:nodate}, and \cite{uustalu:proof:nodate}, the sequent $S \mid \Gamma \vdash A$ in our calculus system splits in three parts.
Succedent is a single formula $A$, and antecedent is divided into two parts: an optional formula $S$, called \emph{stoup}, and an ordered list of formulae $\Gamma$.

Rules of SkNMILL:
\begin{displaymath}
  \infer[\pass]{- \mid A , \Gamma \vdash C}{A \mid \Gamma \vdash C}
  \quad
  \infer[\unitl]{\I \mid \Gamma \vdash C}{- \mid \Gamma \vdash C}
  \quad
  \infer[\tl]{A \ot B \mid \Gamma \vdash C}{A \mid B , \Gamma \vdash C}
\end{displaymath}
\begin{displaymath}
  \infer[\tr]{S \mid \Gamma , \Delta \vdash A \ot B}{
    S \mid \Gamma \vdash A
    &
    - \mid \Delta \vdash B
  }
  \quad
  \infer[\lleft]{A \lolli B \mid \Gamma , \Delta \vdash C}{
    - \mid \Gamma \vdash A
    &
    B \mid \Delta \vdash C
    }
\end{displaymath}
\begin{displaymath}
  \infer[\lright]{S \mid \Gamma \vdash A \lolli B}{S \mid \Gamma , A \vdash B}
  \quad
  \infer[\ax]{A \mid \quad \vdash A}{}
  \quad
  \infer[\unitr]{- \mid \quad \vdash \I}{}
\end{displaymath}

Let us check some example why SkNMILL characterizes skew monoidal closed categories.
If we interpret morphism $A \Longrightarrow C$ as a sequent $A \mid \quad \vdash C$, then we can see the natural transformations are derivable in this sequent calculus.
For example, natural transformations $\lambda : \I \ot A \Longrightarrow A , \rho : A \Longrightarrow A \ot \I$, and $\alpha : (A \ot B) \ot C \Longrightarrow A \ot (B \ot C)$ in skew monoidal closed categories are admissible in SkNMILL:
\begin{displaymath}
  \infer[\tl]{\I \ot A \mid \quad \vdash A}{
    \infer[\unitl]{\I \mid A \vdash A}{
      \infer[\pass]{ \mid A \vdash A}{
        \infer[\ax]{A \mid \quad \vdash A}{}
      }
    }
  }
  \quad
  \infer[\tr]{A \mid \quad \vdash A \ot \I}{
    \infer[\ax]{A \mid \quad \vdash A}{}
    &
    \infer[\unitr]{- \mid \quad \vdash \I}{}
  }
  \quad
  \infer[\tl]{(A \ot B) \ot C \mid \quad \vdash A \ot (B \ot C)}{
    \infer[\tl]{A \ot B \mid C \vdash A \ot (B \ot C)}{
      \infer[\tr]{A \mid B , C \vdash A \ot (B \ot C)}{
        \infer[\ax]{A \mid \quad \vdash A}{}
        &
        \infer[\pass]{- \mid B , C \vdash B \ot C}{
          \infer[\tr]{B \mid C \vdash B \ot C}{
            \infer[\ax]{B \mid \quad \vdash B}{}
            &
            \infer[\pass]{- \mid C \vdash C}{
              \infer[\ax]{C \mid \quad \vdash C}{}
            }
          }
        }
      }
    }
  }
\end{displaymath}
Another important thing is that SkNMILL could not prove the inverse of any natural transformations above.
For example, $\rho^{-1}$ is not derivable in SkNMILL:
\begin{displaymath}
  \infer[\tl]{A \ot \I \mid \quad \vdash A}{
    \deduce{A \mid \I \vdash A}{??}
  }
\end{displaymath}
We interpret $\rho^{-1}$ into $A \ot \I \mid \quad \vdash A$, then according to bottom-up proof search strategy, we first apply $\tl$ but we get stuck immediately.
Therefore, $\rho^{-1}$ is not derivable in our sequent calculus.
Other cases are similar.
In categorical semantics section we will see more details about relationship between SkNMILL and skew monoidal closed categories.

In \cite{uustalu:sequent:2018} and \cite{uustalu:deductive:nodate}, Uustalu et al. provided proof equivalences between $\ot$ only and $\lolli$ only systems respectively.
Here we define proof equivelences on $\ot$ and $\lolli$ interaction cases:
\begin{align*}
  \unitl \text{ } (\lright \text{ } f) &\circeq \lright \text{ } (\unitl \text{ } f) &&f : - \mid \Gamma , A \vdash B
  \\
  \tl \text{ } (\lright \text{ } f) &\circeq \lright \text{ } (\tl \text{ } f) &&f : A \mid B , \Gamma , C \vdash D
  \\
  \lleft \text{ } (f, \tr \text{ } (g, h)) &\circeq \tr \text{ } (\lleft \text{ } (f , g), h) &&f: - \mid \Gamma \vdash A, g : B \mid \Delta \vdash C, h : - \mid \Lambda \vdash D
\end{align*}

Next, we check cut-freeness by proving following two cut rules are admissible in the sequent calculus system:
\begin{theorem}
Two cut rules $\mathsf{scut}$ and $\mathsf{ccut}$
  \begin{displaymath}
    \infer[\mathsf{scut}]{S \mid \Gamma , \Delta \vdash C}{
      S \mid \Gamma \vdash A
      &
      A \mid \Delta \vdash C
    }
    \quad
    \infer[\mathsf{ccut}]{S \mid \Delta_0 , \Gamma , \Delta_1 \vdash C}{
      - \mid \Gamma \vdash A
      &
      S \mid \Delta_0 , A , \Delta_1 \vdash C
    }
  \end{displaymath}
  are admissible in sequent calculus system for skew monoidal closed categories.
\end{theorem}
\begin{proof}
  According to previous works \cite{uustalu:sequent:2018} and \cite{uustalu:deductive:nodate} we know two cut rules, $\mathsf{scut}$ and $\mathsf{ccut}$ are admissible in each system separately.
  As our new sequent calculus system is a union of two previous systems, we can just check the $\ot$ and $\lolli$ interaction cases.

  Dealing with $\mathsf{scut}$ first, same as proof strategy in previous papers, we proof by induction on left premise of $\mathsf{scut}$ rule.
  \begin{enumerate}[1. ]
    \item First case is $\mathsf{scut} \text{ } ((\tl \text{ } f), g)$ where $f : A' \mid B' , \Gamma \vdash A, g : A \mid \Delta \vdash C$, then we do subinduction on $g$ to obtain two subcases:
          \begin{enumerate}[a. ]
            \item $g = \lright \text{ } g'$, then we let $\mathsf{scut} \text{ } ((\tl \text{ } f), (\lright \text{ } g')) \defeq \lright \text{ } (\mathsf{scut} \text{ } ((\tl \text{ } f), g'))$.
            \item $g = \lleft \text{ } g'$, then we let $\mathsf{scut} \text{ } ((\tl \text{ } f), (\lright \text{ } g')) \defeq \tl \text{ } (\mathsf{scut} \text{ } (f, (\lright \text{ } g')))$
          \end{enumerate}
    \item $\mathsf{scut} \text{ } ((\lleft f \text{ } (f' , f'')), g)$, where $f' : - \mid \Gamma' \vdash A'$, $f'' : B' \mid \Gamma'' \vdash A$, and $g : A \mid \Delta \vdash C$.
    This case is similar as $\mathsf{scut} \text{ } ((\tl \text{ } f), g)$.
    \item There is no new case for premise is from $\tr$ or $\lright$, because cut formulae are not matched.
  \end{enumerate}
  Next we see the cut-freeness of $\mathsf{ccut}$ rule.
  Similarly, we prove it by induction on the second premise.
  \begin{enumerate}[1. ]
    \item $\mathsf{ccut} \text{ } (f , (\lleft \text{ } (g, h)))$ where $f : - \mid \Gamma \vdash A$, $g : - \mid \Delta_1 \vdash A'$, $h : B' \mid \Delta_2 \vdash C$, then we do subinduciton on $f$:
    \begin{enumerate}[a. ]
      \item The only possibility is $A = A' \ot B'$ and $f = \tr (f_1, f_2)$ where $f_1 : - \mid \Gamma_1 \vdash A'$, $f_2 : - \mid \Gamma_2 \vdash B'$, then depending on $A' \ot B'$ in $\Delta_1$ or $\Delta_2$,
      we let $\mathsf{ccut} \text{ } (f , (\lleft \text{ } (g, h))) \defeq \lleft \text{ } (\mathsf{ccut} \text{ } (f, g), h)$ or $ \lleft \text{ } (g, (\mathsf{ccut} \text{ } (f, h)))$, respectively.
    \end{enumerate}
    \item $\mathsf{ccut} \text{ } (f , (\tr \text{ } (g, h)))$ where $f : - \mid \Gamma \vdash A$, $g : S \mid \Delta_1 \vdash A'$, $h : - \mid \Delta_2 \vdash B'$ is similar as above.
    \item $\mathsf{ccut} \text{ } (f, (\tl \text{ } g))$ and
    $\mathsf{ccut} \text{ } (f, (\lright \text{ } h))$ cases are similar, we permute $\mathsf{ccut}$ up,
    where $g : A' \mid B' , \Delta_0 , A , \Delta_1 \vdash C$, $h : S \mid \Delta_0 , A , \Delta_1 , A' \vdash B'$.
  \end{enumerate}
\end{proof}


%%We have a cut-free proof theory which characterizes skew monoidal closed categories nicely so far.
%%According to Curry-Howard-Lambek correspondence, we can interpret morphisms in skew monoidal closed categories as derivations in the sequent calculus system.
%%This fact allows us to tackle problems in skew monoidal closed categories by methods in proof theory.
%%In particular, we will use Andreoli's focusing technique \cite{andreoli:logic:1992} to analyze and solve coherence problem of skew monoidal closed categories.

\section{Categorical semantics of SkNMILL}
In this section we are going to deal with categorical semantics of SkNMILL.
First we introduce definition of skew monoidal closed categories and construction of free skew monoidal closed category FskMCC($\mathsf{At}$) on set $\mathsf{At}$, a set of atomic formulae.
We conclude this section by proving soundness and completeness between SkNMILL and FskMCC($\mathsf{At}$).

A $skew$ $monoidal$ $closed$ $category$ is a category $\mathbb{C}$ equipped with a unit $\mathsf{I}$, two adjoint functors $- \ot B : \mathbb{C} \longrightarrow \mathbb{C}$ and $B \lolli - : \mathbb{C} \longrightarrow \mathbb{C}$
, $- \ot B  \dashv B \lolli -$, and eight natural transformations
\begin{displaymath}
  \lambda_A : \mathsf{I} \otimes A \Longrightarrow A \qquad
  \rho_A : A \Longrightarrow A \otimes \mathsf{I} \qquad
  \alpha_{A,B,C} : (A\otimes B) \otimes C \Longrightarrow A\otimes (B\otimes C)
\end{displaymath}
\begin{displaymath}
  i_A : \I \lolli A \Longrightarrow A \qquad
  j_A : \I \Longrightarrow A \lolli A \qquad
  \msfL_{A, B, C} : B \lolli C \Longrightarrow (A \lolli B) \lolli (A \lolli C)
\end{displaymath}
\begin{displaymath}
  \eta_{A , B} : A \Longrightarrow B \lolli (A \ot B)
  \qquad
  \epsilon_{A , B} : (A \lolli B) \ot A \Longrightarrow B
  \qquad
  p_{A, B , C} : (A \ot B) \lolli C \Longrightarrow A \lolli (B \lolli C)
\end{displaymath}
\begin{displaymath}
  \mathsf{M}_{A , B ,C} : (B \lolli C) \ot (A \lolli B) \Longrightarrow A \lolli C
\end{displaymath}
satisfying following eighteen equations (m1)-(m5), (c1)-(c5), and (a1)-(a8).
\begin{center}
  (m1)
  % https://q.uiver.app/?q=WzAsMyxbMSwwLCJcXEkgXFxvdCBcXEkiXSxbMCwxLCJcXEkiXSxbMiwxLCJcXEkiXSxbMSwwLCJcXHJob197XFxJfSJdLFswLDIsIlxcbGFtYmRhX3tcXEl9Il0sWzEsMiwiIiwyLHsibGV2ZWwiOjIsInN0eWxlIjp7ImhlYWQiOnsibmFtZSI6Im5vbmUifX19XV0=
\begin{tikzcd}
	& {\I \ot \I} \\
	\I && \I
	\arrow["{\rho_{\I}}", from=2-1, to=1-2]
	\arrow["{\lambda_{\I}}", from=1-2, to=2-3]
	\arrow[Rightarrow, no head, from=2-1, to=2-3]
\end{tikzcd}
(m2)
% https://q.uiver.app/?q=WzAsNCxbMCwwLCIoQSBcXG90IFxcSSkgXFxvdCBCIl0sWzEsMCwiQSBcXG90IChcXEkgXFxvdCBCKSJdLFsxLDEsIkEgXFxvdCBCIl0sWzAsMSwiQSBcXG90IEIiXSxbMywyLCIiLDAseyJsZXZlbCI6Miwic3R5bGUiOnsiaGVhZCI6eyJuYW1lIjoibm9uZSJ9fX1dLFszLDAsIlxccmhvX0EgXFxvdCBCIl0sWzEsMiwiQSBcXG90IFxcbGFtYmRhX3tCfSJdLFswLDEsIlxcYWxwaGFfe0EgLCBcXEkgLCBCfSJdXQ==
\begin{tikzcd}
	{(A \ot \I) \ot B} & {A \ot (\I \ot B)} \\
	{A \ot B} & {A \ot B}
	\arrow[Rightarrow, no head, from=2-1, to=2-2]
	\arrow["{\rho_A \ot B}", from=2-1, to=1-1]
	\arrow["{A \ot \lambda_{B}}", from=1-2, to=2-2]
	\arrow["{\alpha_{A , \I , B}}", from=1-1, to=1-2]
\end{tikzcd}

(m3)
% https://q.uiver.app/?q=WzAsMyxbMCwwLCIoXFxJIFxcb3QgQSApIFxcb3QgQiJdLFsyLDAsIlxcSSBcXG90IChBIFxcb3QgQikiXSxbMSwxLCJBIFxcb3QgQiJdLFswLDEsIlxcYWxwaGFfe1xcSSAsIEEgLEJ9Il0sWzEsMiwiXFxsYW1iZGFfe0EgXFxvdCBCfSJdLFswLDIsIlxcbGFtYmRhX3tBfSBcXG90IEIiLDJdXQ==
\begin{tikzcd}
	{(\I \ot A ) \ot B} && {\I \ot (A \ot B)} \\
	& {A \ot B}
	\arrow["{\alpha_{\I , A ,B}}", from=1-1, to=1-3]
	\arrow["{\lambda_{A \ot B}}", from=1-3, to=2-2]
	\arrow["{\lambda_{A} \ot B}"', from=1-1, to=2-2]
\end{tikzcd}

(m4)
% https://q.uiver.app/?q=WzAsMyxbMCwwLCIoQSBcXG90IEIpIFxcb3QgXFxJIl0sWzIsMCwiQSBcXG90IChCIFxcb3QgXFxJKSJdLFsxLDEsIkEgXFxvdCBCIl0sWzAsMSwiXFxhbHBoYV97QSAsIEIsIFxcSX0iXSxbMiwxLCJBIFxcb3QgXFxyaG9fQiIsMl0sWzIsMCwiXFxyaG9fe0EgXFxvdCBCfSJdXQ==
\begin{tikzcd}
	{(A \ot B) \ot \I} && {A \ot (B \ot \I)} \\
	& {A \ot B}
	\arrow["{\alpha_{A , B, \I}}", from=1-1, to=1-3]
	\arrow["{A \ot \rho_B}"', from=2-2, to=1-3]
	\arrow["{\rho_{A \ot B}}", from=2-2, to=1-1]
\end{tikzcd}

(m5)
% https://q.uiver.app/?q=WzAsNSxbMCwwLCIoQVxcb3QgKEIgXFxvdCBDKSkgXFxvdCBEIl0sWzIsMCwiQSBcXG90ICgoQiBcXG90IEMpIFxcb3QgRCkiXSxbMiwxLCJBIFxcb3QgKEIgXFxvdCAoQyBcXG90IEQpKSJdLFsxLDEsIihBIFxcb3QgQikgXFxvdCAoQyBcXG90IEQpIl0sWzAsMSwiKChBIFxcb3QgKEJcXG90IEMpIFxcb3QgRCkiXSxbMCwxLCJcXGFscGhhX3tBICwgQlxcb3QgQyAsIER9Il0sWzEsMiwiQSBcXG90IFxcYWxwaGFfe0IgLCBDICxEfSJdLFszLDIsIlxcYWxwaGFfe0EgLEIgLENcXG90IER9IiwyXSxbNCwzLCJcXGFscGhhX3tBIFxcb3QgQiAsIEMgLCBEfSIsMl0sWzQsMCwiXFxhbHBoYV97QSAsIEIgLEN9IFxcb3QgRCJdXQ==
\begin{tikzcd}
	{(A\ot (B \ot C)) \ot D} && {A \ot ((B \ot C) \ot D)} \\
	{((A \ot (B\ot C) \ot D)} & {(A \ot B) \ot (C \ot D)} & {A \ot (B \ot (C \ot D))}
	\arrow["{\alpha_{A , B\ot C , D}}", from=1-1, to=1-3]
	\arrow["{A \ot \alpha_{B , C ,D}}", from=1-3, to=2-3]
	\arrow["{\alpha_{A ,B ,C\ot D}}"', from=2-2, to=2-3]
	\arrow["{\alpha_{A \ot B , C , D}}"', from=2-1, to=2-2]
	\arrow["{\alpha_{A , B ,C} \ot D}", from=2-1, to=1-1]
\end{tikzcd}
\end{center}
\begin{center}
  (c1)
  % https://q.uiver.app/?q=WzAsMyxbMSwwLCJcXEkgXFxsb2xsaSBcXEkiXSxbMCwxLCJcXEkiXSxbMiwxLCJcXEkiXSxbMSwyLCIiLDAseyJsZXZlbCI6Miwic3R5bGUiOnsiaGVhZCI6eyJuYW1lIjoibm9uZSJ9fX1dLFsxLDAsImpfe1xcSX0iXSxbMCwyLCJpX3tcXEl9Il1d
\begin{tikzcd}
	& {\I \lolli \I} \\
	\I && \I
	\arrow[Rightarrow, no head, from=2-1, to=2-3]
	\arrow["{j_{\I}}", from=2-1, to=1-2]
	\arrow["{i_{\I}}", from=1-2, to=2-3]
\end{tikzcd}
(c2)
% https://q.uiver.app/?q=WzAsNCxbMCwwLCJBXFxsb2xsaSBDIl0sWzMsMCwiKEEgXFxsb2xsaSBBKSBcXGxvbGxpIChBIFxcbG9sbGkgQykiXSxbMywxLCJcXEkgXFxsb2xsaSAoQSBcXGxvbGxpIEMpIl0sWzAsMSwiQSBcXGxvbGxpIEMiXSxbMCwxLCJcXG1zZkxfe0EgLCBBICxDfSJdLFsxLDIsImpfQSBcXGxvbGxpIChBIFxcbG9sbGkgQykiXSxbMiwzLCJpX3tBIFxcbG9sbGkgQ30iXSxbMywwLCIiLDAseyJsZXZlbCI6Miwic3R5bGUiOnsiaGVhZCI6eyJuYW1lIjoibm9uZSJ9fX1dXQ==
\begin{tikzcd}
	{A\lolli C} &&& {(A \lolli A) \lolli (A \lolli C)} \\
	{A \lolli C} &&& {\I \lolli (A \lolli C)}
	\arrow["{\msfL_{A , A ,C}}", from=1-1, to=1-4]
	\arrow["{j_A \lolli (A \lolli C)}", from=1-4, to=2-4]
	\arrow["{i_{A \lolli C}}", from=2-4, to=2-1]
	\arrow[Rightarrow, no head, from=2-1, to=1-1]
\end{tikzcd}

(c3)
% https://q.uiver.app/?q=WzAsMyxbMCwwLCJCIFxcbG9sbGkgQiJdLFsyLDAsIihBIFxcbG9sbGkgQikgXFxsb2xsaSAoQSBcXGxvbGxpIEIpIl0sWzEsMSwiXFxJIl0sWzAsMSwiXFxtc2ZMX3tBICxCICwgQn0iXSxbMiwxLCJqX3tBIFxcbG9sbGkgQn0iLDJdLFsyLDAsImpfQiJdXQ==
\begin{tikzcd}
	{B \lolli B} && {(A \lolli B) \lolli (A \lolli B)} \\
	& \I
	\arrow["{\msfL_{A ,B , B}}", from=1-1, to=1-3]
	\arrow["{j_{A \lolli B}}"', from=2-2, to=1-3]
	\arrow["{j_B}", from=2-2, to=1-1]
\end{tikzcd}

(c4)
% https://q.uiver.app/?q=WzAsMyxbMCwwLCJCXFxsb2xsaSBDIl0sWzIsMCwiKFxcSSBcXGxvbGxpIEIpIFxcbG9sbGkgKFxcSSBcXGxvbGxpIEMpIl0sWzEsMSwiKFxcSSBcXGxvbGxpIEIpIFxcbG9sbGkgQyJdLFswLDEsIlxcbXNmTF97XFxJICwgQiAsQ30iXSxbMSwyLCIoXFxJIFxcbG9sbGkgQikgXFxsb2xsaSBpX3tDfSJdLFswLDIsImlfe0J9IFxcbG9sbGkgQyIsMl1d
\begin{tikzcd}
	{B\lolli C} && {(\I \lolli B) \lolli (\I \lolli C)} \\
	& {(\I \lolli B) \lolli C}
	\arrow["{\msfL_{\I , B ,C}}", from=1-1, to=1-3]
	\arrow["{(\I \lolli B) \lolli i_{C}}", from=1-3, to=2-2]
	\arrow["{i_{B} \lolli C}"', from=1-1, to=2-2]
\end{tikzcd}

(c5)
% https://q.uiver.app/?q=WzAsNSxbMCwwLCJDIFxcbG9sbGkgRCJdLFsyLDAsIihBIFxcbG9sbGkgQykgXFxsb2xsaSAoQSBcXGxvbGxpIEQpIl0sWzIsMSwiKChBIFxcbG9sbGkgQikgXFxsb2xsaSAoQSBcXGxvbGxpIEMpKSBcXGxvbGxpICgoQSBcXGxvbGxpIEIpIFxcbG9sbGkgKEEgXFxsb2xsaSBEKSkiXSxbMiwyLCIoQiBcXGxvbGxpIEMpIFxcbG9sbGkgKChBIFxcbG9sbGkgQikgXFxsb2xsaSAoQSBcXGxvbGxpIEQpKSJdLFswLDIsIihCXFxsb2xsaSBDKSBcXGxvbGxpIChCIFxcbG9sbGkgRCkiXSxbMCwxLCJcXG1zZkxfe0EgLCBDICwgRH0iXSxbMSwyLCJcXG1zZkxfe0FcXGxvbGxpIEIgLCBBIFxcbG9sbGkgQyAsIEEgXFxsb2xsaSBEfSJdLFsyLDMsIlxcbXNmTF97QSwgQiwgQ30gXFxsb2xsaSAoKEEgXFxsb2xsaSBCKSBcXGxvbGxpIChBIGxvbGxpIEQpKSJdLFs0LDMsIihCIFxcbG9sbGkgQykgXFxsb2xsaSBcXG1zZkxfe0EgLCBCICxEfSIsMl0sWzAsNCwiXFxtc2ZMX3tCICwgQyAsRH0iLDJdXQ==
\begin{tikzcd}
	{C \lolli D} && {(A \lolli C) \lolli (A \lolli D)} \\
	&& {((A \lolli B) \lolli (A \lolli C)) \lolli ((A \lolli B) \lolli (A \lolli D))} \\
	{(B\lolli C) \lolli (B \lolli D)} && {(B \lolli C) \lolli ((A \lolli B) \lolli (A \lolli D))}
	\arrow["{\msfL_{A , C , D}}", from=1-1, to=1-3]
	\arrow["{\msfL_{A\lolli B , A \lolli C , A \lolli D}}", from=1-3, to=2-3]
	\arrow["{\msfL_{A, B, C} \lolli ((A \lolli B) \lolli (A \lolli D))}", from=2-3, to=3-3]
	\arrow["{(B \lolli C) \lolli \msfL_{A , B ,D}}"', from=3-1, to=3-3]
	\arrow["{\msfL_{B , C ,D}}"', from=1-1, to=3-1]
\end{tikzcd}
\end{center}
\begin{center}
  (a1)
  % https://q.uiver.app/?q=WzAsNSxbMCwwLCIoQSBcXG90IChCIFxcb3QgQykpIFxcbG9sbGkgRCJdLFsyLDAsIkEgXFxsb2xsaSAoKEIgXFxvdCBDKSBcXGxvbGxpIEQpIl0sWzIsMSwiQSBcXGxvbGxpIChCIFxcbG9sbGkgKEMgXFxsb2xsaSBEKSkiXSxbMCwxLCIoKEEgXFxvdCBCKSBcXG90IEMpIFxcbG9sbGkgRCJdLFsxLDEsIihBIFxcb3QgQikgXFxsb2xsaSAoQyBcXGxvbGxpIEQpIl0sWzAsMSwicF97QSAsIEIgXFxvdCBDICwgRH0iXSxbMSwyLCJcXGlkIFxcbG9sbGkgcF97QiAsIEMgLER9Il0sWzAsMywiXFxhbHBoYV97QSAsIEIgLCBDfSBcXGxvbGxpIFxcaWQiLDJdLFszLDQsInBfe0EgXFxvdCBCICwgQyAsIER9IiwyXSxbNCwyLCJwX3tBICwgQiAsIEMgXFxsb2xsaSBEfSIsMl1d
\begin{tikzcd}
	{(A \ot (B \ot C)) \lolli D} && {A \lolli ((B \ot C) \lolli D)} \\
	{((A \ot B) \ot C) \lolli D} & {(A \ot B) \lolli (C \lolli D)} & {A \lolli (B \lolli (C \lolli D))}
	\arrow["{p_{A , B \ot C , D}}", from=1-1, to=1-3]
	\arrow["{\id \lolli p_{B , C ,D}}", from=1-3, to=2-3]
	\arrow["{\alpha_{A , B , C} \lolli \id}"', from=1-1, to=2-1]
	\arrow["{p_{A \ot B , C , D}}"', from=2-1, to=2-2]
	\arrow["{p_{A , B , C \lolli D}}"', from=2-2, to=2-3]
\end{tikzcd}

  (a2)
  % https://q.uiver.app/?q=WzAsMyxbMCwwLCJBIFxcbG9sbGkgQyJdLFsxLDEsIigoQiBcXGxvbGxpIEEpIFxcb3QgQikgXFxsb2xsaSBDIl0sWzIsMCwiKEIgXFxsb2xsaSBBKSBcXGxvbGxpIChCIFxcbG9sbGkgQykiXSxbMCwyLCJcXG1zZkxfe0EgLEIsIEN9Il0sWzEsMiwicF97QiBcXGxvbGxpIEEgLCBCICwgQ30iLDJdLFswLDEsIlxcZXBzaWxvbl97QiwgQX0gXFxsb2xsaSBcXGlkIiwyXV0=
\begin{tikzcd}
	{A \lolli C} && {(B \lolli A) \lolli (B \lolli C)} \\
	& {((B \lolli A) \ot B) \lolli C}
	\arrow["{\msfL_{A ,B, C}}", from=1-1, to=1-3]
	\arrow["{p_{B \lolli A , B , C}}"', from=2-2, to=1-3]
	\arrow["{\epsilon_{B, A} \lolli \id}"', from=1-1, to=2-2]
\end{tikzcd}

(a3)
% https://q.uiver.app/?q=WzAsMyxbMCwwLCIoQSBcXG90IEIpIFxcbG9sbGkgQyJdLFsyLDAsIkEgXFxsb2xsaSAoQiBcXGxvbGxpIEMpIl0sWzEsMSwiKEIgXFxsb2xsaSAoQSBcXG90IEIpKSBcXGxvbGxpIChCIFxcbG9sbGkgQykiXSxbMiwxLCJcXGV0YV97QSAsQn0gXFxsb2xsaSBcXGlkIiwyXSxbMCwxLCJwX3tBICwgQiAsQ30iXSxbMCwyLCJcXG1zZkxfe0IgLCBBXFxvdCBCICwgQ30iLDJdXQ==
\begin{tikzcd}
	{(A \ot B) \lolli C} && {A \lolli (B \lolli C)} \\
	& {(B \lolli (A \ot B)) \lolli (B \lolli C)}
	\arrow["{\eta_{A ,B} \lolli \id}"', from=2-2, to=1-3]
	\arrow["{p_{A , B ,C}}", from=1-1, to=1-3]
	\arrow["{\msfL_{B , A\ot B , C}}"', from=1-1, to=2-2]
\end{tikzcd}

(a4)
% https://q.uiver.app/?q=WzAsMyxbMCwwLCIoQSBcXGxvbGxpIEMpIFxcb3QgKEIgXFxsb2xsaSBBKSJdLFsyLDAsIkIgXFxsb2xsaSBDIl0sWzEsMSwiKChCIFxcbG9sbGkgQSkgXFxsb2xsaSAoQiBcXGxvbGxpIEMpKSBcXG90IChCIFxcbG9sbGkgQSkiXSxbMiwxLCJcXGVwc2lsb25fe0IgXFxsb2xsaSBBICwgQiBcXGxvbGxpIEN9IiwyXSxbMCwxLCJcXG1hdGhzZntNfV97QiwgQSAsIEN9Il0sWzAsMiwiXFxtc2ZMX3tCICwgQSwgQ30gXFxvdCBcXGlkIiwyXV0=
\begin{tikzcd}
	{(A \lolli C) \ot (B \lolli A)} && {B \lolli C} \\
	& {((B \lolli A) \lolli (B \lolli C)) \ot (B \lolli A)}
	\arrow["{\epsilon_{B \lolli A , B \lolli C}}"', from=2-2, to=1-3]
	\arrow["{\mathsf{M}_{B, A , C}}", from=1-1, to=1-3]
	\arrow["{\msfL_{B , A, C} \ot \id}"', from=1-1, to=2-2]
\end{tikzcd}

(a5)
% https://q.uiver.app/?q=WzAsMyxbMCwwLCJBIFxcbG9sbGkgQyJdLFsyLDAsIihCIFxcbG9sbGkgQSkgXFxsb2xsaSAoQiBcXGxvbGxpIEMpIl0sWzEsMSwiKEIgXFxsb2xsaSBBKSBcXGxvbGxpICgoQSBcXGxvbGxpIEMpIFxcb3QgKEIgXFxsb2xsaSBBKSkiXSxbMCwxLCJcXG1zZkxfe0IsIEEgLCBDfSJdLFsyLDEsIlxcaWQgXFxsb2xsaSBcXG1hdGhzZntNfV97QiAsIEEgLEN9ICIsMl0sWzAsMiwiXFxldGFfe0EgXFxsb2xsaSBDICwgQiBcXGxvbGxpIEF9IiwyXV0=
\begin{tikzcd}
	{A \lolli C} && {(B \lolli A) \lolli (B \lolli C)} \\
	& {(B \lolli A) \lolli ((A \lolli C) \ot (B \lolli A))}
	\arrow["{\msfL_{B, A , C}}", from=1-1, to=1-3]
	\arrow["{\id \lolli \mathsf{M}_{B , A ,C} }"', from=2-2, to=1-3]
	\arrow["{\eta_{A \lolli C , B \lolli A}}"', from=1-1, to=2-2]
\end{tikzcd}

(a6)
% https://q.uiver.app/?q=WzAsMyxbMCwwLCJcXEkiXSxbMiwwLCJBIFxcbG9sbGkgQSJdLFsxLDEsIkEgXFxsb2xsaSAoXFxJIFxcb3QgQSkiXSxbMCwxLCJqX3tBfSJdLFsyLDEsIlxcaWQgXFxsb2xsaSBcXGxhbWJkYV97QX0iLDJdLFswLDIsIlxcZXRhX3tcXEkgLCBBfSIsMl1d
\begin{tikzcd}
	\I && {A \lolli A} \\
	& {A \lolli (\I \ot A)}
	\arrow["{j_{A}}", from=1-1, to=1-3]
	\arrow["{\id \lolli \lambda_{A}}"', from=2-2, to=1-3]
	\arrow["{\eta_{\I , A}}"', from=1-1, to=2-2]
\end{tikzcd}
(a7)
% https://q.uiver.app/?q=WzAsMyxbMCwwLCJcXEkgXFxvdCBBIl0sWzIsMCwiQSJdLFsxLDEsIihBIFxcbG9sbGkgQSkgXFxvdCBBIl0sWzAsMSwiXFxsYW1iZGFfe0F9Il0sWzIsMSwiXFxlcHNpbG9uX3tBLCBBfSIsMl0sWzAsMiwial97QX0gXFxvdCBcXGlkIiwyXV0=
\begin{tikzcd}
	{\I \ot A} && A \\
	& {(A \lolli A) \ot A}
	\arrow["{\lambda_{A}}", from=1-1, to=1-3]
	\arrow["{\epsilon_{A, A}}"', from=2-2, to=1-3]
	\arrow["{j_{A} \ot \id}"', from=1-1, to=2-2]
\end{tikzcd}

(a8)
% https://q.uiver.app/?q=WzAsMyxbMCwwLCIoQSBcXG90IFxcSSkgXFxsb2xsaSBCIl0sWzIsMCwiQSBcXGxvbGxpIEIiXSxbMSwxLCJBIFxcbG9sbGkgKFxcSSBcXGxvbGxpIEIpIl0sWzAsMSwiXFxyaG9fe0F9IFxcbG9sbGkgXFxpZCJdLFsyLDEsIlxcaWQgXFxsb2xsaSBpX3tCfSIsMl0sWzAsMiwicF97QSAsIFxcSSAsIEJ9IiwyXV0=
\begin{tikzcd}
	{(A \ot \I) \lolli B} && {A \lolli B} \\
	& {A \lolli (\I \lolli B)}
	\arrow["{\rho_{A} \lolli \id}", from=1-1, to=1-3]
	\arrow["{\id \lolli i_{B}}"', from=2-2, to=1-3]
	\arrow["{p_{A , \I , B}}"', from=1-1, to=2-2]
\end{tikzcd}
\end{center}
Equations (m1)-(m5) are called Mac Lane axiomsm.
On the other hand, (c1)-(c5) are internal hom variants of Mac Lan Axioms.
(a1)-(a8) are from \cite{street:skew-closed:2013}, and aims to establish bijections bewteen natural transformations(e.g. there is a bijection among $i$ and $\lambda$).
Notice that (a1) and (a8) are presented in hom-set form in original paper but they are rewritten into internal hom representations here.

Give a set $\mathsf{At}$ which contains countably infinite atomic formulae, we can generate a free skew monoidal closed category according to following rules:
\begin{displaymath}
    \infer[\id]{A \Longrightarrow A}{}
    \quad
    \infer[\mathsf{comp}]{A \Longrightarrow C}{
      A \Longrightarrow B
      &
      B \Longrightarrow C
    }
\end{displaymath}
\begin{displaymath}
  \infer[\otimes]{A \ot B \Longrightarrow C \ot D}{
    A \Longrightarrow C
    &
    B \Longrightarrow D
  }
  \quad
  \infer[\lolli]{A \lolli B \Longrightarrow C \lolli D}{
    C \Longrightarrow A
    &
    B \Longrightarrow D
  }
\end{displaymath}
\begin{displaymath}
  \infer[\lambda]{\I \ot A \Longrightarrow A}{}
  \quad
  \infer[\rho]{A \Longrightarrow A \ot \I}{}
  \quad
  \infer[\alpha]{(A \ot B) \ot C \Longrightarrow A \ot (B \ot C)}{}
\end{displaymath}
\begin{displaymath}
  \infer[i]{\I \lolli A \Longrightarrow A}{}
  \quad
  \infer[j]{\I \Longrightarrow A \lolli A}{}
  \quad
  \infer[\mathsf{L}]{B \lolli C \Longrightarrow (A \lolli B) \lolli (A \lolli C)}{}
\end{displaymath}
\begin{displaymath}
  \infer[\eta]{A \Longrightarrow B \lolli (A \ot B)}{}
  \quad
  \infer[\epsilon]{(A \lolli B) \ot A \Longrightarrow B}{}
  \quad
  \infer[p]{(A \ot B) \lolli C \Longrightarrow A \lolli (B \lolli C)}{}
\end{displaymath}
\begin{displaymath}
  \infer[\mathsf{M}]{(B \lolli C) \ot (A \lolli B) \Longrightarrow A \lolli C}{}
  \quad
  \infer[\mathsf{adj_1}]{A \Longrightarrow B \lolli C}{A \ot B \Longrightarrow C}
  \quad
  \infer[\mathsf{adj_2}]{A \ot B \Longrightarrow C}{A \Longrightarrow B \lolli C}
\end{displaymath}
Following equations are satisfied:
\begin{align*}
  \id \circ f \doteq f \qquad f \doteq f \circ \id \qquad (f \circ g) \circ h \doteq f \circ (g \circ h) &&\text{(category laws)}
  \\
  \id \ot \id \doteq \id \qquad (h \circ f) \ot (k \circ g) \doteq h \ot k \circ f \ot g &&(\ot \text{ functorial})
  \\
  \id \lolli \id \doteq \id \qquad (h \circ f) \ot (k \circ g) \doteq f \lolli k \circ h \lolli g &&(\lolli \text{ functorial})
\end{align*}
\begin{align*}
    \lambda \circ \id \ot f \doteq f \circ \lambda \qquad \rho \circ f \doteq f \ot \id \circ \rho
    \\
    \alpha \circ (f \ot g) \ot h \doteq f \ot (g \ot h) \circ \alpha
    \\
    i \circ \id \lolli f \doteq f \circ i \qquad f \lolli \id \circ j \doteq \id \lolli f \circ j && \lambda , \rho , \alpha , i , j , \msfL , \eta , \epsilon , p , \mathsf{M} \text{ are}
    \\
    (f \lolli g) \lolli (\id \lolli h) \circ \msfL \doteq \id \lolli (f \lolli \id) \circ \msfL \circ g \lolli h &&\text{(extra)natural transformations}
    \\
    f \circ \epsilon \doteq \epsilon \circ (\id \lolli f) \ot \id
    \qquad
    \eta \circ f \doteq \id \lolli (f \ot \id) \circ \eta
    \\
    \id \lolli (\id \lolli f) \circ p \doteq p  \circ (\id \ot \id) \lolli f
    \\
    \id \lolli f \circ \mathsf{M} \doteq \mathsf{M} \circ (\id \lolli \id) \ot (\id \lolli f)
    %%%%Extra natural version?
    %f \lolli (g \lolli h) \circ p \doteq p \circ (f \ot g) \lolli h
    %\\
    %f \lolli h \circ \mathsf{M} \doteq \mathsf{M} \circ (g \lolli h) \ot (f \lolli g)
    %%%%Extra natural version?
\end{align*}
\begin{align*}
  \id \lolli \msfL \circ \msfL \doteq \msfL \lolli \id \circ \msfL \circ \msfL
  \\
  i \circ j \lolli \id \circ \msfL \doteq \id \qquad \msfL \circ j \doteq j
  &&\text{c1-c5}
  \\
  \id \lolli i \circ \msfL \doteq i \lolli \id \qquad i \circ j \doteq \id
\end{align*}
\begin{align*}
    \lambda \circ \rho \doteq \id \qquad \id \doteq \id \ot \lambda \circ \alpha \circ \rho \ot \id
    \\
    \lambda \circ \alpha \doteq \lambda \ot \id \qquad \alpha \circ \rho \doteq \id \ot \rho
    &&\text{m1-m5}
    \\
    \alpha \circ \alpha \doteq
    \id \ot \alpha \circ \alpha \circ \alpha \ot \id
\end{align*}
\begin{align*}
    \id \lolli p \circ p \doteq p \circ p \circ \alpha \lolli \id
    \qquad
    \msfL \doteq p \circ \epsilon \lolli \id
    \\
    p \doteq \eta \lolli \id \circ \msfL
    \qquad
    \mathsf{M} \doteq \epsilon \circ \msfL \ot \id
    \qquad
    \msfL \doteq \id \lolli \mathsf{M} \circ \eta
    &&\text{a1-a8}
    \\
    j \doteq \id \lolli \lambda \circ \eta
    \qquad
    \lambda \doteq \epsilon \circ j \ot \id
    \qquad
    \id \lolli i \circ p \doteq \rho \lolli \id
\end{align*}
\begin{fact}
  We do not need all rules above to generate a free skew monoidal closed category.
  For example, suppose we have $\lambda , \rho , \alpha , \epsilon$, and $\mathsf{adj_1}$, then $i, j, \mathsf{L}, \eta$, and  $\mathsf{adj_2}$ are admissible.
  \begin{align*}
    i &\defeq \mathsf{comp} \text{ } (\rho ,  (\mathsf{adj_2} \text{ } \id))
    \\
    j &\defeq \mathsf{adj_1} \text{ } \lambda
    \\
    \msfL &\defeq \mathsf{adj_1} \text{ } (\mathsf{adj_1} \text{ } (\mathsf{comp} \text{ } (\alpha , (\mathsf{comp} \text{ } (\id \ot \epsilon) , \epsilon))))
    \\
    \eta &\defeq \mathsf{adj_1} \text{ } \id
    \\
    \mathsf{adj_2} &\defeq \comp \text{ } ((f \ot \id), \epsilon) && f : A \Longrightarrow B \lolli C
    \\
    p &\defeq \comp \text{ } (\msfL , (\eta \lolli \id))
    \\
    \mathsf{M} &\defeq \comp \text{ } ((\msfL \ot \id) , \epsilon)
  \end{align*}
  This fact represents the bijections between natural transformations in skew monoidal closed categories.
  %Moreover, in \cite{street:skew-closed:2013} there are two natural transformations $p$ and $\mathsf{M}$.
  %However these two natural transformations and their corresponding morphism equivalences could be defined mutually with $\msfL$ so we do not present them explicitly here.
\end{fact}

We show soundness by induction on derivations in sequent calculus.
Notice that soundness of skew monoidal categories and skew closed categories are proved respectively in \cite{uustalu:sequent:2018} and \cite{uustalu:deductive:nodate}.
However, there are two different ways to interpret sequents in these two papers.
We pick the interpretation in \cite{uustalu:sequent:2018} and show that soundness of $\lleft$ and $\lright$ also hold.
\begin{theorem}
  (Soundness)For any derivation $f : S \mid \Gamma \vdash C$, there is a derivation $\mathsf{sound}$ $f : \ldbc S \mid \Gamma \rdbc_{\ot} \Longrightarrow C$.
\end{theorem}
\begin{proof}
  Proof of soundness is by induction on $f : S \mid \Gamma \vdash C$.

  First case is $f = \lleft \text{ } f' f''$ where $f' : - \mid \Gamma \vdash A$ and $f'' : B \mid \Delta \vdash C$.
  We obtain two repeated application version of adjoint rules first $\mathsf{Gadj_1}$ and $\mathsf{Gadj_2}$(we use $\ldbc - \rdbc_{\ot}$ and $\ldbc - \rdbc_{\lolli}$ to indicate the interpretation):
  \begin{displaymath}
    \infer[\mathsf{Gadj_1}]{S \Longrightarrow \ldbc \Gamma \mid C\rdbc_{\lolli}}{\ldbc S \mid \Gamma \rdbc_{\ot} \Longrightarrow A}
    \quad
    \text{ and }
    \quad
    \infer[\mathsf{Gadj_2}]{\ldbc A \mid \Gamma \rdbc_{\ot} \Longrightarrow C}{A \Longrightarrow \ldbc \Gamma \mid C\rdbc_{\lolli}}
  \end{displaymath}
  Next, for any $f' : - \mid \Gamma \vdash A$ and $f'' : B\mid \Delta \vdash C$:
  \begin{displaymath}
    \infer[\lleft]{A \lolli B \mid \Gamma , \Delta \vdash C}{
      \deduce{- \mid \Gamma \vdash A}{f'}
      &
      \deduce{B \mid \Delta \vdash C}{f''}
    }
  \end{displaymath}
  \begin{displaymath}
  \defeq
  \quad
  \vcenter{
    \infer[\mathsf{Gadj_2}]{\ldbc A\multimap B \mid \Gamma , \Delta \rdbc_{\ot} \Longrightarrow C}{
      \infer[\mathsf{comp}]{\ldbc A\multimap B \mid \Gamma \rdbc_{\ot} \Longrightarrow \ldbc \Delta \mid C\rdbc_{\lolli}}{
        \infer{\ldbc A\multimap B \mid \Gamma \rdbc_{\ot} \Longrightarrow A\multimap B \otimes \ldbc - \mid \Gamma \rdbc_{\ot}}{\text{Lemma 4.3 in \cite{uustalu:sequent:2018}}}
        &
        \infer[\mathsf{Gadj_2}]{A\multimap B \otimes \ldbc - \mid \Gamma \rdbc_{\ot} \Longrightarrow \ldbc \Delta \mid C\rdbc_{\lolli}}{
          \infer[\lolli]{A\multimap B \Longrightarrow \ldbc - \mid \Gamma \rdbc_{\ot} \multimap \ldbc \Delta \mid C\rdbc_{\lolli}}{
            \deduce{\ldbc - \mid \Gamma \rdbc_{\ot} \Longrightarrow A}{\mathsf{sound} \text{ } f}
            &
            \infer[\mathsf{Gadj_1}]{B \Longrightarrow \ldbc \Delta \mid C \rdbc_{\lolli}}{
              \deduce{\ldbc B \mid \Delta \rdbc_{\ot} \Longrightarrow C}{\mathsf{sound} \text{ } g}
            }
          }
        }
      }
    }
  }
  \end{displaymath}
  The other case is $f = \lright \text{ } f'$ where $f' : S \mid \Gamma , A \vdash B$:
  \begin{displaymath}
  \vcenter{
    \infer[\lright]{S \mid \Gamma \vdash A \lolli B}{
    \deduce{S \mid \Gamma , A \vdash B}{f'}
    }
  }
    \quad
    \defeq
    \quad
    \vcenter{
    \infer[\mathsf{adj_1}]{\ldbc S \mid \Gamma \rdbc_{\ot} \Longrightarrow A \lolli B}{
      \infer[\text{Definition of } \ldbc - \rdbc_{\ot}]{\ldbc S \mid \Gamma \rdbc_{\ot} \ot A \Longrightarrow B}{
        \deduce{\ldbc S \mid \Gamma , A \rdbc_{\ot} \Longrightarrow B}{\mathsf{sound} \text{ } f'}
      }
    }
  }
  \end{displaymath}
\end{proof}
\begin{theorem}
  (Weak completeness) For any derivation $f : A\Longrightarrow C$, there is a derivation $\mathsf{complt}$ $f : A \mid \quad \vdash C$.
\end{theorem}
\begin{proof}
In previous studies from Uustalu et al. (\cite{uustalu:sequent:2018} and \cite{uustalu:deductive:nodate}), main cases of completeness are proved.
Therefore, we only have to show new cases in skew monoidal closed categories.
In particular, new cases are unit, counit, $p$, $\mathsf{M}$, and two adjoint rules.
\begin{itemize}
  \item (Counit)Case $f = \epsilon : (A\multimap B)\otimes A \Longrightarrow B$. We define:
  \begin{displaymath}
    \mathsf{complt} \left(
                    \infer[\epsilon]{(A\multimap B)\otimes A \Longrightarrow B}{}
                    \right)
                    \quad
                    \defeq
                    \quad
                    \vcenter{
                    \infer[\tl]{(A\multimap B) \otimes A \mid \quad \vdash B}{
                      \infer[\lleft]{A\multimap B \mid A \vdash B}{
                        \infer[\pass]{- \mid A \vdash A}{
                          \infer[\ax]{A \mid \quad \vdash A}{}
                        }
                        &
                        \infer[\ax]{B \mid \quad \vdash B}{}
                      }
                    }
                    }
  \end{displaymath}
  \item (Unit)Case $f = \eta : A \Longrightarrow B \multimap (A\otimes B)$. We define:
  \begin{displaymath}
    \mathsf{complt} \left(
                    \infer[\eta]{A \Longrightarrow B \lolli (A \ot B)}{}
                    \right)
                    \quad
                    =_{\text{df}}
                    \quad
                    \vcenter{
                    \infer[\multimap \mathsf{R}]{A \mid \quad \vdash B \multimap (A\otimes B)}{
                      \infer[\tr]{A \mid B \vdash A\otimes B}{
                        \infer[\ax]{A \mid \quad \vdash A}{}
                        &
                        \infer[\pass]{- \mid B \vdash B}{
                          \infer[\ax]{B \mid \quad \vdash B}{}
                        }
                      }
                    }
                    }
  \end{displaymath}
  \item ($p$, internal adjointness)Case $f = p : (A \ot B) \lolli C \Longrightarrow A \lolli (B \lolli C)$. We define:
  \begin{displaymath}
    \mathsf{complt} \left(
                    \infer[p]{(A \ot B) \lolli C \Longrightarrow A \lolli (B \lolli C)}{}
                    \right)
                    \quad
                    =_{\text{df}}
                    \quad
                    \vcenter{
                    \infer[\lright]{(A \ot B) \lolli C \mid \quad \vdash A \lolli (B \lolli C)}{
                      \infer[\lright]{(A \ot B) \lolli C \mid A \vdash B \lolli C}{
                        \infer[\lleft]{(A \ot B) \lolli C \mid A , B \vdash C}{
                          \infer[\pass]{- \mid A , B \vdash A \ot B}{
                            \infer[\tr]{A \mid B \vdash A \ot B}{
                              \infer[\ax]{A \mid \quad \vdash A}{}
                              &
                              \infer[\pass]{- \mid B \vdash B}{
                                \infer[\ax]{B \mid \quad \vdash B}{}
                              }
                            }
                          }
                          &
                          \infer[\ax]{C \mid \quad \vdash C}{}
                        }
                      }
                    }
                    }
  \end{displaymath}
  \item ($\mathsf{M}$, internal composition)Case $f = \mathsf{M} : (B \lolli C) \ot (A \lolli B) \Longrightarrow A \lolli C$. We define:
  \begin{displaymath}
    \mathsf{complt} \left(
                    \infer[\mathsf{M}]{(B \lolli C) \ot (A \lolli B) \Longrightarrow A \lolli C}{}
                    \right)
                    \quad
                    =_{\text{df}}
                    \quad
                    \vcenter{
    \infer[\lright]{(B \lolli C) \ot (A \lolli B) \mid \quad \vdash A \lolli C}{
      \infer[\tl]{(B \lolli C) \ot (A \lolli B) \mid A \vdash C}{
        \infer[\lleft]{B \lolli C \mid A \lolli B , A \vdash C}
        {
          \infer[\pass]{- \mid A \lolli B , A \vdash B}{
            \infer[\lleft]{A \lolli B \mid A \vdash B}{
              \infer[\pass]{- \mid A \vdash A}{
                \infer[\ax]{A \mid \quad \vdash A}{}
              }
              &
              \infer[\ax]{B \mid \quad \vdash B}{}
            }
          }
          &
          \infer[\ax]{C \mid \quad \vdash C}{}
        }
      }
    }
    }
  \end{displaymath}
\end{itemize}
Two adjoint rules need invertibility of $\otimes \mathsf{L}$ and $\multimap \mathsf{R}$ which are proved respectively in \cite{uustalu:sequent:2018} and \cite{uustalu:deductive:nodate}.
\begin{itemize}
  \item Case $f = \mathsf{adj_1}$ $g$ where $g : A\otimes B \Longrightarrow C$. We define:
  \begin{displaymath}
    \mathsf{complt} \left( \vcenter{
    \infer[\mathsf{adj_1}]{A\Longrightarrow B\multimap C}{
      \deduce{A\otimes B \Longrightarrow C}{g}
    }
    }
    \right)
                           \quad
                           =_{\text{df}}
                           \quad
                           \vcenter{
                           \infer[\lright]{A \mid \quad \vdash B \multimap C}{
                            \infer[\tl^{-1}]{A \mid B \vdash C}{
                              \deduce{A\otimes B \mid \quad \vdash C}{\mathsf{complt} \text{ } g}
                            }
                           }
                           }
  \end{displaymath}
  \item Case $f = \mathsf{adj_2}$ $g$ where $g : A \Longrightarrow B\multimap C$. We define:
  \begin{displaymath}
    \mathsf{complt} \left( \vcenter{
    \infer[\mathsf{adj_2}]{A \otimes B \Longrightarrow C}{
      \deduce{A \Longrightarrow B\multimap C}{g}
    }
    }
                           \right)
                           \quad
                           =_{\text{df}}
                           \quad
                           \vcenter{
                           \infer[\tl]{A \otimes B \mid \quad \vdash C}{
                            \infer[\lright^{-1}]{A \mid B \vdash C}{
                              \deduce{A \mid \quad \vdash B \multimap C}{\mathsf{complt} \text{ } g}
                            }
                           }
                           }
  \end{displaymath}
\end{itemize}
\end{proof}

\section{Focusing}
The focused sequent calculus here has four phases($\RI , \LI , \Pass$, and $\F$) with  special annotations $\bullet$(tag) on $\vdash$.
\begin{displaymath}
  \infer[\lright]{S \mid \Gamma \vdash^{x}_{\RI} A \lolli B}{S \mid \Gamma , A^{x} \vdash^{x}_{\RI} B}
  \quad
  \infer[\LI 2 \RI]{S \mid \Gamma \vdash^{x}_{\RI} P}{S \mid \Gamma \vdash^{x}_{\LI} P}
\end{displaymath}
\begin{displaymath}
  \infer[\unitl]{\I \mid \Gamma \xvdash_{\LI} P}{- \mid \Gamma \xvdash_{\LI} P}
  \quad
  \infer[\tl]{A \ot B \mid \Gamma \xvdash_{\LI} P}{A \mid B , \Gamma \xvdash_{\LI} P}
  \quad
  \infer[\Pass 2 \LI]{T \mid \Gamma \xvdash_{\LI} P}{T \mid \Gamma \xvdash_{\Pass} P}
\end{displaymath}
\begin{displaymath}
  \infer[\pass]{- \mid A^{x} , \Gamma \xvdash_{\Pass} P }{A^{\circ} \mid \Gamma \xvdash_{\LI} P}
  \quad
  \infer[\F 2 \Pass]{T \mid \Gamma \xvdash_{\Pass} P}{T \mid \Gamma \xvdash_{\F} P}
\end{displaymath}
\begin{displaymath}
  \infer[\tr]{T \mid \Gamma , \Delta \xvdash_{\F} A \ot B}{
    T \mid \Gamma \vdash^{\bullet}_{\RI} A
    &
    - \mid \Delta \vdash_{\RI} B
  }
  \quad
  \infer[\lleft]{A \lolli B \mid \Gamma , \Delta \xvdash_{\F} P}{
    - \mid \Gamma^{\circ} \vdash_{\RI} A
    &
    B \mid \Delta \vdash_{\LI} P
    &
    \text{if } x = \bullet \text{, then } \exists F^{\bullet} \in \Gamma
  }
\end{displaymath}
\begin{displaymath}
  \infer[\ax]{A \mid \quad \vdash_{\F} A}{}
  \quad
  \infer[\unitr]{- \mid \quad \vdash \I}{}
\end{displaymath}
We explain the intuition of this focused system from bottom-up proof search perspective.
One important thing is that even though we have tagged suquents in a proof, but the end sequent should keep tag free.
\begin{itemize}
  \item In phase $\RI$ (`for right invertible'), only right invertible rules can be applied, in particular, $\lright$.
  If conclusion of a $\RI$ application is black, then we add a tag to keep tracking the formulae decomposed from $\RI$.
  In the end, we decompose $C$ until it becomes a positive $P$ where $P \neq A \lolli B$, then move to phase $\LI$.
  \item Next, in phase $\LI$ (`for left invertible') we destruct $S$ by left invertible rules $\tl$ and $\unitl$ until we obtain an irreducible stoup $S'$ where $S'$ is not in $A' \ot B'$ or $\I$.
  In this phase we do not need to enforce the order between $\unitl$ and $\tl$ applicaitons because they are not exchangeable.
  \item Then we move to passivation phase $\Pass$, where only $\pass$ can be applied to sequents.
  Notice that for the active formula in each application of $\pass$, we remove its tag and the sequent goes back to phase $\LI$ to decompose stoup formula again because the passviated formula $A$ could be reducible.
  \item In phase $\F$ (for `focusing'), we have four rules, $\tr$, $\lleft$, $\ax$, and $\unitr$.
  Here we have special resrictions on $\tl$ and $\lleft$.
  In general, we prefer to decompose $\lolli$ in stoup prior then $\ot$ in conclusion.
  However, if there are some formula are packed into conclusion, then whole bottom-up proof strategy fails.
  We will see comparison between involved and naive focused system later.
  For rule $\tr$, we let its left and right premises back to phase $\RI$ because $A$ and $B$ could be negative formulas.
  Especially, the left premise becomes a tagged sequent.
  In a $\lleft$ application, if conclusion sequent is tagged, then at least exists a formula $F$ is black in the context of left premise.
  Remembering that we prefer $\lleft$ prior to $\tr$, so if there is any $\tr$ before $\lleft$, then we have to ensure that we cannot do $\lleft$ first.
  Black tags produced by $\lright$ play essential roles here because if there is no black formula in $\Gamma$, then it implies that we cannot do $\lleft$ after $\tr$.
  In this situation, we must decompose $\lolli$ in stoup prior than $\ot$ in conclusion.
\end{itemize}

Assume that we would prefer to decompose $\lolli$ in stoup prior than $\ot$ in succedent without tag setting above.
In this naive focused sequent calculus, we cannot prove $X \lolli Y \mid Z \vdash_{\RI} (X \lolli Y) \ot Z$, where $X, Y$, and $Z$ are atomic formulae:
\begin{displaymath}
  \infer[\mathsf{\LI 2 \RI}]{X \multimap Y \mid Z \vdash_{\mathsf{RI}} (X\multimap Y) \otimes Z}{
    \infer[\Pass 2 \LI]{X \multimap Y \mid Z \vdash_{\mathsf{LI}} (X\multimap Y) \otimes Z}{
      \infer[\F 2 \Pass]{X \multimap Y \mid Z \vdash_{\Pass} (X\multimap Y) \otimes Z}{
        \infer[\lleft]{X \multimap Y \mid Z \vdash_{\F} (X\multimap Y) \otimes Z}{
          \deduce{- \mid \quad \vdash_{\RI} X}{??}
          &
          \infer[\Pass 2 \LI]{Y \mid Z \vdash_{\LI} (X \lolli Y) \ot Z}{
            \infer[\F 2 \Pass]{Y \mid Z \vdash_{\Pass} (X \lolli Y) \ot Z}{
              \infer[\tr]{Y \mid Z \vdash_{\F} (X \lolli Y) \ot Z}{
                \infer[\lright]{Y \mid \quad \vdash_{\RI} X \lolli Y}{
                  \deduce{Y \mid X \vdash_{\RI} Y}{??}
                }
                &
                \infer[\LI 2 \RI]{- \mid Z \vdash_{\RI} Z}{
                  \infer[\Pass 2 \LI]{- \mid Z \vdash_{\LI} Z}{
                    \infer[\pass]{- \mid Z \vdash_{\Pass} Z}{
                      \infer[\Pass 2 \LI]{Z \mid \quad \vdash_{\LI} Z}{
                        \infer[\F 2 \Pass]{Z \mid \quad \vdash_{\Pass} Z}{
                          \infer[\ax]{Z \mid \quad \vdash_{\F} Z}{}
                        }
                      }
                    }
                  }
                }
              }
            }
          }
        }
      }
    }
  }
\end{displaymath}
The key point is that $X$ in $(X \lolli Y) \ot Z$ is locked, so that we could not move it to the context of left premise to close proof tree.
However, in tagged setting, we can avoid this awkward situation:
\begin{displaymath}
  \infer[\mathsf{\LI 2 \RI}]{X \multimap Y \mid Z \vdash_{\mathsf{RI}} (X\multimap Y) \otimes Z}{
    \infer[\Pass 2 \LI]{X \multimap Y \mid Z \vdash_{\mathsf{LI}} (X\multimap Y) \otimes Z}{
      \infer[\F 2 \Pass]{X \multimap Y \mid Z \vdash_{\Pass} (X\multimap Y) \otimes Z}{
        \infer[\tr]{X \multimap Y \mid Z \vdash_{\F} (X\multimap Y) \otimes Z}{
          \infer[\lright]{X \lolli Y \mid \quad \vdash^{\bullet}_{\RI} X \lolli Y}{
            \infer[\LI 2 \RI]{X \lolli Y \mid X^{\bullet} \vdash^{\bullet}_{\RI} Y}{
              \infer[\Pass 2 \LI]{X \lolli Y \mid X^{\bullet} \vdash^{\bullet}_{\LI} Y}{
                \infer[\F 2 \Pass]{X \lolli Y \mid X^{\bullet} \vdash^{\bullet}_{\Pass} Y}{
                  \infer[\lleft]{X \lolli Y \mid X^{\bullet} \vdash^{\bullet}_{\F} Y}{
                    \infer[\LI 2 \RI]{- \mid X \vdash_{\RI} X}{
                      \infer[\Pass 2 \LI]{- \mid X \vdash_{\LI} X}{
                        \infer[\pass]{- \mid X \vdash_{\Pass} X}{
                          \infer[\Pass 2 \LI]{X \mid \quad \vdash_{\LI} X}{
                            \infer[\F 2 \Pass]{X \mid \quad \vdash_{\Pass} X}{
                              \infer[\ax]{X \mid \quad \vdash_{\F} X}{}
                            }
                          }
                        }
                      }
                    }
                    &
                    \infer[\Pass 2 \LI]{Y \mid \quad \vdash_{\LI} Y}{
                    \infer[\F 2 \Pass]{Y \mid \quad \vdash_{\Pass} Y}{
                    \infer[\ax]{Y \mid \quad \vdash_{F} Y}{}
                    }
                      }
                    }
                  }
                }
              }
            }
            &
                \infer[\LI 2 \RI]{- \mid Z \vdash_{\RI} Z}{
                  \infer[\Pass 2 \LI]{- \mid Z \vdash_{\LI} Z}{
                    \infer[\pass]{- \mid Z \vdash_{\Pass} Z}{
                      \infer[\Pass 2 \LI]{Z \mid \quad \vdash_{\LI} Z}{
                        \infer[\F 2 \Pass]{Z \mid \quad \vdash_{\Pass} Z}{
                          \infer[\ax]{Z \mid \quad \vdash_{\F} Z}{}
                        }
                      }
                    }
                  }
                }
          }
              }
            }
          }
\end{displaymath}
In the mean time, our focused system does not contain redundant non-determinacy.
For example, as the proof equivalences above, $\tr$ and $\lleft$ rule applications could be swapped when $\Gamma, \Delta$, and $\Delta'$ separated properly.
Without loose of generality, we assume $\Gamma, \Delta$, and $\Lambda$ are sufficient and necessary in proving each sequent respectively.
In tagged focused calculus, such exchangeable situation would not happen.
Given a sequent $A \lolli B \mid \Gamma , \Delta , \Lambda \vdash_{\RI} C \ot D$ and supposed that $B$ is irreducible, $C = A' \lolli B'$,
we can construct a proof:
\begin{displaymath}
  \infer[\LI 2 \RI]{A \lolli B \mid \Gamma , \Delta , \Lambda \vdash_{\RI} (A' \lolli B') \ot D}{
    \infer[\Pass 2 \LI]{A \lolli B \mid \Gamma , \Delta , \Lambda \vdash_{\LI} (A' \lolli B') \ot D}{
      \infer[\F 2 \Pass]{A \lolli B \mid \Gamma , \Delta , \Lambda \vdash_{\Pass} (A' \lolli B') \ot D}{
        \infer[\lleft]{A \lolli B \mid \Gamma , \Delta , \Lambda \vdash_{\F} (A' \lolli B') \ot D}{
          \deduce{- \mid \Gamma \vdash_{\RI} A }{\vdots}
          &
          \infer[\LI 2 \RI]{B \mid \Delta , \Lambda \vdash_{\RI} (A' \lolli B') \ot D}{
            \infer[\LI 2 \RI]{B \mid \Delta , \Lambda \vdash_{\LI} (A' \lolli B') \ot D}{
              \infer[\Pass 2 \LI]{B \mid \Delta , \Lambda \vdash_{\LI} (A' \lolli B') \ot D}{
                \infer[\F 2 \Pass]{B \mid \Delta , \Lambda \vdash_{\Pass} (A' \lolli B') \ot D}{
                  \infer[\tr]{B \mid \Delta , \Lambda \vdash_{\F} (A' \lolli B') \ot D}{
                    \deduce{B \mid \Delta \vdash^{\bullet}_{\RI} A' \lolli B'}{\vdots}
                    &
                    \deduce{- \mid \Lambda \vdash_{\RI} D}{\vdots}
                  }
                }
              }
            }
          }
        }
      }
    }
  }
\end{displaymath}
However, the other way around(decompose $\ot$ prior to $\lolli$) is impossible.
Because the sequent is black, if we want to decompose $A \lolli B$, we have to decompose $A' \lolli B'$ first to obtain $A^{\bullet}$ in context then send $\Gamma , \Delta , A'$ to left premise of $\lleft$ application.
In the end, the proof tree cannot close:
\begin{displaymath}
  \infer[\LI 2 \RI]{A \lolli B \mid \Gamma , \Delta , \Lambda \vdash_{\RI} (A' \lolli B') \ot D}{
    \infer[\Pass 2 \LI]{A \lolli B \mid \Gamma , \Delta , \Lambda \vdash_{\LI} (A' \lolli B') \ot D}{
      \infer[\F 2 \Pass]{A \lolli B \mid \Gamma , \Delta , \Lambda \vdash_{\Pass} (A' \lolli B') \ot D}{
        \infer[\tr]{A \lolli B \mid \Gamma , \Delta , \Lambda \vdash_{\F} (A' \lolli B') \ot D}{
          \infer{A \lolli B \mid \Gamma , \Delta \vdash^{\bullet}_{\RI} A' \lolli B'}{
            \infer[\lright]{A \lolli B \mid \Gamma , \Delta , A'^{\bullet} \vdash^{\bullet}_{\RI} B'}{
              \infer[\LI 2 \RI]{A \lolli B \mid \Gamma , \Delta , A'^{\bullet} \vdash^{\bullet}_{\RI} B}{
                \infer[\Pass 2 \LI]{A \lolli B \mid \Gamma , \Delta , A'^{\bullet} \vdash^{\bullet}_{\LI} B}{
                  \infer[\F 2 \Pass]{A \lolli B \mid \Gamma , \Delta , A'^{\bullet} \vdash^{\bullet}_{\Pass} B}{
                    \infer[\lleft]{A \lolli B \mid \Gamma , \Delta , A'^{\bullet} \vdash^{\bullet}_{\F} B}{
                      \deduce{- \mid \Gamma , \Delta , A'^{\bullet} \vdash_{\RI}}{??}
                      &
                      \deduce{B \mid \quad \vdash_{\LI} B'}{??}
                    }
                  }
                }
              }
            }
          }
          &
          \deduce{- \mid \Lambda \vdash_{\RI} D}{\vdots}
        }
      }
    }
  }
\end{displaymath}
The situation is worse if $C$ is positive, we cannot produce tags from $\lright$ application to ensure applicability of $\lleft$, so the whole derivation gets stuck and cannont close.
This fact shows that in tagged focused system, we cannot permute $\tr$ and $\lleft$ applications arbitrarily.
In this sense, two equivalent proofs in SkNMILL become identical in focused calculus.
Therefore, tagged focused system here is not only sound but also without redundant non-determinacy.

Moreover, not only $\tr$ and $\lleft$ equivalent case becomes identical in focused system, but any other equivalent cases hold the same fact.
We can substantiate statement above by following results.
\begin{lemma}\label{AdmissibleInRI}
  Following rules are admissible in tagged focused system:
  \begin{displaymath}
    \infer[\unitl^{\RI}]{\I \mid \Gamma \vdash_{\RI} C}{- \mid \Gamma \vdash_{\RI} C}
    \quad
    \infer[\tl^{\RI}]{A \mid B, \Gamma \vdash_{\RI} C}{A \mid B, \Gamma \vdash_{\RI} C}
    \quad
    \infer[\pass^{\RI}]{- \mid \Gamma \vdash_{\RI} C}{A \mid \Gamma \vdash_{\RI} C}
  \end{displaymath}
  \begin{displaymath}
    \infer[\lleft^{\RI}]{A \lolli B \mid \Gamma , \Delta \vdash_{\RI} C}{
    - \mid \Gamma \vdash_{\RI} A
    &
    B \mid \Delta \vdash_{\RI} C
    }
    \quad
    \infer[\ax^{\RI}]{A \mid \quad \vdash_{\RI} A}{}
    \quad
    \infer[\unitr^{\RI}]{- \mid \quad \vdash_{\RI} \I}{}
  \end{displaymath}
  \begin{displaymath}
    \infer[\tr^{\RI}]{S \mid \Gamma , \Delta \vdash_{\RI} \ldbc \Gamma' \mid A \rdbc_{\lolli} \ot B}{
      S \mid \Gamma , \Gamma' \vdash_{\RI} A
      &
      - \mid \Delta \vdash_{\RI} B
    }
  \end{displaymath}
\end{lemma}
Inductive definition of formula $\ldbc \Gamma \mid C \rdbc_{\lolli}$ is from \cite{uustalu:deductive:nodate}:
\begin{displaymath}
  \ldbc \quad \mid C \rdbc_{\lolli} = C \qquad \ldbc A , \Gamma \mid C \rdbc_{\lolli} = A \lolli \ldbc \Gamma \mid C \rdbc_{\lolli}
\end{displaymath}

$\tr^{\RI}$ is a generalized version of $\tr$.
We endorse it in order to strengthen inductive hypothesis and make our proof smoothly.
For example, considering following proof tree, if we invoke simple version of $\tr$ in phase $\RI$, then we cannot proof its admissibility:
\begin{displaymath}
  \infer[\tr^{\RI}_s]{S \mid \Gamma , \Delta \vdash_{\RI} (A' \lolli B') \ot B}{
    \infer[\lright]{S \mid \Gamma \vdash_{\RI} A' \lolli B'}{
      \deduce{S \mid \Gamma , A' \vdash_{\RI} B'}{f}
    }
    &
    \deduce{- \mid \Delta \vdash_{\RI} B}{g}
  }
\end{displaymath}
In this case, we cannot permute $\tr^{\RI}_s$ up to use inductive hypothesis.
Therefore we need to have a stronger version of $\tr$ which performs $\tr$ and $\lright$ in the same time.
There is no generality loss because we can always take $\Gamma'$ as empty then reduce to simple $\tr$.

If we have a derivation in focused system, then we can define a function $\mathsf{emb}$ which erases phases and tags to obtain a derivation in original unfocused seqeuent calculus.
On the other hand, we can define a function $\mathsf{focus}$ to map each derivation in cut-free sequent calculus to a derivaiton in focused calculus because lemma \ref{AdmissibleInRI} holds.
Specifically, for each derivation in cut-free sequent calculus, function $\mathsf{focus}$ adds phase label $\RI$ on $\vdash$ and rename each rule into its $\RI$ version.

Moreover, we can show that two equivalent derivations in SkNMILL sequent calculus become identical in focused seqeunt calculus.
Therefore, deciding two proofs to be equivalent could be done by checking whether their corresponding focused derivation are identical.
%Next, we can extend this result to decide equality of morphisms in skew monoidal closed categories.

We take one more step to extend the results to categorical side.
According to categorical semantics section above, we have soundness and completeness in our hands, so we can construct an effective procedure to determine equivalence of morphisms in a free skew monoidal closed category.
\begin{enumerate}
  \item Given two moprhisms $f$ and $g$ in a free skew monoidal closed category.
  \item Use $\mathsf{complt}$ function to obtain two derivations $\mathsf{complt} \text{ } f$ and $\mathsf{complt} \text{ } g$ in sequent calculus system.
  \item Apply $\mathsf{complt} \text{ } f$ and $\mathsf{complt} \text{ } g$ to function $\mathsf{focus}$ respectively.
  \item If $\mathsf{focus} \text{ } (\mathsf{complt} \text{ } f) = \mathsf{focus} \text{ } (\mathsf{complt} \text{ } g)$, then $f \doteq g$, else $f \not\doteq g$.
\end{enumerate}

%%\begin{theorem}
%%  \begin{enumerate}
%%    \item For any derivation $f : S \mid \Gamma \vdash_{\mathsf{RI}} C$, there is a derivation $\mathsf{emb}$ $f : S \mid \Gamma \vdash C$.
%%    \item For any derivation $f : S \mid \Gamma \vdash C$, there is a derivation $\mathsf{focus}$ $f : S \mid \Gamma \vdash_{\mathsf{RI}} C$.
%%  \end{enumerate}
%%\end{theorem}
%%\begin{theorem}
%%  \begin{enumerate}
%%    \item For any $f : S \mid \Gamma \vdash_{\RI} C$, $\mathsf{focus} \text{ } (\mathsf{emb} \text{ } f) = f$.
%%    \item For any $f, g : S \mid \Gamma \vdash C$,  if $f \circeq g$, then $\mathsf{focus} \text{ } f = \mathsf{focus} \text{ } g$.
%%    \item For any $f : S \mid \Gamma \vdash C$, $\mathsf{emb} \text{ } (\mathsf{focus} \text{ } f) \circeq f$.
%%  \end{enumerate}
%%\end{theorem}

\section{Conclusion}

  \bibliographystyle{eptcs}
  \bibliography{docRefs}
\end{document}
