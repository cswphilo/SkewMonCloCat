%%*************************************%%
%%     A template and instructions     %%
%% concerning manuscripts submitted to %%
%%   BULLETIN OF THE SECTION OF LOGIC  %%
%%                                     %%
%%             ver 1.01                %%
%%          August 01, 2020            %%
%%*************************************%%


% ********** BSL PART: DO NOT TAMPER WITH IT! **********

\documentclass[manuscript]{BSLstyle} % This is the class for Bulletin of the Section of Logic.
									 % It should be loaded with the `manuscript' option.


% These fields will be filled by the editorial office

%\Presenter{John Smith} %Name of the editor who accepted the paper
%
%\Received{1}{1}{2020} %Date of the paper's submission (format {d(d)}{mm}{yyyy})
%
%\Ordinal{01} %Number of papers hitherto accepted in the current year + 1
%
%\PublishedOnline{1}{6}{2020} %Date of paper's online publication (format {d(d)}{m(m)}{yyyy})
%
%%\Page{1} %Number of the first page of the paper
%
%\Published{50}{1}{2021} %Volume, number, and year of the issue the paper is published in (format {vv}{n}{yyyy})
%
%\AuthAbbrv{a} %If the length of the sequence of authors' names in the header exceeds the header's length, it is possible to pick one of two abbreviated forms: (a) S. Jones, J. Smith, H. Anders (i.e., the abbreviated forms of all names -- if provided; argument: "a") or (b) Susan Jones \textit{et al.} (argument: "b")



% ********** AUTHORS' PART (optional) **********

% You can put your own macros or packages here. Avoid, however, loading the packages that are already pre-loaded by the BSLstyle class. A detailed list of such packages can be found in Section 2 of this template.

% Please do not tamper with any dimensions defined in the BSLstyle class!

\usepackage{mathrsfs}     % Additional packages that are not pre-loaded by BSLstyle.cls
\usepackage{bussproofs}	  %	and are necessary for proper compilation of the template.tex file.

\newcommand{\MILL}{$\mathtt{MILL}$}

% ********** TITLE PART (mandatory) **********

% The \AuthorEmail command has the following syntax: \AuthorEmail[#1]{#2}{#3}[#4], where:
% #1 is an optional argument which can contain an author's ORCID number in the format: dddd-dddd-dddd-dddd
% (an author is requested to fill this argument should they have an ORCID number),
% #2 is a mandatory argument which should contain an author's full name,
% #3 is an optional argument which can contain an author's name in an abbreviated form (in case the full name does not fit within the header)
% #4 is a mandatory argument which should contain an author's email address,
% #5 is an optional argument which can contain a thanks footnote text if an author wants to include one.
% If a paper is multi-authored, then each author should fill a separate \AuthorEmail field.
% The authors' names will be displayed in the same order as below.

% \AuthorEmail[0000-0000-0000-0000]{Susan Jones}{s.jones@univie.ac.at}[\textit{Thanks} note by Susan Jones.]
% \AuthorEmail{John Smith}[J. Smith]{john.smith@berkeley.edu}[\textit{Thanks} note by John Smith.]
% \AuthorEmail[9999-9999-9999-9999]{Hans Anders}[H. Anders]{anders@uva.nl}

\AuthorEmail{Tarmo Uustalu}{tarmo@ru.is}
\AuthorEmail{Niccol{\` o} Veltri}{niccolo@cs.ioc.ee}
\AuthorEmail{Cheng-Syuan Wan}{cswan@cs.ioc.ee}

% The \Affiliation command has the following syntax: \Affiliation{#1}{#2}{#3}{#4}{#5}, where:
% #1 is a mandatory argument which should contain a university's / institution's name,
% #2 is a mandatory argument which should contain an institute's / department's / faculty's name,
% #3 is a mandatory argument which should contain a work address in the format: postal code, street name and number,
% #4 is a mandatory argument which should contain the names of a town and country,
% #5 is a mandatory argument which should contain the indices of authors with this affiliation
% (e.g., if a given affiliation is assigned to authors 1 and 3, one should put "1,3" as #5).
% If there are multiple affiliations occurring in the paper, then each of them needs to be included in a separate \Affiliation command.
% In such a case the order in which subsequent \Affiliation commands are executed can be arbitrary.
% If a given argument is not applicable, leave it empty.

% \Affiliation{University of California}{Department of Philosophy}{94720-2390, 314 Moses Hall \#2390}{Berkeley, California, USA}{1,2}
% \Affiliation{University of Amsterdam}{Faculty of Science\\Institute for Logic, Language and Computation}{1098 XG, Science Park 107}{Amsterdam, The Netherlands}{3}
% \Affiliation{University of Vienna}{Institute of Mathematics\\ Kurt G\"odel Research Center for Mathematical Logic}{1090, Augasse 2--6, UZA 1 -- Building 2}{Vienna, Austria}{1}

\Affiliation{Reykjavik University}{Department of Computer Science}{102 Menntavegi 1 }{Reykjavik, Iceland}{1}
\Affiliation{Tallinn University of Technology}{Department of Software Science}{12618 Akadeemia tee 21b}{Tallinn, Estonia}{1,2,3}

% The \Title command has the following syntax: \Title[#1]{#2}[#3], where:
% #1 is an optional argument which can contain the paper's title in an abbreviated form if the full title does not fit within the header
% #2 is a mandatory argument which should contain the paper's title in a full form. If it needs to be broken into multiple lines, one can do that by inserting the "\\" symbols in the appropriate places,
% #3 is an optional argument which can contain a footnote text if an author wants to attach one to the title.

% \Title[Full title or abbreviated title (if provided)]{Full title \\ Full subtitle}[Title \textit{thanks} note.]

\Title[Proof Theory of Skew Non-Commutative (extended version)]{Proof Theory of Skew Non-Commutative}[Title \textit{thanks} note.]

\begin{document}

	\begin{abstract}
		The abstract should briefly summarize the contents of the paper and \textbf{should not contain any references}.
		
		If needed, it can be split into several paragraphs.
		
		The abstract should be followed by a list of keywords. The authors \textbf{must provide at least 3 keywords} which should be typed in lowercase (except for proper names) and separated with commas.
		
		The authors can \textbf{optionally} provide the \textit{Mathematical Subject Classification} codes by using the \texttt{msc} environment which has an \textbf{optional argument} indicating the year (the default value is \texttt{2010}). However, it is \textbf{not compulsory}.
	\end{abstract}
	\begin{keywords}
		keyword 1, keyword 2, keyword 3
	\end{keywords}
	\begin{msc}[2020]
		code 1, code 2, code 3
	\end{msc}


\section{Title part}

There are three commands in the preamble the authors are requested to use and fill with arguments:
\begin{itemize}
	\item \texttt{AuthorEmail}
	\item \texttt{Affiliation}
	\item \texttt{Title}.
\end{itemize}

\subsection{AuthorEmail}

The \texttt{\textbackslash AuthorEmail} command has the following syntax:
\begin{center}
\texttt{\textbackslash AuthorEmail[\#1]\{\#2\}[\#3]\{\#4\}[\#5]},
\end{center}
where:
\begin{description}
\item[\texttt{\#1}] is an \underline{optional} argument which can contain an author's \textbf{ORCID number} in the format: dddd-dddd-dddd-dddd (an author is requested to fill this argument should they have an ORCID number),
\item[\texttt{\#2}] is a \underline{mandatory} argument which should contain an author's \textbf{full name},
\item[\texttt{\#3}] is an \underline{optional} argument which can contain an author's name \textbf{in an abbreviated form} (in case the full name does not fit within the header),
\item[\texttt{\#4}] is a \underline{mandatory} argument which should contain an author's \textbf{email address},
\item[\texttt{\#5}] is an \underline{optional} argument which can contain a thanks footnote text if an author wants to include one.
\end{description}
If a paper is multi-authored, then each author should execute a separate\linebreak \texttt{\textbackslash AuthorEmail} executed. The order of authors' names will reflect the order in which commands have been executed.

\subsection{Affiliation}

The \texttt{\textbackslash Affiliation} command has the following syntax:
\begin{center}
	\texttt{\textbackslash Affiliation\{\#1\}\{\#2\}\{\#3\}\{\#4\}\{\#5\}},
\end{center}
where:
\begin{description}
	\item[\texttt{\#1}] is a \underline{mandatory} argument which should contain a \textbf{university's / institution's name},
	\item[\texttt{\#2}] is a \underline{mandatory} argument which should contain an \textbf{institute's / department's / faculty's name},
	\item[\texttt{\#3}] is a \underline{mandatory} argument which should contain a \textbf{work address in the format: postal code, street name and number},
	\item[\texttt{\#4}] is a \underline{mandatory} argument which should contain the \textbf{names of a town and country},
	\item[\texttt{\#5}] is a \underline{mandatory} argument which should contain the \textbf{indices of authors with this affiliation} (e.g., if a given affiliation is assigned to authors 1 and 3, one should put `1,3' as \texttt{\#5}).
\end{description}
If there are multiple affiliations occurring in the paper, then each of them needs to be included in a separate \texttt{\textbackslash Affiliation} command. In such a case the order in which subsequent \texttt{\textbackslash Affiliation} commands are executed can be arbitrary.

If a given argument is not applicable, leave it empty.

\subsection{Title}

The \texttt{\textbackslash Title} command has the following syntax:
\begin{center}
	\texttt{\textbackslash Title[\#1]\{\#2\}[\#3]},
\end{center}
where:
\begin{description}
	\item[\texttt{\#1}] is an \underline{optional} argument which can contain the \textbf{paper's title in an abbreviated form} if the full title does not fit within the header,
	\item[\texttt{\#2}] is a \underline{mandatory} argument which should contain\textbf{ paper's title in the full form}. If it needs to be broken into multiple lines, one can do that by inserting the `\texttt{\textbackslash\textbackslash}' commands in the appropriate places,
	\item[\texttt{\#3}] is an \underline{optional} argument which can contain a \textbf{footnote text} if the author wants to attach one to the title.
\end{description}


\section{Extra packages and commands}

The authors can define their own commands in the preamble of the document. They can also use additional packages, however since the \texttt{BSLstyle} class automatically loads certain packages, they are asked not to include these packages in the preamble as it can lead to compilation errors. The list of packages pre-loaded by \texttt{BSLstyle} is as follows:

\begin{center}
	\bgroup
	\renewcommand*{\arraystretch}{1.5}
	\begin{tabular}{lll}
	\textbullet\ \texttt{amsmath}&\qquad\textbullet\ \texttt{graphicx}&\qquad\textbullet\ \texttt{datetime}\\
	\textbullet\ \texttt{amsfonts}&\qquad\textbullet\ \texttt{enumitem}&\qquad\textbullet\ \texttt{totpages}\\
	\textbullet\ \texttt{amscd}&\qquad\textbullet\ \texttt{url}&\qquad\textbullet\ \texttt{fancyhdr}\\
	\textbullet\ \texttt{amssymb}&\qquad\textbullet\ \texttt{hyperref}&\qquad	\textbullet\ \texttt{footmisc}\\
	\textbullet\ \texttt{amsthm}&\qquad\textbullet\ \texttt{xargs}&\qquad\textbullet\ \texttt{setspace}\\
	\textbullet\ \texttt{caption}&\qquad\textbullet\ \texttt{titling}&\qquad\textbullet\ \texttt{textcase}\\
	\textbullet\ \texttt{etoolbox}&\qquad\textbullet\ \texttt{lineno}&\qquad\textbullet\ \texttt{xstring} \\
	\textbullet\ \texttt{cleveref}&\qquad\textbullet\ \texttt{aliascnt}&
	\end{tabular}
	\egroup
\end{center}


\section{Maths: Environments and formulas}

\subsection{Mathematics environments}

The authors are requested to use predefined mathematics environments the list of which is presented below:

\begin{itemize}
	\item \texttt{definition} (\texttt{\textbackslash begin\{definition\}\ldots\textbackslash end\{definition\}})
	\item \texttt{theorem} (\texttt{\textbackslash begin\{theorem\}\ldots\textbackslash end\{theorem\}})
	\item \texttt{remark} (\texttt{\textbackslash begin\{remark\}\ldots\textbackslash end\{remark\}})
	\item \texttt{proposition} (\texttt{\textbackslash begin\{proposition\}\ldots\textbackslash end\{proposition\}})
	\item \texttt{corollary} (\texttt{\textbackslash begin\{corollary\}\ldots\textbackslash end\{corollary\}})
	\item \texttt{fact} (\texttt{\textbackslash begin\{fact\}\ldots\textbackslash end\{fact\}})
	\item \texttt{conjecture} (\texttt{\textbackslash begin\{conjecture\}\ldots\textbackslash end\{conjecture\}})
	\item \texttt{lemma} (\texttt{\textbackslash begin\{lemma\}\ldots\textbackslash end\{lemma\}})
	\item \texttt{example} (\texttt{\textbackslash begin\{example\}\ldots\textbackslash end\{example\}})
	\item \texttt{claim} (\texttt{\textbackslash begin\{claim\}\ldots\textbackslash end\{claim\}})
	\item \texttt{proof} (\texttt{\textbackslash begin\{proof\}\ldots\textbackslash end\{proof\}}).
\end{itemize}
Each of the above-mentioned environments (except for \textit{proof}) has its unnumbered version which will be yielded if an astherisk is added to the environment command (e.g., \texttt{\textbackslash begin\{example\textsuperscript{*}\}\ldots\textbackslash end\{example\textsuperscript{*}\}}).

In case the author wants to use a mathematics environment which is not listed above, they should define in it in the preamble in a standard \texttt{amsthm}-manner and assign to it one of the following theorem styles:
\begin{itemize}
	\item \texttt{definition}
	\item \texttt{plain}
	\item \texttt{remark}.
\end{itemize}
The anchor for numbering should be \texttt{theorem}. For instance:
\begin{itemize}
\item[]\texttt{\textbackslash theoremstyle\{definition\}}
\item[]\texttt{\textbackslash newtheorem\{statement\}[theorem]\{Statement\}}
\end{itemize}
If one wants the newly introduced environment to properly cooperate with the \texttt{cleveref} package, they should put in the preamble the following lines of code (below the exemplary new environment is \texttt{Statement} defined in the \texttt{definition} style):
\begin{itemize}
	\item[]\texttt{\textbackslash theoremstyle\{definition\}}
	\item[]\texttt{\textbackslash newaliascnt\{Statement\}\{theorem\}}
	\item[]\texttt{\textbackslash newtheorem\{statement\}[Statement]\{Statement\}}
	\item[]\texttt{\textbackslash aliascntresetthe\{Statement\}}
	\item[]\texttt{\textbackslash crefname\{Statement\}\{statement\}\{statements\}}
\end{itemize}

Below come some examples of usage of mathematical environments.

\begin{definition}[Strong finite model property~{\cite[Sect. 6.2]{BlackburnEtAl01}}]
	Let $\Lambda$ be a normal modal logic, $\mathsf{M}$ a set of finitely based models such that $\Lambda=\Lambda_\mathsf{M}$, and $f$ a function mapping natural numbers to natural numbers. $\Lambda$ has the $f(n)$-size model property with respect to $\mathsf{M}$ if every $\Lambda$-consistent formula $\phi$ is satisfiable in a model in M containing at most $f(|\phi|)$ states.
	
	$\Lambda$ has the strong finite model property with respect to $\mathsf{M}$ if there is a computable function $f$ such that $\Lambda$ has the $f(n)$-size model property with respect to $\mathsf{M}$. $\Lambda$ has the polysize model property with respect to $\mathsf{M}$ if there is a polynomial $p$ such that $\Lambda$ has the $p(n)$-size model property with respect to $\mathsf{M}$.
	
	$\Lambda$ has the $f(n)$-size model property (respectively, strong finite model property, polysize model property) if there is a set of finitely based models $\mathsf{M}$ such that $\Lambda=\Lambda_\mathsf{M}$ and $\Lambda$ has the $f(n)$-size model property (respectively, strong finite model property, polysize model property) with respect to $\mathsf{M}$.
\end{definition}

\begin{lemma}[Zorn's Maximum Principle~\cite{Zorn35}]
	In a closed set $\mathfrak{A}$ of sets $A$ there exists at least one, $A^*$,
	not contained as a proper subset in any other $A\in\mathfrak{A}$. 
\end{lemma}

\begin{theorem}[McKinsey \& Tarski~\cite{McKinseyTarski44}]
	$\mathsf{S4}\vdash\varphi$ iff $\mathfrak{A}_X\vDash\varphi$ for every dense-in-itself metrizable space $X$.
\end{theorem}

\begin{conjecture*}[Goldbach]
	Every even integer greater than 2 can be expressed as the sum of two prime numbers.
\end{conjecture*}

\begin{remark}
	Every countable subset of $\mathbb{R}$ has Lebesgue measure 0.
\end{remark}

If a list (such as the \texttt{itemize} or \texttt{enumerate} environment) is placed at the beginning of a mathematical environment such as \texttt{definition}, \texttt{theorem}, \texttt{proof}, it automatically starts in a new line.

\begin{fact}[Axioms of ZFC~\cite{Kunen83}]
	\begin{enumerate}[label=\textsc{Axiom} \arabic*., align=left, leftmargin=\parindent]
		\setcounter{enumi}{-1}
		\item\label{ax::0} \textit{Set existence.}
		\begin{align*}
			\exists x(x=x).
		\end{align*}
		\item\label{ax::1} \textit{Extensionality.}
		\begin{align*}
		\forall x\forall y(\forall z(z\in x\leftrightarrow z\in y)\to x=y).
		\end{align*}
		\item\label{ax::2} \textit{Foundation.}
		\begin{align*}
		\forall x[\exists y(y\in x)\to\exists y(y\in x\land\neg\exists z(z\in x\land z\in y))].
		\end{align*}
		\item\label{ax::3} \textit{Comprehension scheme.} For each formula $\phi$ with free variables among $x,z,w_1,\ldots,w_n$,
		\begin{align*}
		\forall z\forall w_1,\ldots,w_n\exists y\forall x(x\in y\leftrightarrow x\in z\land\phi).
		\end{align*}
		\item\label{ax::4} \textit{Pairing.}
		\begin{align*}
		\forall x\forall y\exists z(x\in z\land y\in z).
		\end{align*}
		\item\label{ax::5} \textit{Union.}
		\begin{align*}
		\forall\mathscr{F}\exists A\forall Y\forall x(x\in Y\land Y\in\mathscr{F}\to x\in A).
		\end{align*}
		\item\label{ax::6} \textit{Replacement Scheme.} For each formula $\phi$ with free variables among $x, y, A, w_l,\ldots,w_n$,
		\begin{align*}
		\forall A\forall w_1,\ldots,w_n[\forall x\in A\exists!y\phi\to\exists Y\forall x\in A\exists y\in Y\phi].
		\end{align*}
	\end{enumerate}
On the basis of Axioms \href{ax::0}{0}, \href{ax::1}{1}, \href{ax::3}{3}, \href{ax::4}{4}, \href{ax::5}{5} and \href{ax::6}{6}, one may define $\subset$ (subset), $\emptyset$ (empty set), $S$ (ordinal successor; $S(x) = x \cup \{x\}$), and the notion of wellordering. The following axioms are then defined.
\begin{enumerate}[resume*]
	\item \textit{Infinity.}
	\begin{align*}
	\exists x(\emptyset\in x\land\forall y\in x(S(y)\in x)).
	\end{align*}
	\item \textit{Power set.}
	\begin{align*}
	\forall x\exists y\forall z(z\subset x\to z\in y).
	\end{align*}
	\item \textit{Choice.}
	\begin{align*}
	\forall A\exists R(R\text{ well orders }A).
	\end{align*}
\end{enumerate}
\end{fact}

\begin{corollary}[van Benthem~\cite{vanBenthem77}]
	$E$ is not provably arithmetical in $ZF$.
\end{corollary}

\begin{proof}
	$ZF + AC \vdash E(\phi^m, \phi^o)$ and $ZF \vdash E(\phi^m, \phi^o) + AC^{u0}$. The latter implies, by Jech's result, that ${\sim}ZF \vdash E(\phi^m, \phi^o)$. But then $E$ cannot be provably arithmetical in $ZF$, since $ZF+AC$ is conservative over $ZF$ with respect to arithmetical statements. (If $\phi$ is arithmetical,	i.e., all quantifiers in $\phi$ are relativized to $\omega$, and $ZF + AC \vdash \phi$, then, since $ZF \vdash (ZF)^L$ and $ZF \vdash (AC)^L$, $ZF \vdash \phi^L$, where $L$ defines the constructible universe. Now $\omega$ is absolute and, therefore, $ZF \vdash \phi$.)
\end{proof}

\begin{proposition}[Segerberg~\cite{Segerberg71}]
	Suppose that $L$ is a classical system. Let $\mathscr{C}$ be any class o frames. If every modal axiom of $L$ is valid in $\mathscr{C}$, then $L$ is consistent with respect to $\mathscr{C}$.
\end{proposition}

\begin{proof}
	The proof goes by induction on the length of derivations in $L$. Every nonmodal axiom is easily seen to be valid in $\mathscr{C}$. The modal axioms are valid in $\mathscr{C}$ by hypothesis.
	
	Suppose $A$ and $A\to B$ are valid in $\mathscr{C}$. Let $M$ be any model on any frame in $\mathscr{C}$. Take any $w$ in $M$. Then $M, w \models A$ and $M, w \models A\to B$. So, by truth definition, $M, w \models B$. Hence $\mathsf{MP}$ preserves validity in $\mathscr{C}$.
	
	Suppose finally that $A\leftrightarrow B$ is valid in $\mathscr{C}$. Let $M$ be any model on any frame in $\mathscr{C}$. Since $A\leftrightarrow B$ is true in $M$, $\lVert A\rVert^M = \lVert B\rVert^M$. Then $A\leftrightarrow B$ must hold at every point in $M$. Hence $\mathsf{RE}$ preserves validity in $\mathscr{C}$.
\end{proof}

\subsection{Mathematical formulas}

Below are some examples of mathematical formulas.

The so-called Dirac delta is a measure $\delta(x):\mathcal{B}(\mathbb{R})\longrightarrow\overline{\mathbb{R}}_+$ defined as follows:
\begin{equation}\label{eq::1}
\delta(A)=\begin{cases}1,&\text{if }0\in A,\\
0,&\text{if }0\notin A.
\end{cases}
\end{equation}
If we switch to informal definition, then the formula (\ref{eq::1}) is replaced by:
\begin{equation}\label{eq::2}
\delta(x)=\begin{cases}+\infty,&\text{if }x=0,\\
0,&\text{if }x\neq0.
\end{cases}
\end{equation}

Here is a simple sequent-based proof of the formula $((A\to C)\lor(B\to C))\to((A\land B)\to C)$:

\begin{prooftree}
	\AxiomC{}
	\RightLabel{(\textsf{Ax})}
	\UnaryInfC{$A,B,C\vdash C$}
	\RightLabel{(\textsf{MP})}
	\UnaryInfC{$A,B,A\to C\vdash C$}
	\AxiomC{}
	\RightLabel{(\textsf{Ax})}
	\UnaryInfC{$A,B,C\vdash C$}
	\RightLabel{(\textsf{MP})}
	\UnaryInfC{$A,B,B\to C\vdash C$}
	\RightLabel{($\lor\vdash$)}
	\BinaryInfC{$A,B,(A\to C)\lor(B\to C)\vdash C$}
	\RightLabel{($\land\vdash$)}
	\UnaryInfC{$A\land B,(A\to C)\lor(B\to C)\vdash C$}
	\RightLabel{($\vdash\to$)}
	\UnaryInfC{$(A\to C)\lor(B\to C)\vdash A\land B\to C$}
	\RightLabel{($\vdash\to$)}
	\UnaryInfC{$\vdash((A\to C)\lor(B\to C))\to(A\land B\to C)$}
\end{prooftree}

And here is a formula that estimates the number of elements of a structure yielded by a genering stream reasoning algorithm for DatalogMTL (see~\cite{WalegaEtAl19}):

\begin{equation}
\left(4\cdot\left(\frac{w+2\cdot step}{\mathsf{gcd}(\mathcal{T}_I\cup\mathsf{N}\cup\{step\})}+1 \right)^2\right)\cdot\mathsf{P}\cdot|\mathcal{O}_I|^\mathsf{A}
\end{equation}

\section{Sectioning: This is a section header}

Here come the contents of the section.

\subsection{This is a subsection header}

Here come the contents of the subsection.

\subsubsection{This is a subsubsection header}

Here come the contents of the subsubsection.

\paragraph{This is a paragraph header}

Here come the contents of the paragraph.

\subparagraph{This is a subparagraph header}

Here come the contents of the subparagraph.


\section{Bibliography management}

The authors are requested to use \textsc{Bib}\negthinspace\TeX{} to process their bibliographies. It involves creating a separate \texttt{.bib} file with bibliography entries and putting it in the same folder as the main \texttt{.tex} source file.

In order for \LaTeX{} to generate a bibliography which will be formatted in accordance with \texttt{BSLbibstyle}, one needs to execute the following sequence of commands:
\begin{itemize}
	\item[] \texttt{\textbackslash bibliographystyle\{BSLbibstyle\}}
	\item[] \texttt{\textbackslash bibliography\{\#1\}}
\end{itemize}
in the place where the bibliography is to be displayed (\texttt{\#1} is the name (without the file type extension) of the \texttt{.bib} file with bibliography entries). The authors can use the attached bibliography template (named \texttt{biblio.bib}) to create their own bibliography file. More information about bibliography management with \textsc{Bib}\negthinspace\TeX{} see~\cite{BibTeX}.

There are three rules the authors are asked to abide by when preparing their \texttt{.bib} files:

\begin{description}
	\item[Rule 1:] Always use journal names and names of proceedings series in their \textbf{full form}. For instance: \texttt{Journal of Logic and Computation} rather than (\texttt{J. Logic Comput.}) or \texttt{Lecture Notes in Computer Science} rather than \texttt{LNCS}.
	\item[Rule 2:] Whenever for a given publication occurring in the bibliography there exists a DOI number, include it in the bibliography entry in the \texttt{.bib} file. Use, however, \textbf{plain DOI numbers} rather than\linebreak full links, so for example \texttt{10.2307/2267577} rather than\linebreak \texttt{http://dx.doi.org/10.2307/2267577}.
	\item[Rule 3:] When providing page numbers of a given bibliography entry use an ndash (i.e., \texttt{{-}{-}}) rather than a hyphen (i.e., \texttt{-}) to separate the first page and the last page numbers. For example: \texttt{pages = \{153{-}{-}169\}} rather than \texttt{pages = \{153-169\}}.
\end{description}

\Acknowledgements{Acknowledgements such as funding information or thanks for reviewers' remarks can be put here.}

\bibliographystyle{BSLbibstyle}
\bibliography{biblio}


\end{document}