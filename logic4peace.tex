\documentclass{article}

\usepackage[a4paper,top=2.5cm,bottom=2.5cm,left=2.5cm,right=2.5cm]{geometry}

\begin{document}

\title{Logics of skew categorical structures}

\author{Tarmo Uustalu \and Niccol\`o Veltri \and Cheng-Syuan Wan}

\date{~}

\maketitle

\thispagestyle{empty}

There is an important correspondence, first noticed and highlighted by
Joachim Lambek \cite{Lam}, between logics on the one hand and structures on
categories on the other hand, where the latter establish notions of
proof and proof identity for the former. Intuitionistic logic is,
famously, the logic corresponding to Cartesian monoidal closed
categories, while multiplicative intuitionistic linear logic, for
example, is the logic of symmetric monoidal closed categories and
Lambek's syntactic calculus is the logic of nonsymmetric semimonoidal
biclosed (``residuated'') categories.

In 2012, Szachl\'anyi \cite{Szl} introduced skew monoidal categories
as a weakening of monoidal categories. In skew monoidal categories,
the laws of unitality and associativity are not natural isomorphisms
as in monoidal categories, but merely natural transformations in a
specific direction. Although considered in their own right only
recently, skew monoidal categories arise naturally in many categorical
contexts and have therefore quickly received quite some traction since
their inception. Since Szachl\'anyi's seminal work, other types of
skew categorical structure have been studied: skew closed and monoidal
closed categories by Street \cite{Str}, braided skew monoidal
categories by Bourke and Lack \cite{BL}. These works have not,
however, addressed the logical aspects of these types of categorical
structure.

In a recent series of papers
\cite{UVZ:lambek,UVZ:skpcl,UVZ:pn,Vel:ss,UVW}, we have investigated
several types of skew structure---skew monoidal, partially normal skew
monoidal, skew prounital closed, symmetric skew monoidal and skew
monoidal closed categories---from the perspective of structural proof
theory, specifically sequent calculi and normal forms of sequent
derivations, so-called focused derivations. We have shown that these
structures define meaningful logics of resources where the consumption
and production of resources is more controlled than in linear
logics. We have exploited proof-theoretic methods to formulate and
prove coherence theorems for these structures.

In the talk, we will explain these points on the examples of skew
monoidal and skew monoidal closed categories.

\small

\begin{thebibliography}{99}

\bibitem{Lam} J.~Lambek. Deductive systems and categories I: Syntactic
  calculus and residuated categories. \emph{Math. Syst. Theor.} 2(4),
  pp.~287--318, 1968.

\bibitem{Szl} K.~Szlach\'anyi. Skew-monoidal categories and
  bialgebroids. \emph{Adv. Math.} 231(3--4), pp.~1694--1730, 2012.

\bibitem{Str} R.~Street. Skew-closed categories. \emph{J. Pure
Appl. Algebra} 217(6), pp.~973--988, 2013.

\bibitem{BL} J.~Bourke, S.~Lack. Braided skew monoidal categories.
\emph{Theor. Appl. Categ.}, 35(2), pp.~19--63, 2020.

\bibitem{UVZ:lambek} T.~Uustalu, N.~Veltri, N.~Zeilberger. The sequent
  calculus of skew monoidal categories. In C.~Casadio, P.~J.~Scott,
  eds., \emph{Joachim Lambek: The Interplay of Mathematics, Logic, and
  Linguistics}, v.~20 of \emph{Outstanding Contributions to Logic},
  pp.~377--406. Springer, 2021.

\bibitem{UVZ:skpcl} T.~Uustalu, N.~Veltri, N.~Zeilberger. Deductive
  systems and coherence for skew prounital closed categories. \linebreak In
  C.~Sacerdoti Coen, A.~Tiu, eds., \emph{Proc. of 15th Int. Wksh. on
    Logical Frameworks and Metalanguages: Theory and Practice, LFMTP
    2020}, v.~332 of \emph{Electron. Proc. in Theor. Comput. Sci.},
  pp. 35--53. Open Publishing Assoc., 2021.

\bibitem{UVZ:pn} T.~Uustalu, N.~Veltri, N.~Zeilberger. Proof theory of
  partially normal skew monoidal categories. In D. I.~Spivak,
  J.~Vicary, eds., \emph{Proc. of 3rd Ann. Int. Applied Category
    Theory Conf., ACT~2020}, v.~333 of \emph{Electron. Proc. in
    Theor. Comput. Sci.}, pp.~230--246. Open Publishing Assoc., 2021.

\bibitem{Vel:ss} N.~Veltri. Coherence via focusing for symmetric skew
  monoidal categories. In A.~Silva, R.~Wassermann, R.~de Queiroz,
  eds., \emph{Proc. of 27th Wksh. on Logic, Language, Information and
    Computation, WoLLIC~2021}, v.~13038 of \emph{Lect. Notes in
    Comput. Sci.}, pp.~184--200. Springer, 2021.

\bibitem{UVW} T.~Uustalu, N.~Veltri, C.-S.~Wan. Proof theory of skew
  non-commutative MILL. In A.~Indrzejczak, \linebreak M.~Zawidzki, eds.,
  \emph{Proc. of 10th Int. Conf. on Non-classical Logics: Theory and
    Applications, NCL~2022}, \emph{Electron. Proc. in
    Theor. Comput. Sci.}, Open Publishing Assoc., to appear.

\end{thebibliography}

\end{document}
