\documentclass[submission,copyright,creativecommons]{eptcs}
\providecommand{\event}{LSFA 2023} % Name of the event you are submitting to
%\usepackage{breakurl}             % Not needed if you use pdflatex only.
\usepackage{underscore}           % Only needed if you use pdflatex.
\usepackage{amsmath}
\usepackage{amsthm}
\usepackage{amsfonts}
\usepackage{amssymb}
\usepackage{enumerate}
\usepackage{hyperref}
\usepackage{quiver}
\usepackage[all,cmtip]{xy}
\usepackage{proof}

%% \theorembodyfont{}
\newtheorem{theorem}{Theorem}[section]
\newtheorem{corollary}[theorem]{Corollary}
\newtheorem{lemma}[theorem]{Lemma}
\newtheorem{remark}[theorem]{Remark}
\newtheorem{proposition}[theorem]{Proposition}
\theoremstyle{definition}
\newtheorem{defn}{Definition}[section]
\newtheorem{example}{Example}[section]
%\newtheorem{defn}[definition]{Definition}
%\newtheorem{example}[definition]{Definition}
%\newtheorem*{proof}{Proof : }
\newtheorem{fact}[theorem]{Fact}
\makeatletter
\newsavebox{\@brx}
\newcommand{\llangle}[1][]{\savebox{\@brx}{\(\m@th{#1\langle}\)}%
  \mathopen{\copy\@brx\kern-0.5\wd\@brx\usebox{\@brx}}}
\newcommand{\rrangle}[1][]{\savebox{\@brx}{\(\m@th{#1\rangle}\)}%
  \mathclose{\copy\@brx\kern-0.5\wd\@brx\usebox{\@brx}}}
\makeatother
\newcommand{\ldbc}{[\![}
\newcommand{\rdbc}{]\!]}
\newcommand{\tbar}{[\vec{x}/\vec{t}]}
\newcommand{\ltbar}{[\vec{x}, x/\vec{t}, x]}
\newcommand{\tl}{\otimes \mathsf{L}}
\newcommand{\tr}{\otimes \mathsf{R}}
\newcommand{\lright}{{\multimap}\mathsf{R}}
\newcommand{\lleft}{{\multimap}\mathsf{L}}
\newcommand{\pass}{\mathsf{pass}}
\newcommand{\unitl}{\mathsf{IL}}
\newcommand{\unitr}{\mathsf{IR}}
\newcommand{\andlone}{\land \mathsf{L_{1}}}
\newcommand{\andltwo}{\land \mathsf{L_{2}}}
\newcommand{\andli}{\land \mathsf{L}_{i}}
\newcommand{\andr}{\land \mathsf{R}}
\newcommand{\orl}{\lor \mathsf{L}}
\newcommand{\orrone}{\lor \mathsf{R_{1}}}
\newcommand{\orrtwo}{\lor \mathsf{R_{2}}}
\newcommand{\orri}{\lor \mathsf{R}_{i}}
\newcommand{\ax}{\mathsf{ax}}
\newcommand{\id}{\mathsf{id}}
\newcommand{\ot}{\otimes}
\newcommand{\lolli}{\multimap}
\newcommand{\illol}{\rotatebox[origin=c]{180}{$\multimap$}}
\newcommand{\I}{\mathsf{I}}
\newcommand{\msfL}{\mathsf{L}}
\newcommand{\defeq}{=_{\mathsf{df}}}
\newcommand{\comp}{\mathsf{comp}}
\newcommand{\C}{\mathsf{C}}
\newcommand{\RI}{\mathsf{RI}}
\newcommand{\LI}{\mathsf{LI}}
\newcommand{\Pass}{\mathsf{P}}
\newcommand{\F}{\mathsf{F}}
\newcommand{\sw}{\mathsf{sw}}
\newcommand{\tP}{\mathbb{P}}
\newcommand{\tL}{\mathbb{L}}
\newcommand{\tR}{\mathbb{R}}
\newcommand{\tE}{\mathbb{E}}
\newcommand{\xvdash}{\vdash^{x}}
\newcommand{\yvdash}{\vdash^{y}}
\newcommand{\ex}{\mathsf{ex}}

\newcommand{\otd}{\ot^D}
\newcommand{\lollid}{\;\textsuperscript{$D$}\!\!\lolli}
\newcommand{\spl}{\raisebox{-1.5pt}[0.5\height]{\hspace{1pt}\vdots\hspace{1pt}}}

\newcommand{\highlight}[1]{\textcolor{blue}{#1}}

\newcommand{\proofbox}[1]{\begin{tabular}{l} #1 \end{tabular}}

\newcommand{\MILL}{$\mathtt{MILL}$}
\newcommand{\NMILL}{$\mathtt{NMILL}$}
\newcommand{\SkNMILL}{$\mathtt{SkNMILL}$}
\newcommand{\SkNMILLA}{$\mathtt{SkNMILLA}$}
\newcommand{\FSkMCC}{\mathsf{FDSkM}}

\newcommand\cheng[1]{\mbox{}
{\marginpar{\color{darkgreen}CSW}}
{\sf\noindent\color{darkgreen}#1}}%
\newcommand\niccolo[1]{\mbox{}
{\marginpar{\color{red}NV}}
{\sf\noindent\color{red}#1}}%



\title{Towards Skew Non-Commutative \MILL \text{ }with Additives}
% \author{Tarmo Uustalu
% \institute{Reykjavik University, Iceland}
% \institute{Tallinn University of Technology, Estonia}
% \email{tarmo@ru.is}
% \and
\author{
Niccol{\`o} Veltri \qquad\qquad Cheng-Syuan Wan
\institute{Tallinn University of Technology, Estonia}
\email{\quad niccolo@cs.ioc.ee \quad\qquad cswan@cs.ioc.ee}
}
\def\titlerunning{Towards Skew Non-Commutative \MILL \text{ }with Additives}
\def\authorrunning{N. Veltri \& C.-S. Wan}
\begin{document}
\maketitle
\begin{abstract}
This work concerns the proof theory of (left) skew monoidal categories and their variants (e.g. closed monoidal, symmetric monoidal), continuing the line of work initiated in recent years by Uustalu et al.
Skew monoidal categories are a weak version of Mac Lane's monoidal categories, where the structural laws are not required to be invertible, they are merely natural transformations with a specific orientation. 
Numerous variants of skew monoidal categories are described as sequent calculi, which can be identified as restricted substructural fragments of intuitionistic linear logic. These calculi enjoy cut elimination and admit a focusing strategy, sharing resemblance with Andreoli's normalization technique for linear logic. The focusing procedure is useful for solving the coherence problem of the considered categories with skew structure.

Here we investigate possible extensions of the sequent calculi of Uustalu et al. with additive connectives. 
As a first step, we extend the sequent calculus with additive conjunction and disjunction, corresponding to studying the proof theory of skew monoidal categories with binary Cartesian products and coproducts. 
We introduce a new focused sequent calculus of derivations in normal form, which employs tag annotations to reduce non-deterministic choices in bottom-up proof search.
Apart from statements and proofs on pen and paper, we also want to formalize the focused sequent calculus and verify its correctness in the Agda proof assistant.
We believe this to be beneficial for the development of modular normalization techniques for substructural logics arising as an extension of our sequent calculus, e.g. full Lambek calculus or intuitionistic linear logic.
\end{abstract}

\section{Introduction}

Substructural logics are logical systems in which the usage of one or more structural rule is disallowed or restricted. A well-known example is given by the syntactic calculus of Lambek \cite{lambek:mathematics:58}, in which all the structural rules of exchange, weakening and contraction are disallowed. Variants of the Lambek calculus allow exchange or a cyclic form of exchange, while others disallow even associativity \cite{moot:logic:12}. In Girard's linear logic, which have been studied both in the presence and absence of an exchange rule \cite{girard:linear:87,abrusci:noncommutative:1990}, selective versions of weakening and contraction can be recorvered via the use of modalities. Application of substructural logics are abundant in a variety of different fields, from computational investigations of natural languages to the design and development of resource-aware programming languages.

In recent years, in collaboration with Tarmo Uustalu and Noam Zeilberger, we initiated a program intended to study a family of \emph{semi-substructural} logics, inspired by developments in category theory by Szlach{\'a}nyi, Street, Bourke, Lack and others \cite{szlachanyi:skew-monoidal:2012,lack:skew:2012,street:skew-closed:2013,lack:triangulations:2014,buckley:catalan:2015,bourke:skew:2017,bourke:skew:2018,bourke:lack:braided:2020}. Korn{\'e}l Szlach\'anyi introduced \emph{skew monoidal categories} as a weakening of MacLane's monoidal categories in which the structural morphisms of associativity and unitality (often also called associator and unitors) are not required to be invertible, they are merely natural transformation in a particular direction. As such they can be considered \emph{semi-associative} and \emph{semi-unital} variants of monoidal categories. Bourke and Lack also introduced notions of braiding and symmetry for skew monoidal categories which involve three objects instead of two \cite{bourke:lack:braided:2020}. Skew monoidal categories arise naturally in semantics of programming languages \cite{altenkirch:monads:2014}, and semi-associativity has strong connections with combinatorial structures such as the Tamari lattice and Stasheff associahedra \cite{zeilberger:semiassociative:19,moortgat:tamari:20}

Semi-substructural logics correspond to the internal languages of skew monoidal categories and their extensions, which sit in-between (certain fragments of) non-associative and associative intuitionistic linear logic. Semi-associativity and semi-unitality can be hard-coded in the sequent calculus in two steps. First consider sequents of the form $S \mid \Gamma \vdash A$, where the antecedent is split into an optional formula $S$, which we call a stoup, and an ordered list of formulae $\Gamma$. The succedent consists of a single formula $A$. Then restrict the application of introduction rules to allow only one direction of associativity and unitality. For example, left-introduction rules are allowed to act only on the formula in stoup position, not on formulae in $\Gamma$.

In our investigations we have explored deductive systems for $(i)$ skew semigroup \cite{zeilberger:semiassociative:19}, $(ii)$ skew monoidal \cite{uustalu:sequent:2021}, $(iii)$ skew (prounital) closed \cite{uustalu:deductive:nodate} and $(iv)$ skew monoidal closed categories \cite{UVW:protsn}, corresponding to skew variants of the fragments of non-commutative intuitionistic linear logic consisting of connectives $(i)$ $\otimes$, $(ii)$ $(\I,\otimes)$, $(iii)$ $\lolli$ and $(iv)$ $(\I,\otimes,\lolli)$. We have also studied partial normality conditions, when one or more among associator and unitors is allowed to have an inverse \cite{uustalu:proof:nodate}, and extensions with exchange {\`a} la Bourke and Lack \cite{veltri:coherence:2021}.

When studying meta-theoretic properties of these semi-structural deductive systems, we have been mostly interested in categorical and proof-theoretic semantics. In the latter, we have particularly investigated normalization strategies inspired by Andreoli's focused sequent calculus for classical linear logic \cite{andreoli:logic:1992} and employed the resulting normal forms to solve the \emph{coherence problem} for the corresponding categories with skew structure. For the latter categories, the word problem is more nuanced than in the normal non-skew case studied by MacLane \cite{maclane1963natural}. Our study also revealed that the focused sequent calculi of semi-substructural logics can serve as cornerstones for a compositional and modular understanding of normalization techniques for other richer substructural logics.

In this work we begin the investigation of  semi-substructural logics with \emph{additive connectives}. We start in Section \ref{sec:sequent-calculus} by considering a fragment of non-commutative linear logic consisting of skew multiplicative unit $\I$ and conjunction $\ot$, and additive conjunction $\land$ and disjunction $\lor$. We describe a cut-free sequent calculus and a congruence relation identifying derivations up-to $\eta$-equivalence and permutative conversions. In Section \ref{sec:categorical}, we discuss categorical semantics in terms of skew monoidal categories with binary products and coproducts. In Section \ref{sec:focusing}, we subsequently introduce a sequent calculus of proofs in normal form, again inspired by the ideas of Andreoli, which describes a sound and complete root-first proof search strategy for the original sequent calculus. Completeness is achieved by marking sequents with lists of \emph{tags}, a mechanism introduced by Uustalu et al. \cite{UVW:protsn} and inspired by Scherer and R{\'e}my's saturation technique \cite{scherer:simple:2015}, which help to  eliminate completely all unnecessary non-determinism in proof search and faithfully capture normal forms wrt. the congruence relation on derivations in the original sequent calculus. %In a sense, this is similar to what proof nets are used for.

To showcase the modularity of our normalization strategy, in Section \ref{sec:extensions} we discuss extensions of the logic with other connectives, such as additive units and implication, and other structural rules, such has partial normality conditions and exchange {\`a} la Bourke and Lack.

The sequent calculi of Sections \ref{sec:sequent-calculus} and \ref{sec:focusing}, as well as the effective normalization procedure, have been formalized in the Agda proof assistant. The code is freely available at
%\begin{center}
  \url{https://github.com/cswphilo/FormalizationSKewMonCloCat}.
%\end{center}
We briefly comment on some of the design choices and reflect on the formalization experience in Section \ref{sec:formalization}. 

\section{Sequent Calculus}\label{sec:sequent-calculus}
We introduce a sequent calculus for a skew variant of non-commutative multiplicative intuitionistic linear logic (\NMILL) with additive conjunction and disjunction.

Formulae are inductively generated by the grammar $A,B ::= X \ | \ \I \ | \ A \ot B \ | \ A \land B \ | \ A \lor B$, where $X$ comes from a set $\mathsf{At}$ of atomic formulae. 
We use $\I , \ot , \land$, and $\lor$ to denote multiplicative verum, multiplicative conjunction, additive conjunction and additive disjunction, respectively.

A sequent is a triple of the form $S \mid \Gamma \vdash A$.
The antecedent is split in two parts: an optional formula $S$, called \emph{stoup} \cite{girard:constructive:91}, and an ordered list of formulae $\Gamma$, called \emph{context}.
The succedent $A$ is a single formula.
The peculiar design of sequents, involving the presence of the stoup in the antecedent, comes from previous work on deductive systems with skew structure by Uustalu, Veltri, Wan and Zeilberger \cite{uustalu:sequent:2021,uustalu:proof:nodate,uustalu:deductive:nodate,veltri:coherence:2021,UVW:protsn}.
The metavariable $S$ always denotes a stoup, i.e., $S$ can be a single formula or empty, in which case we write $S = {-}$. $X,Y,Z$ are always names of atomic formulae.

Derivations of a sequent $S \mid \Gamma \vdash A$ are inductively generated by the following rules:

\begin{equation}\label{eq:seqcalc}
  \def\arraystretch{2.5}
  \begin{array}{c}
    \infer[\ax]{A \mid \quad \vdash A}{}
    \qquad
    \infer[\pass]{{-} \mid A , \Gamma \vdash C}{A \mid \Gamma \vdash C}
    \qquad
    \infer[\unitl]{\I \mid \Gamma \vdash C}{{-} \mid \Gamma \vdash C}
    \qquad
    \infer[\unitr]{{-} \mid \quad \vdash \I}{}
    \\
    \qquad
    \infer[\tl]{A \ot B \mid \Gamma \vdash C}{A \mid B , \Gamma \vdash C}
    \qquad
    \infer[\tr]{S \mid \Gamma , \Delta \vdash A \ot B}{
      S \mid \Gamma \vdash A
      &
      {-} \mid \Delta \vdash B
    }
    \\
    \infer[\andlone]{A \land B \mid \Gamma \vdash C}{A \mid \Gamma \vdash C}
    \qquad
    \infer[\andltwo]{A \land B \mid \Gamma \vdash C}{B \mid \Gamma \vdash C}
    \qquad
    \infer[\andr]{S \mid \Gamma \vdash A \land B}{
      S \mid \Gamma \vdash A
      &
      S \mid \Gamma \vdash B
    }
    \\
    \infer[\orl]{A \lor B \mid \Gamma \vdash C}{
      A \mid \Gamma \vdash C
      &
      B \mid \Gamma \vdash C
    }
    \qquad
    \infer[\orrone]{S \mid \Gamma \vdash A \lor B}{S \mid \Gamma \vdash A}
    \qquad
    \infer[\orrtwo]{S \mid \Gamma \vdash A \lor B}{S \mid \Gamma \vdash B}
  \end{array}
\end{equation}
% The inference rules in (\ref{eq:seqcalc}) are reminiscent of the ones in the sequent calculus for \NMILL\ \cite{abrusci:noncommutative:1990}, but there are some crucial differences.
The inference rules are similar to the ones in \NMILL \ \cite{abrusci:noncommutative:1990}, but with some essential differences. 
\begin{enumerate}
\item The left logical rules $\unitl$, $\tl$, $\lleft$, $\andli$, and $\orl$, when read bottom-up, can only be applied on the formula in the stoup position. 
That is, it is generally not possible to remove a unit $\I$, or decompose a tensor $A \ot B$ or a disjunction $A \lor B$, when these formulae are located in the context.
\item The right tensor rule $\tr$, when read bottom-up, splits the antecedent of the conclusion but the formula in the stoup, whenever this is present, always moves to the first (left) premise.
The stoup formula of the conclusion is prohibited to move to the second premise even if $\Gamma$ is empty. 
\item The presence of the stoup implies a distinction between antecedents of the form $A \mid \Gamma$ and ${-} \mid A, \Gamma$. The structural rule $\pass$ (for `passivation'), when read bottom-up, allows the moving of the leftmost formula in the context to the stoup position whenever the stoup is initially empty.
% \item The logical connectives of \NMILL\ typically include two ordered implications $\lolli$ and $\illol$, which are two variants of linear implication arising from the removal of the exchange rule from intuitionistic linear logic. In \SkNMILL\ only one of the ordered implications (the left implication $\lolli$) is present. It is currently not clear to us whether the inclusion of the second implication to our logic is a meaningful addition and whether it corresponds to some particular categorical notion.
\end{enumerate}
These restrictions allow the derivability of sequents $(A \ot B) \ot C \mid ~\vdash A \ot (B\ot C)$ (semi-associativity), $\I \ot A \mid ~ \vdash A$ and $A \mid ~ \vdash A \ot \I$ (semi-unitality), while forbidding the derivability of their invereses, where the formulae in the stoup and in the succedent have been swapped. This is in line with the intended categorical semantics, see Section \ref{sec:categorical}.
% The restrictions in 1--4 are essential for precisely capturing all the features of skew monoidal closed categories and nothing more, as we discuss in Section \ref{sec:catsem}.
% Notice also that, similarly to the case of \NMILL, all structural rules of exchange, contraction, and weakening are absent. We give names to derivations and we write $f : S \mid \Gamma \vdash A$ when $f$ is a particular derivation of the sequent $S \mid \Gamma \vdash A$.
Notice that, similarly to the case of \NMILL, all structural rules of exchange, contraction, and weakening are absent. We give names to derivations and we write $f : S \mid \Gamma \vdash A$ when $f$ is a particular derivation of the sequent $S \mid \Gamma \vdash A$.

\begin{theorem}
  The sequent calculus enjoys cut admissibility: the following two cut rules are admissible
    \begin{displaymath}
      \infer[\mathsf{scut}]{S \mid \Gamma , \Delta \vdash C}{
        S \mid \Gamma \vdash A
        &
        A \mid \Delta \vdash C
      }
      \qquad
      \infer[\mathsf{ccut}]{S \mid \Delta_0 , \Gamma , \Delta_1 \vdash C}{
        {-} \mid \Gamma \vdash A
        &
        S \mid \Delta_0 , A , \Delta_1 \vdash C
      }
    \end{displaymath}
\end{theorem}

\niccolo{Perhaps this discussion about commutativity of conjunction/disjunction could be removed}

  Although the sequent calculus is non-commutative, the additive connectives are still commutative due to the nature of cartesian and cocartesian structure.
  \begin{displaymath}
    % \hspace{-1cm}
    \begin{array}{cc}
      (\text{commutativity of } \land) & (\text{commutativity of } \land)
      \\[10pt]
      \proofbox{
        \infer[\andr]{A \land B \mid \quad \vdash B \land A}{
          \infer[\andltwo]{A \land B \mid \quad \vdash B}{
            \infer[\ax]{B \mid \quad \vdash B}{}
          }
          &
          \infer[\andlone]{A \land B \mid \quad \vdash A}{
            \infer[\ax]{A \mid \quad \vdash A}{}
          }
        }
      }
      &
      \proofbox{
        \infer[\orl]{A \lor B \mid \quad \vdash B \lor A}{
          \infer[\orrtwo]{A \mid \quad \vdash B \lor A}{
            \infer[\ax]{A \mid \quad \vdash A}{}
          }
          &
          \infer[\orrone]{B \mid \quad \vdash B \lor A}{
            \infer[\ax]{B \mid \quad \vdash B}{}
          }
        }
      }
    \end{array}
  \end{displaymath}
  
While the left $\land$-rules only act on the formula in stoup position (as all the other left logical rules), other $\land$-rules $\andli^{\mathsf{C}}$ acting on formulae in context are admissible.
  \begin{displaymath}
    \begin{array}{c}
      \proofbox{
        \infer[\andlone^{\mathsf{C}}]{S \mid \Gamma , A \land B , \Delta \vdash C}{
          S \mid \Gamma , A ,\Delta \vdash C
        }
      }
      \quad
      \proofbox{
        \infer[\andltwo^{\mathsf{C}}]{S \mid \Gamma , A \land B , \Delta \vdash C}{
          S \mid \Gamma , B ,\Delta \vdash C
        }
      }
    \end{array}
  \end{displaymath}
However, this is not the case for the other left logical rules. For example, a general left $\lor$-rule $\orl^{\mathsf{C}}$ acting on a disjunction in context is inadmissible. should be forbidden since it would make some inadmissible sequents provable in the sequent calculus. 
For example, the sequent $X \land Y \mid Y \lor X \vdash (X \ot Y) \lor (Y \ot X)$ is not admissible (this can be proved using the normalization procedure of Section \ref{sec:focusing}) but a proof can be found using $\orl^{\mathsf{C}}$:
\begin{displaymath}
  \footnotesize
  \begin{array}{c}
%    \text{an instance of the failed attempts}
%    &
%    \text{a proof with } \orl^{\mathsf{C}}
%    \\[10pt]
%%    \infer[\andlone]{X \land Y \mid Y \lor X \vdash (X \ot Y) \lor (Y \ot X)}{
%%    \infer[\orrone]{X \mid Y \lor X \vdash (X \ot Y) \lor (Y \ot X)}{
%%      \infer[\tr]{X \mid Y \lor X \vdash X \ot Y}{
%%        \infer[\ax]{X \mid \quad \vdash X}{}
%%        &
%%        \infer[\pass]{- \mid Y \lor X \vdash Y}{
%%          \infer[\orl]{Y \lor X \mid \quad \vdash Y}{
%%            \infer[\ax]{Y \mid \quad \vdash Y}{}
%%            &
%%            \deduce{X \mid \quad \vdash Y}{??}
%%          }
%%        }
%%      }
%%    }
%%  }
%%  &
  \infer[\orl^{\mathsf{C}}]{X \land Y \mid Y \lor X \vdash (X \ot Y) \lor (Y \ot X)}{
    \infer[\orrone]{X \land Y \mid Y \vdash (X \ot Y) \lor (Y \ot X)}{
      \infer[\andlone]{X \land Y \mid Y \vdash X \ot Y}{
        \infer[\tr]{X \mid Y \vdash X \ot Y}{
          \infer[\ax]{X \mid \quad \vdash X}{}
          &
          \infer[\pass]{- \mid Y \vdash Y}{
            \infer[\ax]{Y \mid \quad \vdash Y}{}
          }
        }
      }
    }
    &
    \infer[\orrtwo]{X \land Y \mid X \vdash (X \ot Y) \lor (Y \ot X)}{
      \infer[\andltwo]{X \land Y \mid X \vdash Y \ot X}{
        \infer[\tr]{Y \mid X \vdash Y \ot X}{
          \infer[\ax]{Y \mid \quad \vdash Y}{}
          &
          \infer[\pass]{- \mid X \vdash X}{
            \infer[\ax]{X \mid \quad \vdash X}{}
          }
        }
      }
    }
  }
  \end{array}
\end{displaymath}

We introduce a congruence relation $\circeq$ on the sets of cut-free derivations generated by the rules in (\ref{eq:seqcalc}). 
\begin{figure}[t]
  \begin{equation}
  \label{fig:circeq}
  \footnotesize
  \begin{array}{rlll}
    \ax_{\I} &\circeq \unitl \text{ } (\unitr)
    \\
    \ax_{A \ot B} &\circeq \tl \text{ } (\tr \text{ } (\ax_{A} , \pass \text{ } \ax_{B}))
    \\
    \ax_{A \land B} &\circeq \andr \ (\andlone \ \ax_A , \andltwo \ \ax_B)
    \\
    \ax_{A \lor B} &\circeq \orl \ (\orrone \ \ax_A , \orrtwo \ \ax_B)
    \\
    \tr \text{ } (\pass \text{ } f, g) &\circeq \pass \text{ } (\tr \text{ } (f, g)) &&f : A' \mid \Gamma \vdash A, g : {-} \mid \Delta \vdash B
    \\
    \tr \text{ } (\unitl \text{ } f, g) &\circeq \unitl \text{ } (\tr \text{ } (f , g)) &&f : {-} \mid \Gamma \vdash A , g : {-} \mid \Delta \vdash B
    \\
    \tr \text{ } (\tl \text{ } f, g) &\circeq \tl \text{ } (\tr \text{ } (f , g)) &&f : A' \mid B' , \Gamma \vdash A , g : {-} \mid \Delta \vdash B
    \\
    \tr \ (\andlone \ f , g) &\circeq \andlone \ (\tr \ (f , g)) &&f : A' \mid \Gamma \vdash A, g : - \mid \Delta \vdash B
    \\
    \tr \ (\andltwo \ f , g) &\circeq \andltwo \ (\tr \ (f , g)) &&f : B' \mid \Gamma \vdash A, g : - \mid \Delta \vdash B
    \\
    \tr \ (\orl \ (f_1 , f_2) , g) &\circeq \orl \ (\tr \ (f_1 , g) , \tr \ (f_2 , g)) &&f_1 : A' \mid \Gamma \vdash A , f_2 : B' \mid \Gamma \vdash A , g : - \mid \Delta \vdash B
    \\
    \andr \ (\pass \ f , \pass \ g) &\circeq \pass \ (\andr \ (f , g)) &&f : A' \mid \Gamma \vdash A , g : A' \mid \Gamma \vdash B
    \\
    \andr \ (\unitl \ f , \unitl \ g) &\circeq \unitl \ (\andr \ (f , g)) &&f : - \mid \Gamma \vdash A , g : - \mid \Gamma \vdash B
    \\
    \andr \ (\tl \ f , \tl \ g) &\circeq \tl \ (\andr \ (f , g)) &&f : A' \mid B' , \Gamma \vdash A , g : A' \mid B' , \Gamma \vdash B
    \\
    \andr \ (\andlone \ f , \andlone \ g) &\circeq \andlone \ (\andr \ (f , g)) &&f : A' \mid \Gamma \vdash A , g : A' \mid \Gamma \vdash B
    \\
    \andr \ (\andltwo \ f , \andltwo \ g) &\circeq \andltwo \ (\andr \ (f , g)) &&f : B' \mid \Gamma \vdash A , g : B' \mid \Gamma \vdash B
    \\
    \andr \ (\orl \ (f_1 , f_2) , \orl \ (g_1 , g_2)) &\circeq \orl \ (\andr \ (f_1 , g_1) , \andr \ (f_2 , g_2)) &&f_1 : A' \mid \Gamma \vdash A , f_2 : B' \mid \Gamma \vdash A , 
    \\
    & &&g_1 : A' \mid \Gamma \vdash B , g_2 : B' \mid \Gamma \vdash B
    \\
    \orrone \ (\pass \ f) &\circeq \pass \ (\orrone \ f) &&f : A' \mid \Gamma \vdash A
    \\
    \orrone \ (\unitl \ f) &\circeq \unitl \ (\orrone \ f) &&f : - \mid \Gamma \vdash A
    \\
    \orrone \ (\tl \ f) &\circeq \tl \ (\orrone \ f) &&f : A' \mid B' , \Gamma \vdash A
    \\
    \orrone \ (\andlone \ f) &\circeq \andlone \ (\orrone \ f) &&f : A' \mid \Gamma \vdash A
    \\
    \orrone \ (\andltwo \ f) &\circeq \andltwo \ (\orrone \ f) &&f : B' \mid \Gamma \vdash A
    \\
    \orrone \ (\orl \ (f , g)) &\circeq \orl \ (\orrone \ f , \orrone \ g) &&f : A' \mid \Gamma \vdash A , g : B' \mid \Gamma \vdash A
    \\
    \orrtwo \ (\pass \ f) &\circeq \pass \ (\orrtwo \ f) &&f : A' \mid \Gamma \vdash B
    \\
    \orrtwo \ (\unitl \ f) &\circeq \unitl \ (\orrtwo \ f) &&f : - \mid \Gamma \vdash B
    \\
    \orrtwo \ (\tl \ f) &\circeq \tl \ (\orrtwo \ f) &&f : A' \mid B' , \Gamma \vdash B
    \\
    \orrtwo \ (\andlone \ f) &\circeq \andlone \ (\orrtwo \ f) &&f : A' \mid \Gamma \vdash B
    \\
    \orrtwo \ (\andltwo \ f) &\circeq \andltwo \ (\orrtwo \ f) &&f : B' \mid \Gamma \vdash B
    \\
    \orrtwo \ (\orl \ (f , g)) &\circeq \orl \ (\orrtwo \ f , \orrtwo \ g) &&f : A' \mid \Gamma \vdash B , g : B' \mid \Gamma \vdash B
  \end{array}
  \end{equation}
  \caption{Equivalence of sequent calculus derivations.}
\end{figure}
The first four equations ($\eta$-conversions) characterize the $\ax$ rule for non-atomic formulae. The remaining equations are permutative conversions. The congruence $\circeq$ has been carefully chosen to serve as the proof-theoretic counterpart of the equational theory of skew monoidal categories with products and coproducts, which we introduce next.

\niccolo{In equations 10-16, the left- and right-hand sides of the equations should be swapped (so that equations, when read from left to right, becomes conversion rules)}

  % , introduced in Definition \ref{def:skewcat}. The subsystem of equations involving only $(\I,\ot)$ originated in \cite{uustalu:sequent:2021} while the subsystem involving only $\lolli$ is from \cite{uustalu:deductive:nodate}.

\section{Categorical Semantics}\label{sec:categorical}
  A \emph{skew monoidal category} \cite{szlachanyi:skew-monoidal:2012,lack:skew:2012,lack:triangulations:2014} is a category $\mathbb{C}$ with a unit object $\I$, a functor $\ot : \mathbb{C} \times \mathbb{C} \rightarrow \mathbb{C}$
and three natural transformations $\lambda$, $\rho$, $\alpha$ typed
%\begin{displaymath}
  $\lambda_A : \I \ot A \to A$, $\rho_A : A \to A \ot \I$ and $\alpha_{A,B,C} : (A \ot B) \ot C \to A \ot (B \ot C)$,
%\end{displaymath}
satisfying the following equations due to Mac Lane \cite{maclane1963natural}:
\begin{center}
  %(m1)
  % https://q.uiver.app/?q=WzAsMyxbMSwwLCJcXEkgXFxvdCBcXEkiXSxbMCwxLCJcXEkiXSxbMiwxLCJcXEkiXSxbMSwwLCJcXHJob197XFxJfSJdLFswLDIsIlxcbGFtYmRhX3tcXEl9Il0sWzEsMiwiIiwyLHsibGV2ZWwiOjIsInN0eWxlIjp7ImhlYWQiOnsibmFtZSI6Im5vbmUifX19XV0=
\begin{tikzcd}
	& {\I \ot \I} \\[-.2cm]
	\I && \I
	\arrow["{\rho_{\I}}", from=2-1, to=1-2]
	\arrow["{\lambda_{\I}}", from=1-2, to=2-3]
	\arrow[Rightarrow, no head, from=2-1, to=2-3]
\end{tikzcd}
\qquad
%(m2)
% https://q.uiver.app/?q=WzAsNCxbMCwwLCIoQSBcXG90IFxcSSkgXFxvdCBCIl0sWzEsMCwiQSBcXG90IChcXEkgXFxvdCBCKSJdLFsxLDEsIkEgXFxvdCBCIl0sWzAsMSwiQSBcXG90IEIiXSxbMywyLCIiLDAseyJsZXZlbCI6Miwic3R5bGUiOnsiaGVhZCI6eyJuYW1lIjoibm9uZSJ9fX1dLFszLDAsIlxccmhvX0EgXFxvdCBCIl0sWzEsMiwiQSBcXG90IFxcbGFtYmRhX3tCfSJdLFswLDEsIlxcYWxwaGFfe0EgLCBcXEkgLCBCfSJdXQ==
\begin{tikzcd}
	{(A \ot \I) \ot B} & {A \ot (\I \ot B)} \\[-.3cm]
	{A \ot B} & {A \ot B}
	\arrow[Rightarrow, no head, from=2-1, to=2-2]
	\arrow["{\rho_A \ot B}", from=2-1, to=1-1]
	\arrow["{A \ot \lambda_{B}}", from=1-2, to=2-2]
	\arrow["{\alpha_{A , \I , B}}", from=1-1, to=1-2]
\end{tikzcd}

%(m3)
% https://q.uiver.app/?q=WzAsMyxbMCwwLCIoXFxJIFxcb3QgQSApIFxcb3QgQiJdLFsyLDAsIlxcSSBcXG90IChBIFxcb3QgQikiXSxbMSwxLCJBIFxcb3QgQiJdLFswLDEsIlxcYWxwaGFfe1xcSSAsIEEgLEJ9Il0sWzEsMiwiXFxsYW1iZGFfe0EgXFxvdCBCfSJdLFswLDIsIlxcbGFtYmRhX3tBfSBcXG90IEIiLDJdXQ==
\begin{tikzcd}
	{(\I \ot A ) \ot B} && {\I \ot (A \ot B)} \\[-.3cm]
	& {A \ot B}
	\arrow["{\alpha_{\I , A ,B}}", from=1-1, to=1-3]
	\arrow["{\lambda_{A \ot B}}", from=1-3, to=2-2]
	\arrow["{\lambda_{A} \ot B}"', from=1-1, to=2-2]
\end{tikzcd}
\qquad
%(m4)
% https://q.uiver.app/?q=WzAsMyxbMCwwLCIoQSBcXG90IEIpIFxcb3QgXFxJIl0sWzIsMCwiQSBcXG90IChCIFxcb3QgXFxJKSJdLFsxLDEsIkEgXFxvdCBCIl0sWzAsMSwiXFxhbHBoYV97QSAsIEIsIFxcSX0iXSxbMiwxLCJBIFxcb3QgXFxyaG9fQiIsMl0sWzIsMCwiXFxyaG9fe0EgXFxvdCBCfSJdXQ==
\begin{tikzcd}
	{(A \ot B) \ot \I} && {A \ot (B \ot \I)} \\[-.3cm]
	& {A \ot B}
	\arrow["{\alpha_{A , B, \I}}", from=1-1, to=1-3]
	\arrow["{A \ot \rho_B}"', from=2-2, to=1-3]
	\arrow["{\rho_{A \ot B}}", from=2-2, to=1-1]
\end{tikzcd}

%(m5)
% https://q.uiver.app/?q=WzAsNSxbMCwwLCIoQVxcb3QgKEIgXFxvdCBDKSkgXFxvdCBEIl0sWzIsMCwiQSBcXG90ICgoQiBcXG90IEMpIFxcb3QgRCkiXSxbMiwxLCJBIFxcb3QgKEIgXFxvdCAoQyBcXG90IEQpKSJdLFsxLDEsIihBIFxcb3QgQikgXFxvdCAoQyBcXG90IEQpIl0sWzAsMSwiKChBIFxcb3QgKEJcXG90IEMpIFxcb3QgRCkiXSxbMCwxLCJcXGFscGhhX3tBICwgQlxcb3QgQyAsIER9Il0sWzEsMiwiQSBcXG90IFxcYWxwaGFfe0IgLCBDICxEfSJdLFszLDIsIlxcYWxwaGFfe0EgLEIgLENcXG90IER9IiwyXSxbNCwzLCJcXGFscGhhX3tBIFxcb3QgQiAsIEMgLCBEfSIsMl0sWzQsMCwiXFxhbHBoYV97QSAsIEIgLEN9IFxcb3QgRCJdXQ==
\begin{tikzcd}
	{(A\ot (B\ot C)) \ot D} && {A \ot ((B \ot C) \ot D)} \\[-.2cm]
	{((A \ot B)\ot C) \ot D} & {(A \ot B) \ot (C \ot D)} & {A \ot (B \ot (C \ot D))}
	\arrow["{\alpha_{A , B\ot C , D}}", from=1-1, to=1-3]
	\arrow["{A \ot \alpha_{B , C ,D}}", from=1-3, to=2-3]
	\arrow["{\alpha_{A ,B ,C\ot D}}"', from=2-2, to=2-3]
	\arrow["{\alpha_{A \ot B , C , D}}"', from=2-1, to=2-2]
	\arrow["{\alpha_{A , B ,C} \ot D}", from=2-1, to=1-1]
\end{tikzcd}
\end{center}
A skew monoidal category with binary coproducts is \emph{(binary) left-distributive} if the canonical morphism typed $(A \ot C) + (B \ot C) \to (A + B) \ot C$ has an inverse $l : (A + B) \ot C \to (A \ot C) + (B \ot C)$. We will be interested in skew monoidal categories with binary products and coproducts, which moreover are left-distributive. We simply call these distributive skew monoidal categories.

A \emph{(strict) skew monoidal functor} $F : \mathbb{C} \rightarrow \mathbb{D}$ between skew monoidal categories $(\mathbb{C} , \I , \ot)$ and $(\mathbb{D} , \I' , \ot')$ is a functor from $\mathbb{C}$ to $\mathbb{D}$ satisfying
    $F \I = \I'$ and $F (A \ot B) = F A \ot' F B$, also preserving the structural laws $\lambda$, $\rho$ and $\alpha$ on the nose. A skew monoidal functor is \emph{distributive} if it also strictly preserves products, coproducts and left-distributivity.

The formulae, derivations and the equivalence relation $\circeq$ of the sequent calculus determine a \emph{syntactic} distributive skew monoidal category $\FSkMCC(\mathsf{At})$. Its objects are formulae. The operations $\I$ and $\ot$ are the logical connectives. The set of maps between objects $A$ and $B$ is the set of derivations $A \mid ~ \vdash B$ quotiented by the equivalence relation $\circeq$. The identity map on $A$ is the equivalence class of $\ax_A$, while composition is given by $\mathsf{scut}$. The structural laws $\lambda$, $\rho$, $\alpha$ are all admissible. Products and coproducts are the additive connectives $\land$ and $\lor$. Left-distributivity follows from the logical rules of $\lor$ and $\ot$.

Distributive skew monoidal categories form models of our sequent calculus.
Moreover, the sequent calculus as a presentation of a distributive skew monoidal category is the \emph{initial} one among these
models. Equivalently, $\FSkMCC(\mathsf{At})$ is the \emph{free}
such category on the set $\mathsf{At}$.
\begin{theorem}\label{thm:models}
  Let $\mathbb{D}$ be a distributive skew monoidal category. Given a function $F_{\mathsf{At}} : \mathsf{At} \rightarrow |\mathbb{D}|$ evaluating atomic formulae as objects of $\mathbb{D}$, there exists a unique distributive skew monoidal functor $F : \FSkMCC(\mathsf{At}) \rightarrow \mathbb{D}$ for which $F_0 X = F_{\mathsf{At}} X$ for any atom $X$.
\end{theorem}
The construction of the functor $F$ and the proof of uniqueness proceed similarly to the proofs of Theorems 3.1 and 3.2 in \cite{UVW:protsn}.

\section{A Focused Sequent Calculus with Tag Annotations}\label{sec:focusing}
When oriented from left-to-right, the equations in (\ref{fig:circeq}) become a  rewrite system, which is confluent with unique normal forms. Here we provide an explicit description of the proof normal forms of (\ref{eq:seqcalc}).
%Similar to the strategy employed in \cite{UVW:protsn},
For any sequent $S \mid \Gamma \vdash A$, a root-first proof search procedure can be defined as follows. First apply right invertible rules on the sequent until the principal connective of the succedent is non-negative, then apply left invertible rules until the stoup becomes either empty or non-positive.
At this stage, only non-invertible rules can be applied, so the question is: how to arrange the order between non-invertible rules without causing undesired non-determinism and losing completeness with respect to the sequent calculus in (\ref{eq:seqcalc}) and its equivalence $\circeq$?

Similarly to \cite{UVW:protsn}, our strategy is to prioritize left non-invertible rules over right ones, unless this does not lead to a valid derivation and the other way around is necessary.
For example, consider the sequent $X \land Y \mid \quad \vdash (X \land Y) \lor Z$. Proof search fails if we apply $\andli$ before $\orrone$. A valid proof is obtained only when applying $\orrone$ before $\andli$.

\niccolo{Maybe we can remove the proof tree below.}

\begin{equation}\label{eq:ex:focused}
  \begin{array}{c}
    \infer[\andlone]{X \land Y \mid \quad \vdash (X \land Y) \lor Z}{
      \infer[\orrone]{X \mid \quad \vdash (X \land Y) \lor Z}
      {
         \infer[\andr]{X \mid \quad \vdash X \land Y}{
           \infer[\ax]{X \mid \quad \vdash X}{}
           &
           \deduce{X \mid \quad \vdash Y}{??}
         }
      }
     }
     \qquad
     \infer[\orrone]{X \land Y \mid \quad \vdash (X \land Y) \lor Z}{
      \infer[\andr]{X \land Y \mid \quad \vdash X \land Y}{
        \infer[\andlone]{X \land Y \mid \quad \vdash X}{
          \infer[\ax]{X \mid \quad \vdash X}{}
        }
        &
        \infer[\andltwo]{X \land Y \mid \quad \vdash Y}{
          \infer[\ax]{Y \mid \quad \vdash Y}{}
        }
      }
    }
  \end{array}
\end{equation}

Note that following each right non-invertible rule, we should decompose succedents using right invertible rules.
In this case, there is only one right invertible rule, $\andr$ which would create two premises when applied and eventually lead to an application of a right non-invertible rule.
Therefore we introduce tag annotations to monitor the proof search process after an application of a right non-invertible rule.
We define four tags, $\tP , \tL , \tR , \tE$ which respectively correspond to $\pass , \andlone , \andltwo$ and the rest of non-invertible rules.
In particular, right non-invertible rules will produce a valid non-empty list of tags to mark their premises (only left premise is marked in the $\tr$ case).
\begin{defn}\label{tag:validity}
  A non-empty list of tags $l$ is valid if 
  \begin{itemize}
    \item there is one $\tE$ occurrence in $l$, or
    \item both of $\tL$ and $\tR$ have one occurrence in $l$.
  \end{itemize}
\end{defn}
The validity condition aims to prevent undesired non-determinism.
The first bullet covers the cases where all branches following a right non-invertible rule are applied by $\pass$ and the second covers the cases where all branches are applied only by one of $\andli$.
In both cases, left non-invertible rules should be applied before the right non-invertible rule.
% For example, considering a sequent $- \mid A' , \Gamma \vdash (A \land B) \lor C$, if 
% For the first bullet, an $\tE$ occurrence in $l$ means that there is one branch after a right non-invertible rule ending with $\ax , \unitr$, or a right non-invertible rule.

In the rest of this paper, lists of tags are always non-empty unless otherwise specified.
\\
Derivations in the focused sequent calculus with tag annotations are generated by the rules.
\begin{equation}\label{eq:focus}
  \begin{array}{lc}
    \text{(right invertible)} & %\\[-4pt] &
    \proofbox{
      \infer[\andr]{S \mid \Gamma \vdash^{l_1?, l_2?}_{\RI} A \land B}{
        S \mid \Gamma \vdash^{l_1?}_{\RI} A
        &
        S \mid \Gamma \vdash^{l_2?}_{\RI} B
      }
    \qquad
    \infer[\LI 2 \RI]{S \mid \Gamma \vdash^{t?}_{\RI} P}{S \mid \Gamma \vdash^{t?}_{\LI} P}
    }
    \\[10pt]
    \text{(left invertible)} & %\\[-4pt] &
    \proofbox{
      \infer[\unitl]{\I \mid \Gamma \vdash_{\LI} P}{{-} \mid \Gamma \vdash_{\LI} P}
    \qquad
    \infer[\tl]{A \ot B \mid \Gamma \vdash_{\LI} P}{A \mid B , \Gamma \vdash_{\LI} P}
    \\
    \infer[\orl]{A \lor B \mid \Gamma \vdash P}{
      A \mid \Gamma \vdash P
      &
      B \mid \Gamma \vdash P
    }
    \qquad
    \infer[\F 2 \LI]{T \mid \Gamma \vdash^{t?}_{\LI} P}{T \mid \Gamma \vdash^{t?}_{\F} P}
    }
    \\[10pt]
    \text{(focusing)} &    %\\[-4pt] &
    \proofbox{
    \infer[\pass]{{-} \mid A , \Gamma \vdash^{\tP?}_{\F} P }{
        A \mid \Gamma \vdash_{\LI} P
    }
    \qquad
    \infer[\ax]{X \mid \quad \vdash^{\tE?}_{\F} X}{}
    \qquad
    \infer[\unitr]{{-} \mid \quad \vdash^{\tE?}_{\F} \I}{}
    }
    \\[10pt]
    \multicolumn{2}{c}{
    \infer[\tr]{T \mid \Gamma , \Delta \vdash^{\tE?}_{\F} A \ot B}{
      T \mid \Gamma \vdash^{l}_{\RI} A
      &
      {-} \mid \Delta \vdash_{\RI} B
      &
      l \ \text{valid}
    }
    \qquad
    \infer[\orrone]{T \mid \Gamma \vdash^{\tE?}_{\F} A \lor B}{
      T \mid \Gamma \vdash^{l}_{\RI} A
      &
      l \ \text{valid}
    }
    \qquad
    \infer[\orrtwo]{T \mid \Gamma \vdash^{\tE?}_{\F} A \lor B}{
      T \mid \Gamma \vdash^{l}_{\RI} B
      &
      l \ \text{valid}
    }
    }
    \\[6pt]
    \multicolumn{2}{c}{
    \infer[\andlone]{A \land B \mid \Gamma \vdash^{\tL?}_{\F} P}{A \mid \Gamma \vdash_{\LI} P}
    \qquad
    \infer[\andltwo]{A \land B \mid \Gamma \vdash^{\tR?}_{\F} P}{B \mid \Gamma \vdash_{\LI} P}
    }
  \end{array}
\end{equation}
$P$ is a positive formula, which means it is not in the form of $\land$.
$T$ is a negative stoup, which means that it is not in the form of $\I , \ot$, or $\lor$.
Each tag (list of tags) notation with a question mark means that either the sequent is untagged or there is a corresponding tag (a list of tag).
We discuss the proof search procedures of untagged and tagged sequents separately.
\\
The proof search of a sequent $S \mid \Gamma \vdash_\RI A$ proceeds as follows:
\begin{itemize}
  \item[($\vdash_{\RI}$)] We apply the right invertible rule $\andr$ eagerly to decompose the succedent until its principal connective is not $\land$, then we move to the left invertible phase.
  \item[($\vdash_{\LI}$)] We apply left invertible rules until the stoup becomes negative, then move to the focusing phase.
  % \item[($\vdash_{\Pass}$)] In this phase, if the stoup is empty, then we can apply $\pass$ to move the head element of the context to stoup then go back to the left invertible phase.  If we do not (or cannot) apply $\pass$, we move to the focusing phase via $\F2 \Pass$.
  \item[($\vdash_{\F}$)] In the focusing phase, we apply one of the remaining rules. Since the sequents are untagged, $\pass, \ax , \unitr , \andli$ can be directly applied when stoups, contexts, and succedents are appropriate.
  For the right non-invertible rules, as mentioned above, their premises will be tagged by a list of tags and moved to the tagged right invertible phase.
\end{itemize}
The proof search of a sequent $T \mid \Gamma \vdash^{l}_\RI A$ proceeds as follows:
\begin{itemize}
  \item[($\vdash^{l}_{\RI}$)] We apply the $\andr$ rule to decompose the succedent and separate the list of tags carefully until the succedent is not in $\land$ and the list of tags is a singleton list, then we move to $\LI$ by $\LI2 \RI$.
  \item[($\vdash^{t}_{\LI}$)] Since the stoup is either empty or a negative formula, we switch to phase $\F$ immediately.
  % \item[($\vdash^{t}_{\Pass}$)] In this phase, $\pass$ can be applied only when the stoup is empty and the conclusion sequent has the tag $\tP$. For other cases, we should move to the focusing phase directly.
  \item[($\vdash^{t}_{\F}$)] We apply one of $\ax , \unitr$, and right non-invertible rules when the sequent is tagged with $\tE$ and the stoup and succedent formula are appropriate. 
  Left non-invertible rules can be applied only when the tag is correct, i.e.$\tP$, $\tL$, and $\tR$ respectively correspond to $\pass$, $\andlone$, and $\andltwo$.
\end{itemize}
We reconstruct the derivation on right-hand side in (\ref{eq:ex:focused}) within the focused calculus with tag annotations.
$\sw$ is an abbreviation of multiple phase switching rules.
\begin{equation}\label{eq:ex:tag:focused}
  \infer[\orrone]{X \land Y \mid \quad \vdash_{\F} (X \land Y) \lor Z}{
    \infer[\andr]{X \land Y \mid \quad \vdash^{[ \tL , \tR ]}_{\RI} X \land Y}{
      \infer[\sw]{X \land Y \mid \quad \vdash^{\tL}_{\RI} X}{
        \infer[\andlone]{X \land Y \mid \quad \vdash^{\tL}_{\F} X}{
          \infer[\sw]{X \mid \quad \vdash_{\LI} X}{
            \infer[\ax]{X \mid \quad \vdash_{\F} X}{}
          }
        }
      }
      &
      \infer[\sw]{X \land Y \mid \quad \vdash^{\tR}_{\RI} Y}{
        \infer[\andltwo]{X \land Y \mid \quad \vdash^{\tR}_{\F} Y}{
          \infer[\sw]{Y \mid \quad \vdash_{\LI} Y}{
            \infer[\ax]{Y \mid \quad \vdash_{\F} Y}{}
          }
        }
      }
    }
  }
\end{equation}
Notice that the list of tags is not pre-determined when a right non-invertible rule is applied.
For example, the proof search in (\ref{eq:ex:tag:focused}) actually proceeds as: we apply $\orrone$ and assume that there is a valid list of tags. We continue the proof search and apply the $\andli$ in the tagged phase. When the proof tree closes, we concatenate the singleton list from the leftmost sequent to right and check whether the result list is valid.
In this case, $[ \tL , \tR ]$ is valid.

\subsection{Meta properties of the focused calculus}
\begin{theorem}\label{theorem:focus:sound:complete}
  The focused calculus is sound and complete with the sequent calculus (\ref{eq:seqcalc}).
\end{theorem}
Soundness is immediate because there exist functions $\mathsf{emb}_{ph} : S \mid \Gamma \vdash^{l?}_{ph} A \to S \mid \Gamma \vdash A$, for all $ph \in \{ \RI , \LI , P , F \}$, which erase all phase and tag annotations.
Completeness follows from the fact that the following rules are all admissible:
\begin{equation}\label{eq:admis}
  \hspace*{-4mm}
    \begin{array}{c}
      \infer[\unitl^{\RI}]{\I \mid \Gamma \vdash_{\RI} C}{{-} \mid \Gamma \vdash_{\RI} C}
      \quad
      \infer[\tl^{\RI}]{A \ot B \mid \Gamma \vdash_{\RI} C}{A \mid B, \Gamma \vdash_{\RI} C}
      \quad
      \infer[\pass^{\RI}]{{-} \mid \Gamma \vdash_{\RI} C}{A \mid \Gamma \vdash_{\RI} C}
      \quad
      \infer[\ax^{\RI}]{A \mid \quad \vdash_{\RI} A}{}
      \quad
      \infer[\unitr^{\RI}]{{-} \mid \quad \vdash_{\RI} \I}{}
  \\[6pt]
      \infer[\orl^{\RI}]{A \lor B \mid \Gamma \vdash_{\RI} C}{
      A \mid \Gamma \vdash_{\RI} C
      &
      B \mid \Gamma \vdash_{\RI} C
      }
      \qquad
      \infer[\tr^{\RI}]{S \mid \Gamma , \Delta \vdash_{\RI} A \ot B}{
        S \mid \Gamma \vdash_{\RI} A
        &
        {-} \mid \Delta \vdash_{\RI} B
      }
  \\[6pt]
      \infer[\andlone^{\RI}]{A \land B \mid \Gamma \vdash_{\RI} C}{A \mid \Gamma \vdash_{\RI} C}
      \qquad
      \infer[\andltwo^{\RI}]{A \land B \mid \Gamma \vdash_{\RI} C}{B \mid \Gamma \vdash_{\RI} C}
      \qquad
      \infer[\orrone^{\RI}]{S \mid \Gamma \vdash_{\RI} A \lor B}{S \mid \Gamma \vdash_{\RI} A}
      \qquad
      \infer[\orrone^{\RI}]{S \mid \Gamma \vdash_{\RI} A \lor B}{S \mid \Gamma \vdash_{\RI} B}
    \end{array}
  \end{equation}
We prove admissibility of the rules except right non-invertible ones by structural induction on derivations.
The same strategy does not apply to right non-invertible rules because of the case that derivation ends with $\andr$.
Taking $\orrone^{\RI}$ for an instance:
\begin{displaymath}
  \begin{array}{c}
    \proofbox{
    \infer[\orrone^{\RI}]{S \mid \Gamma \vdash_{\RI} (A' \land B') \lor B}{
      \infer[\andr]{S \mid \Gamma \vdash_{\RI} A' \land B'}{
        \deduce{S \mid \Gamma \vdash_{\RI} A'}{f}
        &
        \deduce{S \mid \Gamma \vdash_{\RI} B'}{g}
      }
    }
    }
    \quad
    =
    \quad
    ??
  \end{array}
\end{displaymath}
The inductive hypothesis applied to $f$ and $g$ would produce wrong sequents for the target conclusion.
We fix the issue by following three lemmata.
\begin{lemma}\label{lem:RI:invert}
  Phase $\RI$ is invertible, which means that for every $f : S \mid \Gamma \vdash_{\RI} A$, there is a list of sequents $fs : [S \mid \Gamma \vdash_{\LI} A_i]_{i \in [1 , \dots , n]}$ such that $f = \andr^{*} fs$.
\end{lemma}
\begin{proof}
  Proof proceeds by structural induction on $f : S \mid \Gamma \vdash_{\RI} A$.
  \begin{itemize}
    \item If $f = \LI2 \RI \text{ } f_1$, then $A$ is a positive formula and $fs = S \mid \Gamma \vdash_{\LI} A$.
    \item If $f = \andr \text{ } f_1 f_2$, then by inductive hypothesis, we have $fs_1 : [S \mid \Gamma \vdash_{\LI} A_i]_{i \in [1 , \dots , n]}$ and $fs_2 : [S \mid \Gamma \vdash_{\LI} B_i]_{i \in [1 , \dots , m]}$ such that $fs = [fs_1 , fs_2]$.
  \end{itemize}
\end{proof}
\begin{lemma}\label{lem:BigStep}
  \begin{displaymath}
    \infer[\andr^{*}_t]{T \mid \Gamma \vdash^{l}_{\RI} A}{
      \deduce{[T \mid \Gamma \vdash^{t_i}_{\F} A_i]_{i \in [1 , \dots , n]}}{fs}
    }
  \end{displaymath}
  is admissible, where $l = [t_1 , \dots , t_n]$ and $A = A_1 \land \dots \land A_n$.
\end{lemma}
\begin{proof}
  Proof proceeds by induction on $A$.
  \begin{itemize}
    \item If $A$ is a positive formula, we have $fs = [ f ]$ and $\andr^*_{t} \text{ } fs = \F 2 \LI \text{ } (\LI 2 \RI \text{ } f)$.
    \item If $A = A' \land B'$, we define:
    \begin{displaymath}
      \small
      \begin{array}{cc}
        \infer[\andr^*_{t}]{T \mid \Gamma \vdash^{l} A' \land B'}{
          \deduce{[T \mid \Gamma \vdash^{t_i}_{\F} A_i]_{i \in [1 , \dots , n]}}{fs}
        }
        \\
        &
        =
        \quad
        \proofbox{
        \infer[\andr]{T \mid \Gamma \vdash^{l_1 , l_2}_{\RI} A' \land B'}{
          \infer[\andr^{*}_{t}]{T \mid \Gamma \vdash^{l_1}_{\RI} A'}{
            \deduce{[T \mid \Gamma \vdash^{t_i}_{\F} A'_i]_{i \in [1 , \dots , m_1]}}{fs_1}
          }
          &
          \infer[\andr^{*}_{t}]{T \mid \Gamma \vdash^{l_2}_{\RI} B'}{
            \deduce{[T \mid \Gamma \vdash^{t_i}_{\F} B'_i]_{i \in [1 , \dots , m_2]}}{fs_2}
          }
        }
       }
      \end{array}
    \end{displaymath}
    where $l = [l_1 , l_2]$.
  \end{itemize}
\end{proof}
\begin{lemma}\label{lem:GenRightRules}
  Generalized $\orri^{\LI}$ and $\tr^{\LI}$
  \begin{displaymath}
    \begin{array}{c}
      \infer[\orrone^{\LI}]{S \mid \Gamma \vdash_{\LI} A \lor B}{
        \deduce{[S \mid \Gamma \vdash_{\LI} A_i]_{i \in [1 , \dots , n]}}{fs}
      }
      \quad
      \infer[\orrtwo^{\LI}]{S \mid \Gamma \vdash_{\LI} A \lor B}{
        \deduce{[S \mid \Gamma \vdash_{\LI} B_i]_{i \in [1 , \dots , n]}}{fs}
      }
      \\
      \infer[\tr^{\LI}]{S \mid \Gamma , \Delta \vdash_{\LI} A \ot B'}{
        \deduce{[S \mid \Gamma \vdash_{\LI} A_i]_{i \in [1 , \dots , n]}}{fs}
        &
        - \mid \Delta \vdash_{\RI} B'
      }
    \end{array}
  \end{displaymath}
  are admissible in the focused calculus, where $A = A_1 \land \dots \land A_n$ and $B = B_1 \land \dots \land B_n$.
\end{lemma}
\begin{proof}
  The list of derivations $fs$ is non-empty, so we let $fs = [f_1 , fs']$.
  Proof proceeds by induction on $f_1$.
  We present the proof of $\orrone^{\LI}$, while the proofs of other two cases, $\orrtwo^{\LI}$ and $\tr^{\LI}$, are similar.
  \begin{itemize}
    \item If $f_1$ is the conclusion of a left invertible rule, then it is necessary that the remaining elements in $fs'$ end with the same rule as well.
    Therefore, we permute the rule with $\orrone^{\LI}$ and apply the inductive hypothesis.
    \item If $f_1 = \F 2 \LI f'_1$, then it is necessary for all the remaining elements in $fs'$ to also end with $\F 2 \LI$. In this case, we apply $\F 2 \LI$ to all elements in $fs$ and generate a list of tags $l$ by examining each element in $fs$.
    We add $\tP$ for each $\pass$, $\tL$ for each $\andlone$, $\tR$ for each $\andltwo$, and $\tE$ for all other rules in $\F$.
    There are two possibilities:
    \begin{itemize}
      \item $l$ is valid. In this case, we switch to $\F$ and apply  $\orrone^{\LI}$ followed by $\andr^{*}_{t}$:
      \begin{displaymath}
        \begin{array}{cc}
          \infer[\orrone^{\LI}]{T \mid \Gamma \vdash_{\LI} A \lor B}{
            \infer[ {[\F 2 \LI]} ]{[T \mid \Gamma \vdash_{\LI} A_i]_{i \in [1 , \dots , n]}}{
              \deduce{[T \mid \Gamma \vdash_{\F} A_i]_{i \in [1 , \dots , n]}}{fs^{*}}
            }
          }
          \\[10pt]
          &
          =
          \quad
          \proofbox{
          \infer[\F 2 \LI]{T \mid \Gamma \vdash_{\LI} A \lor B}{
            \infer[\orrone]{T \mid \Gamma \vdash_{\F} A \lor B}{
              \infer[\andr^{*}_t]{T \mid \Gamma \vdash^{l}_{\RI} A}{
                \deduce{[T \mid \Gamma \vdash^{t_i}_{\F} A_i]_{i \in [1 , \dots , n]}}{fs^{*'}}
              }
            }
          }
        }
        \end{array}
      \end{displaymath}
      Note that $fs^{*'}$ is a list of sequents whose sequents are tagged version of those in $fs^{*}$.
      For example, an $f_i : A' \land B' \mid \Gamma \vdash_{\F} A_i$ in $fs^{*}$ is corresponding to $A' \land B' \mid \Gamma \vdash^{\tL}_{\F} A_i$ in $fs^{*'}$.
      % \begin{displaymath}
      %   \begin{array}{c}
      %     \proofbox{
      %       \infer[\andlone]{A' \land B' \mid \Gamma \vdash_{\F} A_i}{A' \mid \Gamma \vdash_{\LI} A_i}
      %     }
      %     \quad
      %     \text{corresponds to}
      %     \quad
      %     \proofbox{
      %     \infer[\andlone]{A' \land B' \mid \Gamma \vdash^{\tL}_{\F} A_i}{A' \mid \Gamma \vdash_{\LI} A_i}
      %   }
      %   \end{array}
      % \end{displaymath}
      % in $fs^{*'}$.
      \item $l$ is invalid. In this case, all elements in $fs$ end with the same left non-invertible rule, so we permute the rule down with $\orrone^{\LI}$ and continue recursively.
      Here is an example where all elements in $fs$ are conclusions of $\pass$:
      \begin{displaymath}
        \begin{array}{cc}
          \proofbox{
            \infer[\orrone^{\LI}]{- \mid A' , \Gamma \vdash_{\LI} A \lor B}{
              \infer[ {[\F 2 \LI]} ]{[- \mid A' , \Gamma \vdash_{\LI} A_i]_{i \in [1 , \dots , n]}}{
                \infer[ {[\pass]} ]{[- \mid A' , \Gamma \vdash_{\LI} A_i]_{i \in [1 , \dots , n]}}{
                  \deduce{[A' \mid \Gamma \vdash_{\LI} A_i]_{i \in [1 , \dots , n]}}{fs^{*}}
                }
              }
            }
          }
          \\[10pt]
          &
          =
          \quad
          \proofbox{
            \infer[\F 2 \LI]{- \mid A' , \Gamma \vdash_{\LI} A \lor B}{
              \infer[\pass]{- \mid A' , \Gamma \vdash_{\F} A \lor B}{
                \infer[\orrone^{\LI}]{A' \mid \Gamma \vdash_{\LI} A \lor B}{
                  \deduce{[A' \mid \Gamma \vdash_{\LI} A_i]_{i \in [1 , \dots , n]}}{fs^{*}}
                }
              }
            }
          }
        \end{array}
      \end{displaymath}
    \end{itemize}
  \end{itemize}
  % \begin{itemize}
  %   \item If $f_1$ is a conclusion of an left invertible rule, then we simply permute the rule with $\orrone^{\LI}$ and apply the inductive hypothesis.
  %   \item If $f_1 = \Pass 2 \LI \text{ } f'_1$, then we do a further case analysis on $f'_1$:
  %   \begin{itemize}
  %     \item If $f'_1 = \pass \text{ } f''_1$, then we check the rest sequents in $fs$. 
  %     \begin{itemize}
  %       \item If all $f_i$ are in the form of $\Pass 2 \LI \text{ } (\pass \text{ } f''_i)$ then we permute down $\Pass 2 \LI $ and $\pass \text{ }$ and apply inductive hypothesis.
  %       \begin{displaymath}
  %         \begin{array}{cc}
  %           \proofbox{
  %             \infer[\orrone^{\LI}]{- \mid A' , \Gamma \vdash_{\LI} A \lor B}{
  %               \infer[ {[\Pass 2 \LI]} ]{- \mid A' , \Gamma \vdash_{\LI} A_i , i \in [1 , \dots , n]}{
  %                 \infer[ {[\pass]} ]{- \mid A', \Gamma \vdash_{\Pass} A_i , i \in [1 , \dots , n]}{
  %                   \deduce{[A' \mid \Gamma \vdash_{\LI} A_i]_{i \in [1 , \dots , n]}}{fs}
  %                 }
  %               }
  %             }
  %           }
  %           \\[10pt]
  %           &
  %           =
  %           \quad
  %           \proofbox{
  %             \infer[\Pass 2 \LI]{- \mid A' , \Gamma \vdash_{\LI} A \lor B}{
  %               \infer[\pass]{- \mid A' , \Gamma \vdash_{\Pass} A \lor B}{
  %                 \infer[\orrone^{\LI}]{A' \mid \Gamma \vdash_{\LI} A \lor B}{
  %                   \deduce{[A' \mid \Gamma \vdash_{\LI} A_i]_{i \in [1 , \dots , n]}}{fs}
  %                 }
  %               }
  %             }
  %           }
  %         \end{array}
  %       \end{displaymath}
  %       Rules with brackets $[ \ ]$ are applied to each element in the list.
  %       \item If there exists an $f_j = \F 2 \Pass f'_j$, then there exists a list of sequents $fs' : [- \mid A' , \Gamma \vdash^{t_i}_{\Pass} A_i]_{i \in 1 , \dots , n}$ and a list of tags $l = [t_1 , \dots , t_n]$, where $fs$ is obtained by removing tags and apply $\Pass 2 \LI$ on each elements in $fs'$.
  %       The corresponding tag to $f'_j$ is $\tE$ because it is impossible to be a conclusion of any of $\andli$, so the validity of $l$ is justified.
  %       With $fs'$ and $l$ in hand, we define: 
  %       \begin{displaymath}
  %         \begin{array}{cc}
  %           \infer[\orrone^{\LI}]{- \mid A' , \Gamma \vdash_{\LI} A \lor B}{
  %             \deduce{[- \mid A' , \Gamma \vdash_{\LI} A_i]_{i \in [1 , \dots , n]}}{fs}
  %           }
  %           \\[10pt]
  %           &
  %           =
  %           \quad
  %           \proofbox{
  %           \infer[\sw]{- \mid A' , \Gamma \vdash_{\LI} A \lor B}{
  %             \infer[\orrone]{- \mid A' , \Gamma \vdash_{\F} A \lor B}{
  %               \infer[\andr^{*}_t]{- \mid A' , \Gamma \vdash^{l}_{\RI} A}{
  %                 \deduce{[- \mid A' , \Gamma \vdash^{t_i}_{\Pass} A_i]_{i \in [1 , \dots , n]}}{fs'}
  %               }
  %             }
  %           }
  %         }
  %         \end{array}
  %       \end{displaymath}
  %     \end{itemize}
  %     \item If $f'_1 = \F 2 \Pass \text{ } f''_1$, then we do a further case analysis on $f''_1$:
  %     \begin{itemize}
  %       \item If $f''_1 = \andlone \text{ } f'''_1$, then we check if the rest elements of $fs$ have the same structure with $f'_1$. If they do, then we permute $\andlone$ down and apply the inductive hypothesis.
  %       If it is not the case, then either there exists an $f_j = \Pass 2 \LI \text{ } (\F 2 \Pass \text{ } (\andltwo f'_j))$ or $f_j = \Pass 2 \LI \text{ } (\F 2 \Pass \text{ } f'_j)$ where $f'_j$ is a conclusion of a rule other than $\andlone$.
  %       In either case, there exists a list of sequents $fs' : [A' \land B' \mid \Gamma \vdash^{t_i}_{\Pass} A_i]_{i \in [1 , \dots , n]}$ and a list of tags $l = [t_1 , \dots , t_n]$.
  %       In the forme case, $f_j$ is corresponding to tag $\tR$ and in the latter, it corresponds to $\tE$ so $l$ is valid in both cases.
  %       We can define a derivation analogously as above.
  %       \item $f''_1 = \andltwo \text{ } f'''_1$ is symmetric to the first case.
  %       \item If $f''_1$ is a derivation ending with other rules, then a similar argument applies.
  %       Notice that $l$ is always valid since $t_1 = \tE$.
  %     \end{itemize}
  %   \end{itemize} 
  % \end{itemize}
\end{proof}
Following the admissible rules in (\ref{eq:admis}), we construct a function $\mathsf{focus} : S \mid \Gamma \vdash A \to S \mid \Gamma \vdash_{\RI} A$ which replaces applications of each rule in (\ref{eq:seqcalc}) within a derivation by the corresponding admissible rule in phase $\RI$.
For example, $\mathsf{focus} \ (\orrone \ f) = \orrone^{\RI} \ (\mathsf{focus} \ f)$.

Furthermore, it can be proved that $\mathsf{emb}_{\RI}$ and $\mathsf{focus}$ are inverse functions of each other and following corollary holds.
\begin{corollary}
  The set of derivations of $S \mid \Gamma \vdash A$ quotiented by $\circeq$ and the set of derivations of $S \mid \Gamma \vdash_{\RI} A$ have a bijective correspondence.
  \begin{itemize}
    \item For all $f, g : S \mid \Gamma \vdash A$, if $f \circeq g$ then $\mathsf{focus} \text{ } f = \mathsf{focus} \text{ } g$.
    \item Given any $f : S \mid \Gamma \vdash A$, $\mathsf{emb}_{\RI} \text{ } (\mathsf{focus} \text{ } f) \circeq f$.
    \item Given any $f : S \mid \Gamma \vdash_{\RI} A$, $\mathsf{focus} \text{ } (\mathsf{emb}_{\RI} \text{ } f) = f$.
  \end{itemize}
\end{corollary}
\begin{proof}
  By structural induction on derivations.
\end{proof}

\section{Extensions of the Logic}\label{sec:extensions}
\subsection{Additive units}\label{subsec:AddUnits}

\subsection{Exchange}\label{subsec:Ex}
We consider the commutative extension of (\ref{eq:seqcalc}) by adding a rule 
\begin{displaymath}
  \infer[\ex]{S \mid \Gamma , B , A , \Delta \vdash C}{
    S \mid \Gamma , A , B , \Delta \vdash C
  }
\end{displaymath}
which swaps adjacent formulae in context.
Note that exchanging stoup formula (when it is non-empty) with a formula in context is forbidden.
Sequent calculus (\ref{eq:seqcalc}) extended with $\ex$ is an internal language of skew symmetric monoidal categories with products and coproducts since $\ex$ characterizes the natural isomorphism $s_{A , B , C} : A \ot (B \ot C) \to A \ot (C \ot B)$ which represents skew symmetry in \cite{bourke:lack:braided:2020}.

To accommodate the new $\ex$ rule, it is necessary to expand the congruence relation $\circeq$ in (\ref{fig:circeq}) with additional generating equations:
\begin{equation}\label{fig:circeq:sym}
  % \footnotesize
  \arraycolsep=2pt
  \def\arraystretch{1.1}
  \hspace{-8.9pt}
\begin{array}{rclll}
\ex_{B , A}  (\ex_{A , B}  f) &\circeq & f &&f: S \mid \Gamma , A , B , \Delta \vdash C
\\
\ex_{A , B}  (\ex_{A , D}  (\ex_{B , D}  f)) &\circeq & \ex_{B , D}  (\ex_{A , D}  (\ex_{A , B}  f)) &&f : S \mid \Gamma , A , B , D , \Delta \vdash C \\
  \unitl \text{ } (\ex_{A , B} f) &\circeq & \ex_{A , B} (\unitl \text{ } f) &&f: {-} \mid \Gamma , A , B , \Delta \vdash C
  \\
  \pass \text{ } (\ex_{A , B}  f) &\circeq & \ex_{A , B}  (\pass \text{ } f) &&f : A' \mid \Gamma , A , B , \Delta \vdash C
  \\
  \tl \text{ } (\ex_{A , B}  f) &\circeq & \ex_{A , B}  (\tl \text{ } f) &&f : A' \mid B' , \Gamma , A , B , \Delta \vdash C
  \\
  \tr \text{ }(\ex_{A , B}  f , g) &\circeq & \ex_{A , B}  (\tr \text{ } (f , g)) && f : S \mid \Gamma_0 , A , B , \Gamma_1 \vdash A',
  \\ & & && g : {-} \mid \Delta \vdash B'
  \\
  \tr \text{ } (f , \ex_{A , B}  g) &\circeq & \ex_{A , B}  (\tr \text{ } (f , g)) &&f : S \mid \Gamma \vdash A', \\ & & && g : {-} \mid \Delta_0 , A , B , \Delta_1 \vdash B'
  \\
  \andlone \ (\ex_{A , B} \ f) &\circeq &\ex_{A , B} \ (\andlone \ f) &&f : A' \mid \Gamma , A , B , \Delta \vdash C
  \\
  \andltwo \ (\ex_{A , B} \ f) &\circeq &\ex_{A , B} \ (\andltwo \ f) &&f : B' \mid \Gamma , A , B , \Delta \vdash C
  \\
  \andr \ (\ex_{A , B} \ f , \ex_{A , B} \ g) &\circeq &\ex_{A , B} \ (\andr \ (f , g)) &&f : S \mid \Gamma , A , B , \Delta \vdash A' , g : S \mid \Gamma , A , B , \Delta \vdash B'
  \\
  \orl \ (\ex_{A , B} \ f , \ex_{A , B} \ g) &\circeq &\ex_{A , B} \ (\orl \ (f , g)) &&f : A' \mid \Gamma , A , B , \Delta \vdash C ,  g : B' \mid \Gamma , A , B , \Delta \vdash C
  \\
  \orrone \ (\ex_{A , B} \ f) &\circeq &\ex_{A , B} \ (\orrone \ f) &&f : S \mid \Gamma , A , B \vdash A'
  \\
  \orrtwo \ (\ex_{A , B} \ f) &\circeq &\ex_{A , B} \ (\orrtwo \ f) &&f : S \mid \Gamma , A , B \vdash B'
  \\
  \ex_{A , B}  (\ex_{A' , B'}  f) &\circeq & \ex_{A' , B'}  (\ex_{A , B}  f) &&f: S \mid \Gamma , A , B , \Delta , A' , B' , \Lambda \vdash C	
\end{array}
\end{equation}
The first equation states that performing the operation of swapping the same two formulas twice yields the same result as doing nothing.
The second equation is a for of the Yang-Baxter equation, which states that the two possible ways of interchanging the order of three adjacent formulas $A , B , C$ to obtain the sequence $C , B , A$ are equivalent.
The remaining equations are permutative conversions.

We study the proof-theoretic semantics of (\ref{eq:seqcalc}) extended with $\ex$ by accommodating the additive rules to the focused sequent calculus introduced in \cite{veltri:coherence:2021}.
\begin{equation}\label{eq:focus:sym}
  \begin{array}{lc}
    \text{(Context)} &
    \proofbox{
      \infer[\ex]{S \mid \Gamma , A \spl \Delta , \Lambda \vdash^{l?}_{\C} C}{S \mid \Gamma \spl \Delta , A , \Lambda \vdash_{\C} C}
      \qquad
      \infer[\RI 2 \C]{S \mid \quad \spl \Gamma \vdash^{l?}_{\C} C}{S \mid \Gamma \vdash^{l?}_{\RI} C}
    }
    \\[20pt]
    \text{(right invertible)} & %\\[-4pt] &
    \proofbox{
      \infer[\andr]{S \mid \Gamma \vdash^{l_1?, l_2?}_{\RI} A \land B}{
        S \mid \Gamma \vdash^{l_1?}_{\RI} A
        &
        S \mid \Gamma \vdash^{l_2?}_{\RI} B
      }
    \qquad
    \infer[\LI 2 \RI]{S \mid \Gamma \vdash^{t?}_{\RI} P}{S \mid \Gamma \vdash^{t?}_{\LI} P}
    }
    \\[20pt]
    \text{(left invertible)} & %\\[-4pt] &
    \proofbox{
      \infer[\unitl]{\I \mid \Gamma \vdash_{\LI} P}{{-} \mid \Gamma \vdash_{\LI} P}
    \qquad
    \infer[\tl]{A \ot B \mid \Gamma \vdash_{\LI} P}{A \mid B \spl \Gamma \vdash_{\C} P}
    \\
    \infer[\orl]{A \lor B \mid \Gamma \vdash P}{
      A \mid \Gamma \vdash P
      &
      B \mid \Gamma \vdash P
    }
    \qquad
    \infer[\F 2 \LI]{T \mid \Gamma \vdash^{t?}_{\LI} P}{T \mid \Gamma \vdash^{t?}_{\F} P}
    }
    \\[20pt]
    \text{(focusing)} &    %\\[-4pt] &
    \proofbox{
    \infer[\pass]{{-} \mid A , \Gamma \vdash^{\tP?}_{\F} P }{
        A \mid \Gamma \vdash_{\LI} P
    }
    \qquad
    \infer[\ax]{X \mid \quad \vdash^{\tE?}_{\F} X}{}
    \qquad
    \infer[\unitr]{{-} \mid \quad \vdash^{\tE?}_{\F} \I}{}
    }
    \\[10pt]
    \multicolumn{2}{c}{
    \infer[\tr]{T \mid \Gamma , \Delta \vdash^{\tE?}_{\F} A \ot B}{
      T \mid \Gamma \vdash^{l}_{\C} A
      &
      {-} \mid \Delta \vdash_{\RI} B
      &
      l \ \text{valid}
    }
    \qquad
    \infer[\orrone]{T \mid \Gamma \vdash^{\tE?}_{\F} A \lor B}{
      T \mid \Gamma \vdash^{l}_{\C} A
      &
      l \ \text{valid}
    }
    \qquad
    \infer[\orrtwo]{T \mid \Gamma \vdash^{\tE?}_{\F} A \lor B}{
      T \mid \Gamma \vdash^{l}_{\C} B
      &
      l \ \text{valid}
    }
    }
    \\[6pt]
    \multicolumn{2}{c}{
    \infer[\andlone]{A \land B \mid \Gamma \vdash^{\tL?}_{\F} P}{A \mid \Gamma \vdash_{\LI} P}
    \qquad
    \infer[\andltwo]{A \land B \mid \Gamma \vdash^{\tR?}_{\F} P}{B \mid \Gamma \vdash_{\LI} P}
    }
  \end{array}
\end{equation}
There are two differences with (\ref{eq:focus}):
\begin{itemize}
  \item Root-first proof search begins in a new phase $\C$, where formulae in context are permuted.
  The idea is to start with a sequent whose formulae are arbitrarily ordered in the context then use $\ex$ to move each formula to its appropriate position.
  Consequently, we start with a sequent $S \mid \Gamma \spl \quad \vdash_{\C}$ and end with a sequent $S \mid \quad \spl \Gamma' \vdash_{\C}$ where $\Gamma'$ is a permutation of $\Gamma$.
  Throughout the process, the context is divided into two parts: $\Gamma \spl \Delta$, where the formulae in $\Gamma$ are ready to be moved while the those in $\Delta$ have already been placed in the correct position.
  Once all formulae in $\Gamma$ have been moved, we switch to phase $\RI$ and continue.
  Note that a sequent in $\C$ may be tagged with a list of tags $l$.
  However, as previously mentioned, we only assume the existence of $l$ during the proof search, and we examine the validity after the search is completed.
  Therefore, we do not consider how to move tags in phase $\C$.
  \item The other difference is in rule $\tl$ which is the only one rule that adds new formulae to the context from the bottom-up reading of the rules.
  The premises of these rules are now required to go back to phase $\C$ to utilize the rule $\ex$ to relocate the new formula $A$ to its correct position within the context.
\end{itemize}

\subsection{Partial associativity and unitality}\label{subsec:asso:uni}

\subsection{Linear implication}\label{subsec:impl}
We consider a skew version of non-commutative intuitionistic linear logic (\SkNMILL) with additive connectives (\SkNMILLA), which is presented by extending (\ref{eq:seqcalc}) with a connective $\lolli$ (linear implication) and the following two rules:
\begin{displaymath}
  \begin{array}{c}
    \infer[\lleft]{A \lolli B \mid \Gamma , \Delta \vdash C}{
      - \mid \Gamma \vdash A
      &
      B \mid \Delta \vdash C
    }
    \qquad
    \infer[\lright]{S \mid \Gamma \vdash A \lolli B}{S \mid \Gamma , A \vdash B}
  \end{array}
\end{displaymath}
Categorical models of \SkNMILLA are skew monoidal closed categories with products and coproducts.

The presence of $\lolli$ and its corresponding rules requires the extension of the congruence relation $\circeq$ with additional generating equations:
\begin{equation}\label{fig:circeq:impl}
  \begin{array}{rlll}
  \ax_{A \lolli B} &\circeq \lright \text{ } (\lleft \text{ } (\pass \text{ } \ax_{A}, \ax_{B} ))
  \\
  \tr \text{ } (\lleft \text{ } (f , g), h) & \circeq \lleft \text{ } (f, \tr \text{ } (g, h)) &&f: {-} \mid \Gamma \vdash A, g : B \mid \Delta \vdash C, h : {-} \mid \Lambda \vdash D
  \\
  \lright \text{ } (\pass \text{ } f) &\circeq \pass \text{ } (\lright \text{ } f) &&f : A' \mid \Gamma , A \vdash B
  \\
  \lright \text{ } (\unitl \text{ } f) &\circeq \unitl \text{ } (\lright \text{ } f) &&f : {-} \mid \Gamma , A \vdash B
  \\
  \lright \text{ } (\tl \text{ } f) &\circeq \tl \text{ } (\lright \text{ } f) &&f : A \mid B , \Gamma , C \vdash D
  \\
  \lright \text{ } (\lleft \text{ } (f, g)) &\circeq \lleft \text{ } (f, \lright \text{ } g) &&f : {-} \mid \Gamma \vdash A', g : B' \mid \Delta , A \vdash B
  \\
  \lright \ (\andlone \ f) &\circeq \andlone \ (\lright \ f) &&f : A' \mid \Gamma , A \vdash B
  \\
  \lright \ (\andltwo \ f) &\circeq \andltwo \ (\lright \ f) &&f : B' \mid \Gamma , A \vdash B
  \\
  \lright \ (\orl \ (f ,g)) &\circeq \orl \ (\lright \ f , \lright \ g) &&f : A' \mid \Gamma , A \vdash B , g : B' \mid \Gamma , A \vdash B
  \\
  \orrone \ (\lleft \  (f , g)) &\circeq \lleft \ (f , \orrone \ g) &&f : - \mid \Gamma \vdash A , g : B \mid \Delta \vdash A'
  \\
  \orrtwo \ (\lleft \  (f , g)) &\circeq \lleft \ (f , \orrtwo \ g) &&f : - \mid \Gamma \vdash A , g : B \mid \Delta \vdash B'
  \end{array}
\end{equation}
The first equation is about $\eta$-conversion of $\lolli$ and the remaining are permutative conversions.

To eliminate undesired non-determinism and maintain completeness with \SkNMILLA, we incorporate (\ref{eq:focus}) into the focused calculus in the section 4.2 of \cite{UVW:protsn} which employs tags to track formulae which are moved from the succedent to context via applications of the rule $\lright$.
Therefore in the new focused calculus, following any right non-invertible rule, a new applicable right invertible rule $\lright$ is introduced, where any formula $A$ moved from succedent to context via $\lright$ is marked as $A^{\bullet}$.
We use $\bullet$ as the tag where there is at least one tagged formula sent to the first premise of $\lleft$ following an right non-invertible rule.

The addition of $\lleft$ introduces a new type of non-determinism, as illustrated by the sequent $\I \lolli \I \mid \I , Y \vdash (\I \land \I) \ot Y$, which can be proved in different ways depending on how the context is split.
Here are two different proofs of the sequent $\I \lolli \I \mid \I , Y \vdash (\I \land \I) \ot Y$.
\begin{equation}\label{eq:ex:lleft:NonDeter}
  \begin{array}{c}
    \infer[\tr]{\I \lolli \I \mid \I , Y \vdash (\I \land \I) \ot Y}{
      \infer[\andr]{\I \lolli \I \mid \I \vdash \I \land \I}{
        \infer[\lleft]{\I \lolli \I \mid \I \vdash \I}{
      \infer[\unitr]{- \mid \quad \vdash \I}{}
      &
      \infer[\unitl]{\I \mid \I \vdash \I}{
        \infer[\pass]{- \mid \I \vdash \I}{
          \infer[\unitl]{\I \mid \quad \vdash \I}{
            \infer[\unitr]{- \mid \quad \vdash \I}{}
          }
        }
      }
    }
    &
    \infer[\lleft]{\I \lolli \I \mid \I \vdash \I}{
      \infer[\pass]{- \mid \I \vdash \I}{
        \infer[\unitl]{\I \mid \quad \vdash \I}{
          \infer[\unitr]{- \mid \quad \vdash \I}{}
        }
      }
      &
      \infer[\unitl]{\I \mid \quad \vdash \I}{
        \infer[\unitr]{- \mid \quad \vdash \I}{}
      }
    }
      }
      &
      \infer[\pass]{- \mid Y \vdash Y}{
        \infer[\ax]{Y \mid \quad \vdash Y}{}
      }
    }
    \\[10pt]
    \infer[\tr]{\I \lolli \I \mid \I , Y \vdash (\I \land \I) \ot Y}{
      \infer[\andr]{\I \lolli \I \mid \I \vdash \I \land \I}{
        \infer[\lleft]{\I \lolli \I \mid \I \vdash \I}{
      \infer[\unitr]{- \mid \quad \vdash \I}{}
      &
      \infer[\unitl]{\I \mid \I \vdash \I}{
        \infer[\pass]{- \mid \I \vdash \I}{
          \infer[\unitl]{\I \mid \quad \vdash \I}{
            \infer[\unitr]{- \mid \quad \vdash \I}{}
          }
        }
      }
    }
    &
    \infer[\lleft]{\I \lolli \I \mid \I \vdash \I}{
      \infer[\unitr]{- \mid \quad \vdash \I}{}
      &
      \infer[\unitl]{\I \mid \I \vdash \I}{
        \infer[\pass]{- \mid \I \vdash \I}{
          \infer[\unitl]{\I \mid \quad \vdash \I}{
            \infer[\unitr]{- \mid \quad \vdash \I}{}
          }
        }
      }
    }
      }
      &
      \infer[\pass]{- \mid Y \vdash Y}{
        \infer[\ax]{Y \mid \quad \vdash Y}{}
      }
    }
  \end{array}
\end{equation}
In this case, we should always perform right non-invertible rules before $\lleft$ to avoid losing possible derivations.
Note that this type of non-determinism is essential because it demonstrates the fact that there are sequents with multiple derivations in \SkNMILLA \ that are not equivalent under $\circeq$.

To capture this non-determinism and ensure completeness, we introduce a new type of tag denoted by $\# \Gamma$ which records the number of formula occurrences in $\Gamma$ that are sent as the first premise of an application of $\lleft$.
\begin{defn}\label{def:tag:validity:impl}
  A non-empty list of tags $l$ is valid if 
  \begin{itemize}
    \item there is one $\tE$ occurrence in $l$,
    \item both of $\tL$ and $\tR$ have one occurrence in $l$,
    \item there is one $\bullet$ in $l$, or
    \item there exists $\Gamma$ and $\Gamma'$ such that $\# \Gamma \neq \# \Gamma'$.
  \end{itemize}
\end{defn}
The validity condition is extended by two cases, where the first one distinguishes the situation where there is at least one new formula introduced in the context due to the previous right non-invertible rule, while the second captures cases of essential non-determinism and avoids situations where all the contexts of the corresponding sequents of $l$ are split in the same way.
It is worth noting that the context of an application of $\lleft$ can be split at different new formula occurrences.
% the focusing strategy in  the focused calculus of \SkNMILL is equipped with tag annotations to track new formulae occurrences in the context via $\lright$.
% We incorporate the tag annotations in the focused \SkNMILL with \ref{eq:focus} to obtain following rules.

Derivations in the focused sequent calculus of \SkNMILLA \ are generated by the rules.
\begin{equation}\label{eq:focus:impl}
  \begin{array}{lc}
    \text{(right invertible)} & %\\[-4pt] &
    \proofbox{
      \infer[\andr]{S \mid \Gamma \vdash^{l_1?, l_2?}_{\RI} A \land B}{
        S \mid \Gamma \vdash^{l_1?}_{\RI} A
        &
        S \mid \Gamma \vdash^{l_2?}_{\RI} B
      }
    \qquad
    \infer[\lright]{S \mid \Gamma \vdash^{l?}_{\RI} A \lolli B}{S \mid \Gamma , A^{\bullet ?} \vdash^{l?}_{\RI} B}
    \qquad
    \infer[\LI 2 \RI]{S \mid \Gamma \vdash^{t?}_{\RI} P}{S \mid \Gamma \vdash^{t?}_{\LI} P}
    }
    \\[10pt]
    \text{(left invertible)} & %\\[-4pt] &
    \proofbox{
      \infer[\unitl]{\I \mid \Gamma \vdash_{\LI} P}{{-} \mid \Gamma \vdash_{\LI} P}
    \qquad
    \infer[\tl]{A \ot B \mid \Gamma \vdash_{\LI} P}{A \mid B , \Gamma \vdash_{\LI} P}
    \\
    \infer[\orl]{A \lor B \mid \Gamma \vdash P}{
      A \mid \Gamma \vdash P
      &
      B \mid \Gamma \vdash P
    }
    \qquad
    \infer[\F 2 \LI]{T \mid \Gamma \vdash^{t?}_{\LI} P}{T \mid \Gamma \vdash^{t?}_{\F} P}
    }
    \\[10pt]
    \text{(focusing)} &    %\\[-4pt] &
    \proofbox{
    \infer[\pass]{{-} \mid A^{\bullet ?} , \Gamma \vdash^{\tP?}_{\F} P }{
        A \mid \Gamma^{\circ} \vdash_{\LI} P
    }
    \qquad
    \infer[\ax]{X \mid \quad \vdash^{\tE?}_{\F} X}{}
    \qquad
    \infer[\unitr]{{-} \mid \quad \vdash^{\tE?}_{\F} \I}{}
    }
    \\[10pt]
    \multicolumn{2}{c}{
    \infer[\tr]{T \mid \Gamma , \Delta \vdash^{\tE?}_{\F} A \ot B}{
      T \mid \Gamma^{\circ} \vdash^{l}_{\RI} A
      &
      {-} \mid \Delta^{\circ} \vdash_{\RI} B
      &
      l \ \text{valid}
    }
    \qquad
    \infer[\orrone]{T \mid \Gamma \vdash^{\tE?}_{\F} A \lor B}{
      T \mid \Gamma^{\circ} \vdash^{l}_{\RI} A
      &
      l \ \text{valid}
    }
    \qquad
    \infer[\orrtwo]{T \mid \Gamma \vdash^{\tE?}_{\F} A \lor B}{
      T \mid \Gamma^{\circ} \vdash^{l}_{\RI} B
      &
      l \ \text{valid}
    }
    }
    \\[6pt]
    \multicolumn{2}{c}{
    \infer[\andlone]{A \land B \mid \Gamma \vdash^{\tL?}_{\F} P}{A \mid \Gamma^{\circ} \vdash_{\LI} P}
    \qquad
    \infer[\andltwo]{A \land B \mid \Gamma \vdash^{\tR?}_{\F} P}{B \mid \Gamma^{\circ} \vdash_{\LI} P}
    }
    \\[6pt]
    \multicolumn{2}{c}{
      \infer[\lleft]{A \lolli B \mid \Gamma , \Delta^{\bullet ?} , \Lambda \vdash^{t?}_{\F} P}{
        - \mid \Gamma , \Delta^{\circ} \vdash_{\RI} A
        &
        B \mid \Lambda^{\circ} \vdash_{\LI} P
        % &
        % t \text{ exists } \supset \ (\Delta \text{ empty and } \ t = \# \Gamma) \text{ or } \ t = \bullet
        &
        \Delta \text{ empty } \supset \ t \text{ does not exist } \text{ or } t = \# \Gamma  \text{ else }  t = \bullet 
      }
    }
  \end{array}
\end{equation}
There are some differences with (\ref{eq:focus}):
\begin{itemize}
  \item The positive formula $P$ now means that $P$ is not in the form of $\land$ nor $\lolli$. $A^{\bullet}$ means that $A$ is tagged and $\Gamma^{\bullet}$ means that all the formula occurrences in $\Gamma$ are tagged. $\Gamma^{\circ}$ means that all the tags of the formula occurrences in $\Gamma$ are erased.
  \item $\lright$ is a new rule in phase $\RI$, which moves formulae from the succedent to the context. If the conclusion of $\lright$ is $T \mid \Gamma \vdash^{l}_{\RI} A \lolli B$, then $A$ in the premise is a new formula and tagged by $\bullet$.
  \item The side condition in rule $\lleft$ has two parts: first, if $\Delta$ is empty, the sequent is either untagged or split in different ways, and second, if $\Delta$ is inhabited, then $t = \bullet$.
  The inhabited or non-inhabited nature of $\Delta$ is crucial because if there is any formula occurrence in $\Delta$, then it indicates that there is an application of a right non-invertible rule before $\lleft$ and the order of rule applications is guaranteed due to the new formula occurrences in $\Delta$.
  If $\Delta$ is empty, then either there is no application of right non-invertible rule before or there is an application of a right non-invertible rule before $\lleft$ is but there is no formula moved to the context via $\lright$.
  In the latter case, tags are used to record the splitting of contexts.
\end{itemize}
We can reconstruct the upper proof in (\ref{eq:ex:lleft:NonDeter}) within (\ref{eq:focus:impl}).
\begin{displaymath}
  \infer[\sw]{\I \lolli \I \mid \I , Y \vdash_{\RI} (\I \land \I) \ot Y}{
    \infer[\tr]{\I \lolli \I \mid \I , Y \vdash_{\F} (\I \land \I) \ot Y}{
      \infer[\andr]{\I \lolli \I \mid \I \vdash^{0 , 1}_{\RI} \I \land \I}{
        \infer[\sw]{\I \lolli \I \mid \I \vdash^{0}_{\RI} \I}{
          \infer[\lleft]{\I \lolli \I \mid \I \vdash^{0}_{\F} \I}{
            \infer[\sw]{- \mid \quad \vdash_{\RI} \I}{
              \infer[\unitr]{- \mid \quad \vdash_{\F} \I}{}
            }
            &
            \infer[\unitl]{\I \mid \I \vdash_{\LI} \I}{
              \infer[\F 2 \LI]{- \mid \I \vdash_{\LI} \I}{
                \infer[\pass]{- \mid \I \vdash_{\F} \I}{
                  \infer[\unitl]{\I \mid \quad \vdash_{\LI} \I}{
                    \infer[\F 2 \LI]{- \mid \quad \vdash_{\LI} \I}{
                      \infer[\unitr]{- \mid \quad \vdash_{\F} \I}{}
                    }
                  }
                }
              }
            }
          }
        }
        &
        \infer[\sw]{\I \lolli \I \mid \I \vdash^{1}_{\RI} \I}{
          \infer[\lleft]{\I \lolli \I \mid \I \vdash^{1}_{\RI} \I}{
            \infer[\sw]{- \mid \I \vdash_{\RI} \I}{
              \infer[\pass]{- \mid \I \vdash_{\F} \I}{
                \infer[\unitl]{\I \mid \quad \vdash_{\LI} \I}{
                  \infer[\F 2 \LI]{- \mid \quad \vdash_{\LI} \I}{
                    \infer[\unitr]{- \mid \quad \vdash_{\F} \I}{}
                  }
                }
              }
            }
            &
            \infer[\unitl]{\I \mid \quad \vdash_{\LI} \I}{
              \infer[\F 2 \LI]{- \mid \quad \vdash_{\LI} \I}{
                \infer[\unitr]{- \mid \quad \vdash_{\F} \I}{}
              }
            }
          }
        }
      }
      &
      \infer[\sw]{- \mid Y \vdash_{\RI} Y}{
        \infer[\pass]{- \mid Y \vdash_{\F} Y}{
          \infer[\F 2 \LI]{Y \mid \quad \vdash_{\LI} Y}{
            \infer[\ax]{Y \mid \quad \vdash_{\F} Y}{}
          }
        }
      }
    }
  }
\end{displaymath}
In this case $[0 , 1]$ is a valid list.

On the other hand, the bottom derivation in (\ref{eq:ex:lleft:NonDeter}) should correspond to a derivation starting with $\lleft$, otherwise it is not a proof within (\ref{eq:focus:impl}) because of an invalid list of tags $[0 , 0]$.
% \begin{displaymath}
%   \begin{array}{cc}
%     \text{valid proof} &
%     \proofbox{

%     }
%     \\
%     \text{invalid proof} &
%     \proofbox{

%     }
%   \end{array}
% \end{displaymath}

% To obtain a generalization of theorem (\ref{theorem:focus:sound:complete}), we generalize the three lemmata (\ref{lem:RI:invert})
\section{Formalization}\label{sec:formalization}


\section{Conclusion}
The paper presents an additive extension of the sequent calculus in \cite{uustalu:sequent:2021}, whose categorical interpretation is skew monoidal categories with products and coproducts.
Derivations in the calculus are quotiented by the congruence relation $\circeq$, where each equivalence class corresponds to precisely one derivation in the focused calculus with tag annotations.
In order to eliminate non-determinism related to permutative conversions of non-invertible rules, the focused calculus prioritizes the left-invertible rules over the right ones and employs tag annotations to ensure completeness.
Moreover, the focused calculus provides a procedure to solve the coherence problem of skew monoidal categories with products and coproducts since $\circeq$ is carefully defined in accord with the equations of morphisms in the categories.
 
This paper takes one step further of a large project aiming at modular analysis of proof systems whose categorical models are categories with skew structure \cite{zeilberger:semiassociative:19, uustalu:sequent:2021,uustalu:proof:nodate,uustalu:deductive:nodate,veltri:coherence:2021,UVW:protsn}.
We plan to extend the sequent calculus in this paper to include the linear implication, which is a skew version of noncommutative intuitionistic linear logic with additive conjunction and disjunction and has skew monoidal closed categories with products and coproducts as its models.
We expect that the tag annotations, which guarantee correct proof searches in this paper, will complement nicely with the tag annotations in \cite{UVW:protsn}, which track new formulae in context.
% We expect the tag annotations as guarantee of correct proof searches in this paper can nicely cooperate with the one in \cite{UVW:protsn} that tracks new formulae in context.


\paragraph{Acknowledgements}
This work was supported by the Estonian Research Council grant PSG749.
% N.V.\ and T.U.\ were supported by the
% Estonian Research Council grants no. \linebreak PSG659, PSG749 and PRG1210, N.V.\ and
% C.-S.W.\ by the ESF funded Estonian IT Academy research measure
% (project 2014-2020.4.05.19-0001) and partially supported by  COST CA19135 - Connecting Education and Research Communities for an Innovative Resource Aware Society. T.U.\ was supported by the Icelandic
% Research Fund grant no.~196323-053.

  \bibliographystyle{eptcs}
  \bibliography{LSFA}
\end{document}
