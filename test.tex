\documentclass{article}
\usepackage{amsmath}
\usepackage{amsfonts}
\usepackage{amssymb}
\usepackage{enumerate}
\usepackage{prftree}
\usepackage{ntheorem}
\usepackage{tikz-cd}
\usepackage{hyperref}
\usepackage{float}
\usepackage{graphicx}
\usepackage{quiver}
\usepackage{bussproofs}
\usepackage[a4paper, total={6in, 8in}]{geometry}
\usepackage{lscape}
\theorembodyfont{}
\newtheorem{theorem}{Theorem}
\newtheorem{corollary}[theorem]{Corollary}
\newtheorem{lemma}[theorem]{Lemma}
\newtheorem{defn}[theorem]{Definition}
\newtheorem{remark}[theorem]{Remark}
\newtheorem*{proof}{Proof : }
\newtheorem{fact}[theorem]{Fact}
\newcommand{\ldbc}{[\![}
\newcommand{\rdbc}{]\!]}
\newcommand{\tbar}{[\vec{x}/\vec{t}]}
\newcommand{\ltbar}{[\vec{x}, x/\vec{t}, x]}
\newcommand{\tl}{\otimes \mathsf{L}}
\newcommand{\tr}{\otimes \mathsf{R}}
\newcommand{\lright}{\multimap \mathsf{R}}
\newcommand{\lleft}{\multimap \mathsf{L}}
\newcommand{\pass}{\mathsf{pass}}
\newcommand{\unitl}{\mathsf{IL}}
\newcommand{\unitr}{\mathsf{IR}}
\newcommand{\ax}{\mathsf{ax}}
\newcommand{\id}{\mathsf{id}}
\newcommand{\ot}{\otimes}
\newcommand{\lolli}{\multimap}
\newcommand{\I}{\mathsf{I}}
\newcommand{\msfL}{\mathsf{L}}
\newcommand{\defeq}{=_{\mathsf{df}}}
\newcommand{\comp}{\mathsf{comp}}
\newcommand{\RI}{\mathsf{RI}}
\newcommand{\LI}{\mathsf{LI}}
\newcommand{\Pass}{\mathsf{P}}
\newcommand{\F}{\mathsf{F}}
\begin{document}
\begin{titlepage}
     \vspace*{\stretch{1.0}}
   \begin{center}
      \Large\textbf{A Focused Sequent Calculus System of Skew Monoidal Closed Categories}\\
      \large\textit{Tarmo Uustalu}\\
      \large\textit{Niccol\`o Veltri}\\
      \large\textit{Cheng-Syuan Wan}
   \end{center}
   \vspace*{\stretch{2.0}}
\end{titlepage}
\section{Introduction}
Recent discoveries on skew monoidal categories \cite{Szlachanyi} \cite{Lack2012} \cite{Lack2014} (check references in previous papers, at least include papers cited by previous three papers) provide us a good reasons to study their corresponding proof systems.
In previous work \cite{Uustalu2018a} \cite{Uustalu2020a} \cite{Uustalu2020b} \cite{Veltri2021}, sequent calculus proof

  \bibliographystyle{plain}
  \bibliography{docRefs}
\end{document}
